\chapter{Advanced Mechanisms}

\label{skill_uses}

\section{Activities}

\begin{multicols}{2}

\togglefalse{examplecharacter}

\subsection{\Glsfmttext{downtime} Activities}
\label{downtimeActions}

\makeRule{poison}{Brewing a poison}{Intelligence}{Medicine}{4}
\index{Poisons}

Basic poisons inflict 2~\glspl{ep} over the course of \pgls{interval}, then stop.
People can notice a poison by smell or taste before they drink it, by rolling \roll{Wits}{Vigilance} (\tn[7]).
Poisons lose their potence over time -- the \gls{ep} loss suffered reduces by to half of what it was each week.

The standard roll begins at \tn[7].
The caster can add any of the following boons to the poison.
However, each boon increases the \gls{tn} by 1.

\begin{itemize}
  \item
  Inflict \pgls{ep} more per \pgls{interval}.
  \item
  Increase the number of \glspl{interval} the poison will work for.
  \item
  Increase the \gls{tn} to notice the poison.
  \item
  Increase the shelf-life of the poison by a week.
\end{itemize}

A poison worthy of \gls{abderian} might inflict 4~\glspl{ep} each \gls{interval}, for 3~\glspl{interval}, and require a \roll{Wits}{Vigilance} roll at \tn[10] but this would raise the \gls{tn} from 7 to 14.

A tie indicates that the poison loses the first boon its creator gave it, and each Failure Margin takes another.
When no boons are left, the poison does nothing but smell bad.

\makeRule{sword_crafting}{Crafting a sword}{Strength}{Crafts}{15}

This requires equipment, such as moulds, and a long night.
It also requires a single level of the Melee Skill.

A tie indicates a completed sword, but a shattered mould.

\makeRule{make_mould}{Creating a weapon mould}{Intelligence}{Crafts}{11}
A weapon mould grants a +2 Bonus to weapon-crafting rolls.
Most metallurgists cannot form weapons without one.

\makeRule{make_armour}{Creating silent plate armour}{Intelligence}{Crafts}{16}
Creating silent plate requires planning from the outset -- existing armour cannot be properly modified.
The parts cost an additional 50\%, and the crafter must have both the Melee and Stealth Skills at level 1.

Every margin on the roll reduces the armour's penalty by 1, to a minimum of -1.

\makeRule{make_song}{Composing a new song}{Intelligence}{Performance}{10}
This won't let the character perform the song -- just compose, and possibly explain it.

Songs can boost or tarnish reputations, granting a Bonus or Penalty to social interactions, equal to half the roll's Margin.

\makeRule{tame_horse}{Taming a Horse}{Intelligence}{Cultivation}{depends on Horse's \roll{Wits}{Brawl}}
Wild horses still run wild in \gls{fenestra}, but catching one without injuring it is difficult, and getting it comfortable with people takes upwards of months.
But if you have more time than money, then it's time well spent.

\iftoggle{stories}{
  \makeRule{make_spell}{Composing a Spell}{Intelligence}{Sphere}{10 plus Level}
  \index{Spells!Creation}
  Crafting new spells requires one week per level of the spell.
  Players can look over the process in the \textit{Book of Stories}, \autopageref{spellWeaving}.
  If the roll fails, then the \gls{gm} will swap one Action or Descriptor for another, and then work through the results, however horrifying.
  The caster cannot use this botched spell -- it only works for that one occasion, by accident.
  On a tie, the research fails to no effect, or the player can choose to have the spell fail and have their character learn the spell.
  The player should not know how the spell will fail beforehand -- the \gls{gm} decides in secret.

  If a spell fails, then the next session should probably open with the aftermath of that failure, and a run-through of all that it entails.
}{}

\columnbreak

\subsection{Journeys Past the \Glsfmttext{edge}}

\makeRule{build_shelter}{Building a shelter}{Strength}{Cultivation}{11}
Each point on the Margin allows an additional person to sleep inside the shelter.
A tie indicates that the shelter holds for \pgls{interval}, then collapses.

\makeRule{climbing}{Climbing}{Speed}{Athletics}{varies greatly}

Rolling a tie indicates that the climber knows they cannot make the climb, or at least fails on the first few holds.
Failure with a Margin of 1 gives the worst possible result -- the climber fails while near the top, while a larger failure Margin means they travelled less far before falling.

\makeAutoRule{dragBody}{Dragging Bodies}{only requires no Penalty beyond -4}
uses the standard \gls{weight} rules, but the character cannot drag anything while they have a Penalty above -4.
This Penalty may come from carrying something too heavy, or from wounds; any Penalty above the limit will stop them dragging anything heavy about.

Rope or other equipment might grant +1 to the effective Strength of the character for the purposes of pulling a massive item or beast.
And many characters can work together, each adding any amount to their total \gls{weight} carried.

\makeRule{fieldDress}{Field Dressing}{Dexterity}{Cultivation}{12}
lets the \gls{guard} keep slain \glspl{monster} in good condition, until they can sell them.
A knife cuts from arse to throat, organs are harvested, \glspl{ingredient} extracted, and the meat hung to dry.

The \gls{tn} raises by +2 in the hot seasons, and reduces by -2 while snow rests on the ground.
On a success, the process takes \pgls{interval}.
On a failure, any \glspl{ingredient} in the \gls{monster}'s body spoil, and the rest halves in value.

\makeRule{ford_river}{Fording a rapid river}{Strength}{Seafaring}{9}
Success gets you to the other side, and failure gets you washed downstream.
A tie gets you a bit of both.

\subsubsection{\Glsfmttext{foraging}}
\glsentrydesc{foraging}

Using \pgls{bandAct} won't do much good unless the group is willing to split up a little (within earshot).
However, if a group can split up entirely, going to different areas, they can each make a separate foraging roll, and gather far more resources.

Once an area has been foraged, it cannot be foraged again during the same season.

\makeRule{navigate_land}{Navigating Land}{Intelligence}{Survival}{by terrain type, Failure Margin adds 2 miles}

\begin{itemize}
  \item
    Mountains are \tn[8].
  \item
    Forests are \tn[11].
  \item
    Marshes are \tn[12].
\end{itemize}

\noindent
Each Failure Margin adds 2 miles to the journey time, so when trying to find a particular house somewhere in a forest, 10 miles away, the \gls{tn} would be 12.
If the roll is an 8, the actual journey would be 18 miles.

\makeRule{scout}{Scouting for an Enemy Camp}{Speed}{Vigilance}{9 plus the enemy's \roll{Wits}{Vigilance}}
A tie indicates someone spotted you before you got away.
Failure indicates not getting away.

\makeRule{start_fire}{Starting a Fire}{Intelligence}{Survival}{10}
A little tinder-box will grant a +2 Bonus, and heavier boxes will allow reuse.
During the cold seasons, fire is a necessity to remove \glspl{ep} (mentioned \vpageref[above]{cold}).

\makeRule{tracking}{Tracking}{Wits}{Survival}{12}
Excessive snow or rain means the \gls{tn} decreases by 2 due to mud or snow remembering every step.
However, it also increases the \gls{tn} by 2 each \gls{interval}.

Both the hunter and the hunted determine their own rate of travel.
The hunter continues making rolls until they catch their quarry or lose them.

\makeRule{whittling}{Whittling Wood}{Dexterity}{Crafts}{10}
Characters can fashion anything with a cost of less than 10~\gls{cp} in less than a day, with only basic woodworking tools.

\subsection{Journeys by Road \& \Glsfmttext{village}}

\makeRule{area_knowledge}{Area knowledge}{Intelligence}{Academics}{set by area}
Cities are \tn[7], Towns are 9, and \glspl{village} are 13.
A successful roll indicates a working knowledge of the place.

\makeRule{makeCamp}{Make Camp}{Intelligence}{Vigilance}{8}
starts with a safety check of the area, during daylight.
That secures enough space for people to wander a short distance to take a piss in the bushes, or just relax for a bit.

\index{Hardened Half-Yurt}
Once night falls, most bed in \pgls{bothy}.
If the \gls{bothy} hasn't the space for the mounts, some traders make space for a horse or two inside their wagons.
Others carry a `hardened half-yurt' -- a large covering, made of leather and wooden latices, which they affix to \pgls{bothy}'s side, to make space for a couple of horses.

\makeAutoRule{march}{Marching}{Move 5 miles per~\glsfmttext{interval}, +1 per \glsentrytext{ep}}
\glsentrydesc{journey}

Humans can endure a hard-march better than most.
Despite their slow gait, they can out-pace almost anything in the long-term.%
\exRef{stories}{Stories}{humanInheritance}

\subsection{Journeys on Water}

\makeRule{mend_sail}{Mending a sail}{Dexterity}{Seafaring}{10}

\makeRule{navigate_ocean}{Navigation open oceans}{Intelligence}{Seafaring}{12}
Each Failure Margin puts the boat off course by 10 miles.
A tie indicates that the navigator knows to remain for a day, and make further observations, rather than push towards an uncertain direction.

\makeRule{swimming}{Swimming}{Speed}{Seafaring}{set by water's speed}
Large rivers might have \tn[8], while an open sea in a storm might be \tn[12].

Characters can swim 1~\gls{step} per \gls{ap} spent.
If they do nothing but swimming, they can add their Athletics Skill to the total at the end of the round.

\subsection{Journeys in the \Glsfmttext{deep}}
\index{Caving|textbf}
\index{Spelunking|see {Caving}}
Caving \glsentrydesc{caving}

\subsubsection{\Glsfmttext{blackWalking}}
\glsentrydesc{blackWalking}

\subsubsection{\Glsfmttext{gagingCave}}
\glsentrydesc{gagingCave}

\subsubsection{\Glsfmttext{echoing}}
\glsentrydesc{echoing}

\subsubsection{\Glsfmttext{hypoxia}}
\glsentrydesc{hypoxia}

\subsubsection{\Glsfmtplural{caveFire}}
\glsentrydesc{caveFire}

\makeRule{run_in_dark}{Running in the Dark}{Wits}{Caving}{12}
\index{Caving!Running in the dark}%
A bright torch adds a +3 Bonus.
A tie means the character comes to a sudden halt, without Damage.
Failure means $1D6$ Damage, plus the character's Speed (running faster means more Damage).

\subsection{News \& Information}

\makeRule{find_rumours}{Finding rumours}{Charisma}{Empathy}{3}
See also, verifying rumours, \vpageref{verify_rumours}.

\makeRule{identify_talisman}{Identifying a \glsfmttext{talisman}}{Intelligence}{Academics}{12, then 14}
Magical items do not darken people's doors often, but once they do, you had best get some educated advice.
The initial roll tells someone how to activate \pgls{talisman}, and which Spheres created it (ties only tell the Spheres).
A second roll, at \tn[14], is required to know what the \gls{talisman} does (ties indicate something about what it does), but not the full picture.

%\subsection{Schemes \& Plots}

\makeRule{find_hidden_message}{Finding a hidden message in a book}{Intelligence}{Vigilance}{7}
The message-hider can set any \gls{tn} they please.

\makeRule{letter_sealing}{Letter sealing}{Dexterity}{Academics}{9}
Proper seals have more than a blob of wax to keep them safe.
Ultra secret letters have parts of the paper cut, then pierce the middle, and loop back around the outside.
While anyone can open these letters, opening them without breaking the seal (so the letter does not appear to have been read) is nearly impossible.
Failure indicates that the letter's seal breaks moments later, as the paper has been cut too thin.
A tie indicates nothing special -- but of course opening the letter won't be quite the challenge it could be.

Opening such a letter and resealing it properly requires an \roll{Intelligence}{Academics} roll, at \tn[14], plus the margin of whoever sealed the letter originally.

\makeRule{verify_rumours}{Verifying rumours}{Intelligence}{Empathy}{13}
Rumours are bad, but someone trying to verify them is worse.
A failure on this roll means that people around have been alerted to the investigation (whether the rumour was true or not).
\index{Investigation}

If the investigator doesn't care about alerting anyone, they can take this as \pgls{restingaction} during \gls{downtime}.

\subsection{Town Activities}

\makeRule{break_door}{Breaking in a door}{Strength}{Crafts}{10}
A tie could indicate that the door has a massive hole in the middle, allowing a sufficiently small person to squeeze through; or it might indicate that the door caves in after multiple, noisy strikes.

\index{Services}
\makeRule{buyService}{Judging services}{Wits}{Empathy}{9}
It's never easy knowing whom to hire.
Every time someone hires someone as part of a service, they should make a roll.

Humans are notoriously bad at this, and are known for hiring the first person they meet in a bar.

Failing the roll means that the \gls{pc} has hired someone useless.
Perhaps they want to work with you because they have no idea how bad they are at their job, or perhaps they simply want to rip you off by taking a guess at the best route and hoping for the best.
The Failure Margin should indicate just how bad the henchman is, so the \gls{gm} should to make the roll in secret.

A tie generally indicates noticing a serious problem with purchased services\ldots just after the purchase completes.

\index{Bards}
Given the stakes, people try to hire others based on previous experience.
Others hope for a nearby bard of some kind -- not just any old minstrel, but someone who has a solid grasp on the reputation of everyone around, from multiple sources, and keeps up to date with all local markets of any size.
The successes and failures of everyone in a marketplace -- from the over-ripe tomatoes they sold, to the sword-smith who makes exceptionally reliable blades -- often become the contents of songs; so anyone listening to a nearby bard can gain a Bonus to their ability to discern reliable services from useless ones.

No Skill could cover this ability.
The bard simply has to exchange gossip at the markets for some months.

\makeRule{pick_pocket}{Picking a pocket}{Dexterity}{Larceny}{12}
Stealing in larger, more populated areas, affords more opportunities, while small \glspl{village}, where everyone is aware of everyone in their personal space, and rarely carry larger sums of money, raises the \gls{tn} significantly.

A tie means the character gets the item, but the victim immediately notices the crime.

\larcenyChart

\makeRule{request_danger_job}{Hiring criminals}{Charisma}{Empathy}{depends on location}
Thieves, brigands, and illegal adventurers cannot work with just anyone who wanders up to ask for `one assassination job, my good man'.
Dangerous jobs require a level of trust.
Charismatic characters who show care and tact stand a much better chance of hiring help.

\sidebox{
  \begin{boxtable}[lY]

    \textbf{Location} & \textbf{Base} \textbf{\glsentrytext{tn}} \\\hline

    City & 9 \\

    Town & 11 \\

    \Gls{village} & 14 \\

  \end{boxtable}
}

Any attempt to hire services which put someone in danger should require a roll.
This includes murder, crafting poisons, selling illegal items, et c.

Working well with someone means that someone can gain a good local reputation (perhaps just among mercenaries, dodgy apothecaries, or librarians), while returning from a job with a missing man means a mark on the \gls{pc}'s reputation.

\makeRule{hookey}{Walking through Town}{Wits}{Empathy}{depends on area, bonus for rank}
The \gls{guard} should guard against the forest, not hide (nor drink) in towns.

\sidebox{
  \begin{boxtable}[cY]

    \textbf{\gls{tn}} & \textbf{Location} \\\hline

    5 & \gls{healersGuild} \\

    6 & Friend's Home \\

    8 & Slums \\

    9 & Tavern \\

    9 & Rich Neighbourhood \\

    10 & Market \\

    12 & \gls{sunGuard} Station \\

  \end{boxtable}
}

Of course people make exceptions for higher-ranking \gls{guard}, and plenty of the guard manage to sneak into town regularly.

\Gls{guard} characters can add any rank they have earned to any rolls.
`Fodder' add nothing, while `Diggers' can add +1 to the roll.

Entering a town requires an \roll{Intelligence}{Larceny} group roll (\tn[7]).
\Glspl{guard} who cannot find alternative clothing to their standard, dark, uniforms, take a -4 penalty to this roll.

A tie indicates that someone has caught the character (see below).
Remaining in town requires a \roll{Wits}{Empathy} roll to go undetected.
Rolling over the \gls{tn} by 1 allows an extra day, then 2 days, then 4, 8, and so on.

\paragraph{Getting caught}
demands an immediate \roll{Wits}{Deceit} roll (\tn[10]).
Failure means a trip to the \gls{court}.
Characters who flee will find their name known across the land, especially to the \gls{sunGuard}, and will find the \gls{warden} taking the failure to appear in their \gls{court} a personal insult.

\end{multicols}


\section{Standards}

\begin{multicols}{2}

\subsection{Patterns in the Rules}

Noticing patterns in the rules can help you to remember them.
Make the following principles a habit, and you'll find your role becomes a lot easier.
And speaking of rolls, let's start with dice stats, and why `7' is the magic number.

\vspace{1\baselineskip}
\noindent
\begin{scriptsize}%
\begin{boxtable}[clXX]

  \hline
  \textbf{Roll} & \textbf{Combinations} & \textbf{Chance} & \textbf{or Greater} \\\hline
  2  & \epsdice{1}\epsdice{1} & 2.78\% & 100\% \\
  3  & \epsdice{1}\epsdice{2} \epsdice{2}\epsdice{1} & 5.56\% & 97.22\% \\
  4  & \epsdice{1}\epsdice{3} \epsdice{3}\epsdice{1} \epsdice{2}\epsdice{2} & 8.33\% & 91.67\% \\
  5  & \epsdice{1}\epsdice{4} \epsdice{4}\epsdice{1} \epsdice{2}\epsdice{3} \epsdice{3}\epsdice{2}  & 11.11\% & 83.33\% \\
  6  & \epsdice{1}\epsdice{5} \epsdice{5}\epsdice{1} \epsdice{2}\epsdice{4} \epsdice{4}\epsdice{2} \epsdice{3}\epsdice{3} & 13.89\% & 72.22\% \\
  7  & \epsdice{1}\epsdice{6} \epsdice{6}\epsdice{1} \epsdice{2}\epsdice{5} \epsdice{5}\epsdice{2} \epsdice{3}\epsdice{4} \epsdice{4}\epsdice{3} & 16.67\% & 58.33\% \\
  8  & \epsdice{2}\epsdice{6} \epsdice{6}\epsdice{2} \epsdice{3}\epsdice{5} \epsdice{5}\epsdice{3} \epsdice{4}\epsdice{4} & 13.89\% & 41.67\% \\
  9  & \epsdice{3}\epsdice{6} \epsdice{6}\epsdice{3} \epsdice{4}\epsdice{5} \epsdice{5}\epsdice{4} & 11.11\% & 27.78\% \\
  10 & \epsdice{4}\epsdice{6} \epsdice{6}\epsdice{4} \epsdice{5}\epsdice{5} & 8.33\% & 16.67\% \\
  11 & \epsdice{5}\epsdice{6} \epsdice{6}\epsdice{5} & 5.56\% & 8.33\% \\
  12 & \epsdice{6}\epsdice{6} & 2.78\% & 2.78\% \\

\end{boxtable}
\end{scriptsize}

\paragraph{Always round half up} -- whether someone is helping another character with half their Bonus, or combat calls for half Damage, or just any time someone divides a number, they round up at 0.5.
One quarter of a +1 bonus is still 0, but half of a +3 bonus is always +2.
Every rule in BIND holds to this pattern, so you will never have to wonder about which results should round up, and which down.
You always round up.

\paragraph{Dangerous actions are not \glspl{restingaction}}
so if someone has to get this spell just right the first time, or judge the chances of a cave-in and commit to a particular tunnel, they do not get a resting action, even if they have a couple of moments (or months) to spare.

If a task must succeed first time, it's not a resting action!

\index{Banding Actions}
\paragraph{It's only a Banding Action when experts usually work together.}
People don't usually build ships alone -- it's far better to divide that work between many people.
On the other hand, poets usually work alone.
Having everyone add another verse is quite possible, but it won't make a poem faster, or better.

\paragraph{When in doubt, set the \glsentrytext{tn} high!}
\Gls{tn} 7 may seem like an standard, but it functions more like a basic number to add to.
A professional \gls{npc} would normally have a Skill at +2, and some relevant Attribute at +1 (at least), along with the Specialist Knack,%
\footnote{See page \pageref{specialist}.}
which grants a +2 bonus.
That leaves professionals with at least a +5 Bonus to do their job.
And if they can take \pgls{restingaction} to do their job, they will roll at least 12 every time.
Therefore, \gls{tn} 12 isn't monstrously high -- it represents a starting figure for basic professionals doing what professionals do.
And if the \emph{average} professional would struggle with a task, then a \gls{tn} of 14 or more fits fine!

\paragraph{The dice tell the story,} but only with interpretation.
A crappy roll to open a door suggests the massive door has wedged properly shut.
A fantastic roll to talk to the local lord might indicate he has family in that character's home \gls{village}.
Explaining dice results can come easier than making up a situation whole-cloth.

\Gls{fenestra} runs on deterministic mechanics -- nothing occurs because of `luck'.
Traits represent reliable elements of the world (such as a character's Strength), while the dice represent unknown elements, such as the wind or the movements of animals.

If you interpret the dice rolls as just how well a character has performed that day, a lot of the system will stop making sense; when one \gls{pc} `just fails' to convince \pgls{warden} to fund their mission, another might step in to `try their luck' (with the dice).
But if the first player to roll understands that the town \gls{warden}'s raging toothache has put him in a foul mood, the rest should understand that the result (or at least the roll) will remain no matter who tries to speak with him.
This leaves room for some other \gls{pc}, with better stats, to succeed in the endeavour (by using the same roll), but does not encourage a ring of players rolling dice like a bunch of bored gamblers.

\end{multicols}

\Needspace{12\baselineskip}
\begin{boxtable}[Lccccccccccc]
  \roll{Speed}{Athletics}, \tn[9] & Fail & Fail & Fail & Fail & Draw & Win & Win & Win & Win & Win & Win \\
  & \rollDice{2} & \rollDice{3} & \rollDice{4} & \rollDice{5} & \rollDice{6} & \rollDice{7} & \rollDice{8} & \rollDice{9} & \rollDice{10} & \rollDice{11} & \rollDice{12} \\
  \roll{Dexterity}{Projectiles}, \tn[10] & Fail & Fail & Fail & Fail & Fail & Fail & Draw & Win & Win & Win & Win \\
\end{boxtable}

\begin{multicols}{2}
\subsubsection{Everything is a Mirror}

\begin{exampletext}
  The guard sprints towards the archer, dodging his arrows.
  With Speed +1 and Athletics +2, she has a +3 bonus.
  The \gls{gm} gives a \gls{tn} of `9'.
\end{exampletext}

\noindent
Here, the \gls{pc} playing the guard can roll a total of 6 numbers which show a win, and 4 which result in failure.

However, from time to time, the \gls{gm} may want to roll dice, either to emphasise an enemy's agency, or to keep players from having to spend too long throwing Mathsrocks across the table.

\begin{exampletext}
  The archer looses another arrow towards the guard.
  With Dexterity +0 and Projectiles +2, he has a +2 bonus.
  The \gls{gm} rolls against a \gls{tn} of `7' plus the player's bonuses, for a total \gls{tn} of `10'.
\end{exampletext}

\noindent
This might look different at a glance, but of course the archer wins on the top 4 numbers, and loses on the lower 6.
Mechanically, the same roll has occurred in each instance.

\end{multicols}

\section{Rolling for the Role}
\label{rollForRoles}

\begin{multicols}{2}

\noindent
It's hard to play `the social character'.
You put all your \glspl{xp} into a high Charisma score because you want to build alliances and understand people, then the \gls{gm} asks you to roleplay such an encounter and your natural stutter and slow wit replace the social graces your character should have.

It's also hard playing a non-social character.
You have been lumped with a character with a Charisma Penalty of -4 and by all the gods you intend to roleplay them, so it's time to ask the \gls{warden} which lady he stole his robe from and then wipe your mouth with the tablecloth.
But the other players are not impressed; all they can see is someone intentionally ruining the encounter rather than your fun-loving improvisations.

Consider the following solution: when players want their character to speak, the \gls{gm} tells them to roll \roll{Charisma}{Empathy} or \roll{Wits}{Whatever}, then sets the \gls{tn} for the encounter.
The dice determine the results, but the player \emph{interprets} the dice.

\subsubsection{Interpretation over Performance}

When players roll high on a social encounter, they get to set the scene.
They may take it as an opportunity for a grand speech, or may prefer to simply describe loosely how the event goes, and leave it at that.

Asking for `roleplaying' in order to make an encounter go smoothly tells players never to interpret failures, which is a great loss.

\begin{speechtext}
  Rolling a total of `1' won't get you into the keep.
  What exactly does Corbelch say to the guards?
\end{speechtext}
This method of players rolling before roleplaying to indicate their roll gives value to the social characters' Traits and legitimacy to the antics of more socially clumsy players saying all the wrong things.
The roll of the dice also acts as a way of saying `I am about to speak', so people can pace conversation without interruption.

\end{multicols}

\pagebreak

\section*{The Right to Improve}

\begin{multicols}{2}

\noindent
This book has problems, and that's fine.
I've put this under a share-alike licence,\footnote{\tt GNU General Public License 3 or (at your option) any later version.} so anyone can grab a copy of the basic \LaTeX~ document it's written in and make improvements.
This isn't the Open Gaming Licence of D20 where they magnanimously allow you to use their word for a mechanic and let you publish things for their products -- this is a publicly owned book.

No longer do imaginative Games Masters have to scribble their inspired house rules onto the back of an old banking statement and Sellotape it to the last page of the core book.
Instead, you have the complete source documents, and can modify it as you please, creating a cohesive book.
If you spot an error, you can correct it.
If you want to add a couple of spells, it's no problem.
Just copy the \href{https://gitlab.com/bindrpg/core}{source files}, download a \LaTeX~ editor, and make the changes you want.
Once you're happy with your changes, you might even send it off to a printing shop for a copy of your own version.

And if you happen to make some useful additions, or even deletions, be sure to add them to another git project, where others can benefit from your genius.

With a little work, we could get a real community-based RPG.
Something that's always free, something that gets a new edition as and when people want, with just the changes that people want -- a continuously evolving work.

This particular version was last revised on \today.

\end{multicols}
