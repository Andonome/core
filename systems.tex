\chapter{Advanced Mechanisms}

\label{skill_uses}

\section{Activities}

\begin{multicols}{2}

\subsection{Breaking Things}

\paragraph{Breaking in a door} -- \roll{Strength}{Crafts}, \tn[10].
A tie could indicate that the door has a massive hole in the middle, and a broken lock, allowing a sufficiently small person to squeeze through.

\paragraph{Excavation} -- \roll{Strength}{Caving}.
The \gls{tn} varies greatly, depending upon the type of rock.

\subsection{Downtime Activities}

\paragraph{Brewing a poison} -- \roll{Intelligence}{Medicine}, \tn[4].
\label{poison}\index{Poisons}

Basic poisons inflict 2 \glspl{fatigue} over the course of \pgls{interval}, then stop.
People can notice a poison by smell or taste before they drink it, by rolling \roll{Wits}{Vigilance} (\tn[7]).
Poisons lose their potence over time -- the \gls{fatigue} loss suffered reduces by to half of what it was each week.

The standard roll begins at \tn[7].
The caster can add any of the following boons to the poison.
However, each boon increases the \gls{tn} by 1.

\begin{itemize}
  \item
  Inflict 1 more \gls{fatigue} on the target over \pgls{interval}.
  \item
  Increase the number of \glspl{interval} the poison will work for.
  \item
  Increase the \gls{tn} to notice the poison.
  \item
  Increase the shelf-life of the poison by a week.
\end{itemize}

A poison might inflict 4 \glspl{fatigue} each \gls{interval}, for 3 \glspl{interval}, and require a \roll{Wits}{Vigilance} roll at \tn[10] but this would raise the \gls{tn} from 7 to 14.

\paragraph{Crafting a sword} -- \roll{Strength}{Crafts}, \tn[10] plus the weapon's \gls{weight}.
This requires equipment, such as moulds, and a long night.
It also requires a single level of the Combat Skill.

A tie could indicate a completed sword, with a shattered mould.

\paragraph{Creating a weapon mould} -- \roll{Intelligence}{Crafts}, \tn[7] plus the weapon's total bonuses.

Anything with a cost of less than 10 \gls{cp} can be fashioned in less than a day, with only basic woodworking tools.

\paragraph{Creating quiet, full plate armour} -- \roll{Intelligence}{Crafts}, \tn[15].
Moulding silent plate requires planning from the outset -- existing armour cannot be properly modified.
The parts cost an additional 50\%, and the crafter must have both the Combat and Stealth Skills at level 1.

Every margin on the roll reduces the armour's penalty by 1, to a minimum of -1.

\paragraph{Composing a new song} -- \roll{Intelligence}{Performance}, \tn[10].

\paragraph{Taming a Horse} -- \roll{Intelligence}{Wyldcrafting} vs Horse's \roll{Wits}{Brawl}.

\subsection{News \& Information}

\paragraph{Finding rumours} -- \roll{Charisma}{Empathy}, \tn[3].

\label{magicidentification}
\paragraph{Identifying a famous magical item} -- \roll{Intelligence}{Academics}, \tn[12].
Magical items do not darken people's doors often, but once they do, you had best get some educated advice.

\subsection{Schemes \& Plots}

\paragraph{Finding a hidden message in a book} -- \roll{Intelligence}{Vigilance} \tn[7].

The message-hider can set any \gls{tn}.

\paragraph{Letter sealing} -- \roll{Dexterity}{Academics}, \tn[9].
\label{letterSealing}
Proper seals have more than a blob of wax to keep them safe.
Ultra secret letters have parts of the paper cut, then pierce the middle, and loop back around the outside.
While anyone can open these letters, opening them without breaking the seal (so the letter does not appear to have been read) is nearly impossible.
Failure indicates that the letter's seal breaks moments later, as the paper has been cut too thin.
A tie indicates nothing special -- but of course opening the letter won't be quite the challenge it could be.

Opening such a letter and resealing it properly requires an \roll{Intelligence}{Academics} roll, at \tn[14], plus the margin of whoever sealed the letter originally.

\subsection{Survival}

\paragraph{Black-Walking} -- \roll{Dexterity}{Caving}, \tn[8].
Despite every caver insisting on a good supply of torches and candles, they all end up feeling the ground in the dark eventually.
Those who know their environment have a knack for crawling efficiently, feeling the surroundings through their fingertips and beards, and remembering every passage they took in the light simply through the sounds of their own breathing echoing uniquely in every cavern-segment.

\paragraph{Building a shelter} -- \roll{Intelligence}{Wyldcrafting}, \tn[11].
Each point on the Margin allows an additional person to sleep inside the shelter.

A tie indicates that the shelter holds for a few hours, then collapses.

\paragraph{Curing a poison} -- \roll{Wits}{Medicine}, \tn[10].

Each Margin cures 1 \glspl{fatigue} caused by poison by the end of the interval.
Of course if the roll fails, each failure margin \emph{inflicts} a \gls{fatigue}.

\subsection{Town Activities}

\paragraph{Judging services} -- \roll{Wits}{Empathy}, \tn[9].

It's never easy knowing whom to hire.
Every time someone hires someone as part of a service, they should make a roll.

Humans are notoriously bad at this, and are known for hiring the first person they meet in a bar.

Failing the roll means that the \gls{pc} has hired someone useless.
Perhaps they want to work with you because they have no idea how bad they are at their job, or perhaps they simply want to rip you off by taking a guess at the best route and hoping for the best.
The Failure Margin should indicate just how bad the henchman is, so the \gls{gm} is encouraged to make the roll in secret.

Given the stakes, people mostly try to hire others based on previous experience.

A tie generally indicates noticing a serious problem with purchased services\ldots just after the purchase completes.

\paragraph{Picking a pocket} -- \roll{Dexterity}{Larceny}, \tn[12] plus the target's \roll{Wits}{Vigilance}.
\index{Pickpocketing}

Stealing in larger, more populated areas, affords many more opportunities, while small villages, where everyone is aware of everyone in their personal space, and rarely carry larger sums of money, raise the \gls{tn} significantly.

A tie means the character gets the item, but the victim immediately notices the crime.

\larcenyChart

\paragraph{Requesting dangerous jobs} -- \roll{Charisma}{Empathy}.

\sidebox{
  \begin{boxtable}[lc]

    \textbf{Location} & \textbf{Base} \textbf{\glsentrytext{tn}} \\\hline

    City & 9 \\

    Town & 11 \\

    Village & 14 \\

  \end{boxtable}
}

Thieves, brigands, and illegal adventurers cannot work with just anyone who wanders up to ask for `one poison arrow, my good man'.
Dangerous jobs require a level of trust.
Charismatic characters who show care and understanding stand a much better chance of hiring help.

Any attempt to hire services which put someone in danger should require a roll.
This includes murder, crafting poisons, selling illegal items, et c.

Working well with someone means that someone can gain a good local reputation (perhaps just among mercenaries, dodgy apothecaries, or librarians), while returning from a job with a missing man means a mark on the \gls{pc}'s reputation.

\subsection{Travel}

\paragraph{Area knowledge } -- \roll{Intelligence}{Academics}.
The character recalls local information about important sites.
Cities are \tn[7], Towns are 9, and villages are 13.

\paragraph{Fording a rapid river} -- \roll{Strength}{Seafaring}, \tn[9].

\index{Gathering Food}\index{Food}
\paragraph{Gathering Food} -- \roll{Wits}{Wyldcrafting}.
Groups can forage while on the road, but taking a resting action requires devoting a full segment of the day to focussing on foraging (see \autopageref{intervals}).
Of course, these fast excursions from the path, to check out anything that happens to catch their eye, can lead to quick decisions, or even to encounters with wandering beasts.

\gatheringChart

\paragraph{Keeping watch over the camp through the night} -- \roll{Strength}{Vigilance}, \tn[7].
Success inflicts 2 \glspl{fatigue} on the person keeping watch.

\paragraph{Climbing} -- \roll{Speed}{Athletics}.

\paragraph{Detect sloping passages} -- \roll{Wits}{Caving}.
Understanding what altitude one has reached immediately indicates whether there might be running water, what type of rocks and minerals compose the surroundings (and therefore the chance of a cave-in), and how far one has to go to the surface.

Despite gradual gradients, or sharp ups and downs, a good caver knows exactly how far they sit from the surface at all times.

Rolling a tie might indicate knowing that one has descended or ascended, but with no idea how much.

\paragraph{Detecting tunnel weaknesses} -- \roll{Intelligence}{Caving}, \tn[9].
Nobody survives long underground unless they can tell if the ceiling might collapse from heavy footfall.

\paragraph{Mending a sail} -- \roll{Dexterity}{Seafaring}.

\paragraph{Navigation open oceans} -- \roll{Intelligence}{Seafaring}, \tn[10].
\index{Navigation}
\index{Sailing}

\paragraph{Navigating on land} -- \roll{Intelligence}{Wyldcrafting}.
\index{Navigation}
\index{Marching}
\label{marching}
\begin{itemize}

  \item
    Mountains are \tn[8].
  \item
    Forests are \tn[11].
  \item
    Marshes are \tn[12].

\end{itemize}

Each failure margin adds 2 Miles to the journey time, so when trying to find a particular house somewhere in a forest, 10 miles away, the \gls{tn} would be 12.
If the roll is an 8, the actual journey would be 18 miles.


\paragraph{Placing fires} -- \roll{Intelligence}{Caving}, \tn[8].
A fire in the wrong place underground can easily choke everyone around to death, or at least until they can't think properly.
Of course, this provides an excellent weapon of war if one can do it properly.
Light the wrong type of fire, and heavy smoke will fall down a tunnel instead of rising.

\subsection{War \& Battery}

\paragraph{Calm an animal} -- \roll{Intelligence}{Wyldcrafting} vs animal's \roll{Wits}{Brawl}.

\paragraph{Intimidating someone into backing off} -- \roll{Strength}{Deceit} vs the target's \roll{Strength}{Empathy}.
\index{Intimidation}

\paragraph{Planning a hidden route into a castle} -- \roll{Intelligence}{Stealth}.

\paragraph{Scouting the forest for an enemy camp nearby} -- \roll{Speed}{Vigilance}, \tn[9] plus the enemy's \roll{Wits}{Vigilance}.
A tie indicates someone spotted you before you got away.
Failure indicates not getting away.

\end{multicols}

\section{Standards}

\begin{multicols}{2}

\subsection{Patterns in the Rules}

Noticing patterns in the rules can help you to remember them.
Make the following principles a habit, and you'll find your role becomes a lot easier.
And speaking of rolls, let's start with dice stats, and why `7' is the magic number.

\vspace{10pt}
\noindent
\begin{scriptsize}%
\begin{boxtable}[clXX]

  \hline
  \textbf{Roll} & \textbf{Combinations} & \textbf{Chance} & \textbf{or Greater} \\\hline
  2  & \epsdice{1}\epsdice{1} & 2.78\% & 100\% \\
  3  & \epsdice{1}\epsdice{2} \epsdice{2}\epsdice{1} & 5.56\% & 97.22\% \\
  4  & \epsdice{1}\epsdice{3} \epsdice{3}\epsdice{1} \epsdice{2}\epsdice{2} & 8.33\% & 91.67\% \\
  5  & \epsdice{1}\epsdice{4} \epsdice{4}\epsdice{1} \epsdice{2}\epsdice{3} \epsdice{3}\epsdice{2}  & 11.11\% & 83.33\% \\
  6  & \epsdice{1}\epsdice{5} \epsdice{5}\epsdice{1} \epsdice{2}\epsdice{4} \epsdice{4}\epsdice{2} \epsdice{3}\epsdice{3} & 13.89\% & 72.22\% \\
  7  & \epsdice{1}\epsdice{6} \epsdice{6}\epsdice{1} \epsdice{2}\epsdice{5} \epsdice{5}\epsdice{2} \epsdice{3}\epsdice{4} \epsdice{4}\epsdice{3} & 16.67\% & 58.33\% \\
  8  & \epsdice{2}\epsdice{6} \epsdice{6}\epsdice{2} \epsdice{3}\epsdice{5} \epsdice{5}\epsdice{3} \epsdice{4}\epsdice{4} & 13.89\% & 41.67\% \\
  9  & \epsdice{3}\epsdice{6} \epsdice{6}\epsdice{3} \epsdice{4}\epsdice{5} \epsdice{5}\epsdice{4} & 11.11\% & 27.78\% \\
  10 & \epsdice{4}\epsdice{6} \epsdice{6}\epsdice{4} \epsdice{5}\epsdice{5} & 8.33\% & 16.67\% \\
  11 & \epsdice{5}\epsdice{6} \epsdice{6}\epsdice{5} & 5.56\% & 8.33\% \\
  12 & \epsdice{6}\epsdice{6} & 2.78\% & 2.78\% \\

\end{boxtable}
\end{scriptsize}

\paragraph{Always round up} -- whether someone is helping another character with half their Bonus, or combat calls for half Damage, or just any time someone divides a number, they round up at 0.5.
One quarter of a +1 bonus is still 0, but half of a +3 bonus is always +2.

Every rule in the book keeps to this pattern, so you will never have to wonder about which rules round up, and which down.
You always round up.

\paragraph{Dangerous actions are not Resting actions}
so if someone has to get this spell just right the first time, or judge the chances of a cave-in and commit to a particular tunnel, they do not get a resting action, even if they have a couple of moments (or months) to spare.

If a task must succeed first time, it's not a resting action!

\paragraph{It's only a Team Roll when working together helps.}
Writing as a group might seem fun, but it won't always help, so writing a play would not normally allow for a Team Roll.
And if the players ask to make a team roll to craft a fantastic statue, reply `no'.
Master carvers don't ask for help chiselling their statues, so the roll has to be a Group Roll, i.e. the lowest score can drag everyone down.

Conversely, anyone building a basic raft would welcome all the help they can get.
This shows that the group should be allowed a Team Roll.

\paragraph{When in doubt, set the \glsentrytext{tn} high!}
\Gls{tn} 7 may seem like an standard, but it functions more like a basic number to add to.
A professional \gls{npc} would normally have a Skill at +2, and some relevant Attribute at +1 (at least), along with the Specialist Knack,%
\footnote{See page \pageref{specialist}.}
which grants a +2 bonus.
That leaves professionals with at least a +5 Bonus to do their job.
And if they can take a Resting Action to do their job, they will roll at least 12 every time.
Therefore, \gls{tn} 12 isn't monstrously high -- it represents a starting figure for basic professionals doing what professionals do.
And if the \emph{average} professional would struggle with a task, then a \gls{tn} of 14 or more fits fine!

\paragraph{The dice tell the story,} but only with interpretation.
A crappy roll to open a door suggests the massive door has wedged properly shut.
A fantastic roll to talk to the local lord might indicate he has family in that character's home village.
Explaining dice results can come easier than making up a situation whole-cloth.

The world of the game runs on completely deterministic mechanics -- nothing occurs because of `luck'.
Traits represent reliable elements of the world (such as a character's Strength), while the dice represent unknown elements, such as the wind or the movements of animals.

If you interpret the dice rolls as just how well a character has performed that day, a lot of the system will stop making sense; when one \gls{pc} `just fails' to convince a warden to fund their mission, another might step in to `try their luck' (with the dice).
But if the first player to roll understands that the town warden's raging toothache has put him in a foul mood, the rest should understand that the result (or at least the roll) will remain no matter who tries to speak with him.
This leaves room for some other \gls{pc}, with better stats, to succeed in the endeavour (by using the same roll), but does not encourage a ring of players rolling dice like a bunch of bored gamblers.

\subsubsection{Everything is a Mirror}

\begin{exampletext}
  The guard sprints towards the archer, dodging his arrows.
  With Speed +1 and Athletics +2, she has a +3 bonus.
  The \gls{gm} gives a \gls{tn} of `9'.
\end{exampletext}

\noindent
Here, the \gls{pc} playing the guard can roll a total of 6 numbers which show a win, and 4 which result in failure.

However, from time to time, the \gls{gm} may want to roll dice, either to emphasise an enemy's agency, or to keep players from having to spend too long throwing Mathsrocks across the table.

\begin{exampletext}
  The archer looses another arrow towards the guard.
  With Dexterity +0 and Projectiles +2, he has a +2 bonus.
  The \gls{gm} rolls against a \gls{tn} of `7' plus the player's bonuses, for a total \gls{tn} of `10'.
\end{exampletext}

\noindent
This might look different at a glance, but of course the archer wins on the top 4 numbers, and loses on the lower 6.
Mechanically, the same roll has occurred in each instance.

\end{multicols}

\begin{boxtable}[Lccccccccccc]
  \roll{Speed}{Athletics}, \tn[9] & Fail & Fail & Fail & Fail & Draw & Win & Win & Win & Win & Win & Win \\
  & \rollDice{2} & \rollDice{3} & \rollDice{4} & \rollDice{5} & \rollDice{6} & \rollDice{7} & \rollDice{8} & \rollDice{9} & \rollDice{10} & \rollDice{11} & \rollDice{12} \\
  \roll{Dexterity}{Projectiles}, \tn[10] & Fail & Fail & Fail & Fail & Fail & Fail & Draw & Win & Win & Win & Win \\
\end{boxtable}

\section[Social Rolls]{Roll Before You Roleplay}

\begin{multicols}{2}

\noindent
It's hard to play `the social character'.
You put all your \glspl{xp} into a high Charisma score because you want to build alliances and understand people, then the \gls{gm} asks you to roleplay such an encounter and your natural stutter and slow wit replace the social graces your character should have.

It's also hard playing a non-social character.
You have been lumped with a character with a Charisma Penalty of -4 and by all the gods you intend to roleplay it, so it's time to ask the town warden which lady he stole his robe from and then wipe your mouth with the tablecloth.
But the other players are not impressed; all they can see is someone intentionally ruining the encounter rather than the fun-loving, amazing improviser that you are.

Consider the following solution: tell the players that if they wish to speak, they must roll \roll{Charisma}{Empathy} or \roll{Wits}{Whatever}, then set the \gls{tn} for the encounter.

\subsubsection{Interpretation over Performance}

When players roll high on a social encounter, they get to set the scene.
They may take it as an opportunity for a grand speech, or may prefer to simply describe loosely how the event goes, and leave it at that.

Asking for `roleplaying' in order to make an encounter go smoothly tells players never to interpret failures, which is a great loss.

\begin{quotation}
  Rolling a total of `1' won't get you into the keep.
  What exactly does Corbelch say to the guards?
\end{quotation}

This method of players rolling before roleplaying to indicate their roll gives value to the social characters' Traits and legitimacy to the antics of more socially clumsy players saying all the wrong things.
The roll of the dice also acts as a way of saying `I am about to speak', so people can pace conversation without interruption.

\end{multicols}

\pagebreak
\section*{The Right to Improve}

\begin{multicols}{2}
\noindent
This book has problems, and that's fine.
I've put this under a share-alike licence,\footnote{\tt GNU General Public License 3 or (at your option) any later version.} so anyone can grab a copy of the basic \LaTeX~ document it's written in and make improvements.
This isn't the Open Gaming Licence of D20 where they magnanimously allow you to use their word for a mechanic and let you publish things for their products -- this is a publicly owned book.

No longer do imaginative Games Masters have to scribble their inspired house rules onto the back of an old banking statement and Sellotape it to the last page of the core book.
Instead, you have the complete source documents, and can modify it as you please, creating a cohesive book.
If you spot an error, you can correct it.
If you want to add a couple of spells, it's no problem.
Just download the source from gitlab.com/bindrpg/, download a \LaTeX~ editor, and make the changes you want.
Once you're happy with your changes, you might even send it off to a printing shop for a copy of your own version.

And if you happen to make some useful additions, or even deletions, be sure to add them to another git project, where others can benefit from your genius.

With a little work, we could get a real community-based RPG.
Something that's always free, something that gets a new edition as and when people want, with just the changes that people want -- a continuously evolving work.

This particular version was last revised on \today.

\end{multicols}
