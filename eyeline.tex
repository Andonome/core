\section*{Shifting the Eye-Line}

\begin{multicols}{2}

\noindent
\Gls{fenestra} does not sit well with our modern instincts.
We all understand that aeroplanes and cows have irrevocably changed the planet for the worse.
But in \gls{fenestra}, people fight against a cruel nature.
They burn the forest while celebrating every toppled tree.
They despise their own world.

And so naturally, the perspective-switch which \gls{fenestra} demands can feel unclean.
But you have to understand, their world feels very different.

We created cows through early genetic manipulation of the original animal -- the auroch.
By the 13th century, aurochs became extinct.

When people travelled during this era, they would only fear other humans.
They hated wolves for stealing their food, and in some parts of the planet, a few animals had the muscle and teeth to kill a few humans here and there\ldots at least those not paying attention, but no animal has every posed a real danger to humanity on the same level as diseases, starvation, or warfare.
No city ever disappeared because of large predators eating everyone.
Humanity has wandered the earth, fearlessly, selecting the best plots of land they could to grow all the food they wished, and fence in the meat.

Disease, starvation and war don't trouble the people in \gls{fenestra} much.
They have been replaced with the basilisk, chitincrawler, and woodspy.
Just as Europeans once considered the black death, and other plagues, a standard way to die, any market place in \gls{fenestra} will have updates on what ate whom recently.
This leaves that world very similar to ours on-balance, but very strange in all the details.

\begin{exampletext}
  A young man, who does random jobs to get by, with little pay, must dig a grave tonight.
  The town has no full-time grave-diggers, because most people do not leave a corpse.
\end{exampletext}

\begin{exampletext}
  Two men in a bar debate the usefulness of wearing a helmet.
  The first details the standard attacks from every predator, which mostly involve ambushes, and grabbing.

  \begin{speechtext}
    ``Helmets stop you spotting an attack.
    And they don't help when something grabs you.

    In fact, they make everything worse in every way, defending only against bandits.
    But bandits don't want to eat you, and they're much more timid than real predators, so you can pay them off if you need to.
    \emph{And} the helmet still stops you spotting the ambush.
  \end{speechtext}

  The interlocutor has a few good points on the other side, so the debate continues -- should soldiers wear helmets?
\end{exampletext}

\begin{exampletext}
  The newest recruit to the \gls{templeOfBeasts} arrives, tired, hungry, and pregnant.
  Not long before, most people who became pregnant would receive some dispensation to take time off, or would receive permission to work as \pgls{jotter} instead of fighting at the \gls{edge}.
  However, this lead to an explosion of pregnancies within the \gls{guard}, which lead to a group of the hardest bastards to walk \gls{fenestra}'s dark roads.

  Nobody wants any more of \emph{that lot}, so instead of a safe desk-job, pregnant women now receive a spear or crossbow (depending on supplies).
\end{exampletext}

\begin{exampletext}
  A shepherd walks out with his tiny flock.
  Not many can survive on the small grassland which surround the \gls{village}, but enough to make a daily walk worth the effort.

  Archers stand watch from the \gls{village} walls, waiting for hours as the sheep slowly graze.

  No matter what the traders say, shepherds consider their lives more valuable than any of their sheep, so all shepherds make sure to keep at least one lame sheep within the flock.
  They break a leg or two, and make sure the break never completely heals.
  As a result, the sheep limps slow enough for the shepherd to out-run it, and if anything comes out of the forest, looking for a meal, they can take the lame sheep while the shepherd runs.
\end{exampletext}

\begin{exampletext}
  The young \glspl{doula} make their way to the island -- a paradise where people can walk freely, if they walk in groups of a dozen or so.

  Whenever someone spots a predator, the elders call for a general hunt until they slay the beast.
  They dig up Mouthdiggers, burn chitincrawler eggs, and lay traps for woodspies.
  Basilisks do not swim to the island, and griffins rarely fly this far out.

  The island remains blissfully free, most of the year, leaving the inhabitants to walk freely.

  The elders have forbidden raising children anywhere on the island -- a lesson hard-learned after generations of mishaps.
  Those who begin life on the island remain free forever, but walking free on \gls{fenestra}'s mainland means death.
  As a result, those few surviving, carefree, people who grew up in paradise never leave.
\end{exampletext}

\end{multicols}
