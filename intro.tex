\chapter*{Introduction}

\section*{Overview}

\begin{multicols}{2}

\noindent
BIND\footnote{`BIND' stands for `BIND is not D\&D'.} is a zero to hero RPG designed to tell stories about a team traversing a dangerous and fantastic landscape while developing their skills.
Thematically, BIND stands on the darker side of fantasy, with no possibility to heal damage through magic.

The rules have an emphasis on getting an output quickly, and keeping players' decisions in the loop.

\subsection*{Character Creation}

Players roll up random characters, then interpret what those Attributes mean.
What kind of gnoll is intelligent yet clumsy?
What kind of dwarf is slow to run, but thinks fast?

As the game progresses, players can spend \glspl{storypoint} to create a backstory and put it into play.
Perhaps the warrior has friends from his days in the army, or a gnomish character feels jealous of his big sister's magical talents as she swoops in to save the party.

Once players spend all their \glspl{storypoint}, they should have a beefy backstory without the need for homework.

\subsection*{Combat}

When fighting starts, players receive a number of `Initiative Points' to spend, giving them real choices.
They can focus on damage, defence, guarding allies, or moving to a safer location.

Fights typically end quickly, as nobody has many \glsentrylongpl{hp}.
\Glsentrylongpl{pc} can survive longer due to a limited store of luck, but once their luck runs out they suffer real wounds.
Their luck returns, but the wounds remain for the remainder of the night.

\subsection*{The Right to Improve}

\noindent
This book has problems, and that's fine.
I've put this under a share-alike licence,\footnote{\tt GNU General Public License 3 or (at your option) any later version.} so anyone can grab a copy of the basic \LaTeX~ document it's written in and make improvements.
This isn't the Open Gaming Licence of D20 where they magnanimously allow you to use their word for a mechanic and let you publish things for their products -- this is a publicly owned book.

No longer do imaginative \acrshortpl{gm} have to scribble their inspired house rules onto the back of an old banking statement and Cellotape it to the last page of the core book.
Instead, you have the complete source documents, and can modify it as you please, creating a cohesive book.
If you spot an error, you can correct it.
If you want to add a couple of spells, it's no problem.
Just download the source from gitlab.com/bindrpg/, download a \LaTeX~ editor, and make the changes you want.
Once you're happy with your changes, you might even send it off to a printing shop for a copy of your own version.

And if you happen to make some useful additions, or even deletions, be sure to add them to another git project, where others can benefit from your genius.

With a little work, we could get a real community-based RPG.
Something that's always free, something that gets a new edition as and when people want, with just the changes that people want -- a continuously evolving work.

This particular version was last revised on \today.

\subsection*{Further Reading}

If you're looking for a pre made campaign world, monsters, and stories to tell, find yourself a copy of \textit{Adventures in Fenestra}.

\end{multicols}

\section*{Special Thanks \ldots}

\begin{multicols}{2}

\subsection*{to the Artists}

Neil McDonnell for the basic photograph which became the Polymorph image on page \pageref{roch:polymorph},

\paragraph{Boris Pecikozi\'c} for the example-story images, (pages 
\pageref{boris:jump}, 
\pageref{boris:brawl}, 
\pageref{boris:meet}), 

\paragraph{Roch Hercka} for the myriad wonderful pencil sketches (pages 
\pageref{roch:races}, 
\pageref{roch:dwarf}, 
\pageref{roch:stances}, 
\pageref{roch:vitals}, 
\pageref{roch:xp1}, 
\pageref{roch:xp2}, 
\pageref{roch:enchanter}, 
\pageref{roch:invocation}, 
\pageref{roch:polymorph}, 
\pageref{roch:runes}, 
\pageref{roch:trogdor}, 
\pageref{roch:light}
).
Find him at artstation.com/hertz.


\paragraph{Vladimir Arabadzhi,}
for the stories image (page \pageref{vlad:stories}).

\paragraph{Studio DA}
for the fiery polymorph image,
(page \pageref{da:fire}%
).

\subsection*{and to the playtesters} Marri Russell, Ross Oliver, Reiss McGee, David Smith, Michael Dyson, Ryan Trotter, Maggie Anderson, 
D\'{o}nal Emerson, Christopher Taylor, June Strang, 
Aleksej, Mihailo, and Proxy;
also thanks to Ari-Matti Piippo and \href{https://www.twitter.com/AliceICecile}{Alice I. Cecile} for their insightful comments,
and Florent Rougon for inspiration on the box-lines code.

\iftoggle{verbose}{%
	\columnbreak
	\begin{exampletext}

The skinny man greets Sean with overbearing enthusiasm as he continues to explained the mission.

``\emph{The book was stolen from our library}'' emphasizing ``our'' to make it obvious that it was as much his library as any other wizard's.
``\emph{It is very dangerous and we must have it back.
It contains a song - a bad one}''.

Sean pulls his face to its own centre for a moment. ``\emph{You mean, you think the song in the book is awful?}''

``\emph{No no no. I mean yes}'', the wizard replied, as happily as ever.
``\emph{The book contains a song, the song contains the magic.
When you play or sing it or whatever it is, things happen.
Bad things}''.

``\emph{Okay.
What kinds of things?}''

``\emph{That's a guild secret I'm afraid, but the important thing to know is to never let him sing.}''

``\emph{Might he do that while we're charging towards him with swords and rope?}'', Sean asks.

``\emph{Oh yes}'', the wizard grinned wider.
``\emph{After all, he is a bard.
We allowed him into the college to show off his odd abilities - those sorcerer powers from his elven heritage.
And he stole our book, from the secret section at the back with all the forbidden books.
He must have stolen the key from me.
Anyway -- we can pay handsomely.
Perhaps two hundred gold in total.
Do you think your friends would be interested?}''

``\emph{I'm going to speak with my guys, but two hundred gold for a apprehending a single criminal? Easiest job we've ever had.}''

The wizard smiled again.

\end{exampletext}

}{}

\end{multicols}

