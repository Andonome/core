\chapter*{Introduction}

\begin{multicols}{2}

\subsection*{Overview}

\begin{itemize}
  \item
  BIND\footnote{`BIND' stands for `BIND is not D\&D'.} was an attempt to fix D\&D which got out of control.
  \item
  BIND is yet another fantasy RPG about killing ogres for gold pieces.
  \item
  BIND mechanics force a fast real-world resolution to every encounter.
  \item
  \ifnum\day=17
    BIND was made with too much hot sauce.
    \emph{I'm sorry, the bottle slipped!}
  \else
  BIND is an `adult game', meaning we all have jobs and such, so the game should make as few demands on everyone's time as possible.
  \fi
  \item
  BIND has no house rules, and never will.
  Everyone has complete access to the files so you can just change the rules, and re-print the \gls{pdf}.
  \glsadd{pdf}
\end{itemize}

\subsection*{Context}

BIND splits information in a slightly different way to other RPGs.

A skeleton of the core rules exist in every adventure module.
This rule book contains extended systems which either explain the core rules with examples, and show using them for various sub-systems (in fact perhaps it should be called `BIND: Systems \& Explanations', but that's less catchy than `BIND: Core').

The adventure modules also contain their own price-lists in the form of handouts, so players can just take the page they want, on its own.
This stops my table's problem of players persistently figuring out how to buy weapons in the core rules (`is it under ``equipment'', or ``combat'', where the weapon tables are?').
Perhaps it solves the same problem at your table.

On a similar note, everything that players repeatedly try to reference sits in the \textit{Book of Stories}.
It has the rules on character creation, improving Traits, crafting backstories, and creating magical items and new spells.

Finally, the \textit{Book of Judgement} provides a high-level view of \gls{fenestra}'s guilds, beasts, and fiends; world-building material, and so many random tables that the game practically plays itself.

\subsection*{Special Thanks \ldots}

\subsubsection*{to the Artists}

\paragraph{Neil McDonnell} for the basic photograph which became the Polymorph image on page \pageref{Roch_Hercka/polymorph},

\paragraph{Roch Hercka} for the many wonderful pencil sketches (pages 
\pageref{Roch_Hercka/dwarf_encumbrance}, 
\pageref{Roch_Hercka/cave_fight}, 
\pageref{Roch_Hercka/stances}, 
\pageref{Roch_Hercka/vitals_shot}, 
\pageref{Roch_Hercka/conjuration_right}, 
\pageref{Roch_Hercka/polymorph}, 
\pageref{Roch_Hercka/dwarvish_runes}, 
\pageref{Roch_Hercka/flashing_light}).
Find him at artstation.com/hertz.

\subsubsection*{and to the playtesters} Marri Russell, Ross Oliver, Reiss McGee, David Smith, Michael Dyson, Ryan Trotter, Maggie Anderson, 
D\'{o}nal Emerson, Christopher Taylor, June Strang, 
Necro (Bojan), Jigzo, Nibbly, Reavy, 
Aleksej, Mihailo, and Proxy,
Matija, Michael, Milijana, Niki, Nina, Ogie\'n,
and Ross.

\subsubsection*{Others}
Thanks to Ari-Matti Piippo and Alice I. Cecile for their insightful comments,
Irina for the intro,
and Florent Rougon for inspiration on the box-lines code.

\subsection*{Licence}

BIND is open source, and available under the {\tt GNU General Public License 3} or (at your option) any later version.

You have full access to all the \href{https://gitlab.com/bindrpg/core}{source files}, including art, and the right to change anything and share those changes with others.

\end{multicols}
