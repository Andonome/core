\chapter{Measurements}
\label{measurements}

Three kinds of \glspl{trait} describe every character in the game: \glspl{attribute} show general aptitude, \glspl{skill} show training, and Knacks show exceptional ability within a narrow focus.
All other \glspl{trait} -- \glspl{hp}, combat abilities, movement rates -- come from combinations of these three.

\section{Attributes}
\label{randomAttributes}

\begin{multicols}{2}

\sidebox{
  \small
  \begin{boxtable}

    \hline
    \textbf{Trait} & \textbf{Description} \\\hline

    -3 & Abysmal -- a total liability. \\

    -2 & Useless and pathetic. \\

    -1 & Poor, clumsy, and a constant irritation. \\

    +0 & Mediocre and unremarkable. \\

    +1 & Notable and worthy. \\

    +2 & Outstanding. \\

    +3 & Peak performance\ldots and often strange. \\

  \end{boxtable}
}

\noindent
Each character's Attributes range from -3 to +3, although most people have a Bonus of `0', in most Attributes.
Since these Attribute Bonuses adjust nearly every roll a character makes, they will determine the success or failure of a multitude of plans.

These averages vary by race.
Elves have a +1 penalty to Wits, so their `normal', is the equivalent of being notably sharp among other peoples.
Meanwhile, humans (being the tallest of races) have a standard +1 Bonus to Strength, so what they call `normal', others call `giant'.

\Gls{fenestra}'s \glspl{monster} often have Attributes higher than this -- there are no hard limits!

\subsection{Body Attributes}

\settoggle{examplecharacter}{false}

\subsubsection[Strength]{Strength \hint{muscle, brawn, toughness, height}}

Strength represents a character's muscles -- their ability to endure, to take damage, lift heavy objects, march for long distances and to wield heavy weapons without penalty.

\subsubsection[Dexterity]{Dexterity \hint{grace, co\"ordination, balance}}

Dexterity represents someone's hand-eye coordination and natural grace.
It's used to attack, parry, block and also to aim projectile weapons.
It is slightly less visible than the other Body Attributes, but others can still see the control over movement, especially when movement becomes difficult, as when hopping across challenging and changeable terrain.

\subsubsection[Speed]{Speed \hint{velocity, tendons,~vim}}

Speed represents a character's movement, how fast they attack, how often they can attack and how quickly they can run.
Since it allows characters to flee dangerous situations, the slowest member of a group will hold the rest back.

A low Speed Bonus in a weak person might simply represent small muscles, while a low Speed Bonus in someone with an excellent Strength Bonus might mean the character is particularly fat.
Players can roll Speed in situations where a character's muscle-to-weight ratio are important, such as when climbing up a cliff or holding onto a ledge for a prolonged period of time.

\subsection{Mind Attributes}

\subsubsection[Intelligence]{Intelligence \hint{memory, logic, tenacity, cunning}}

Intelligent characters understand ideas, remember well, and always come prepared.
They find their own way home and pick up new languages fluidly.
Intelligence also covers artistic endeavours and a multitude of craftsmanship, whether composing songs or forging armour, picturing the finished product ahead of time will take brains.

\subsubsection[Wits]{Wits \hint{alacrity, levity, attention, acumen}}

Where intelligence represents how well a character thinks, Wits just tells you how fast they think.
The character's ability to observe, to tell enemy from friend, to spot people hiding in the bushes, to notice an off taste in that poisoned casserole or to just spot the perfect joke for the occasion are all covered under Wits.
Wits is also the primary Attribute for resisting magical enchantments and spotting illusions.

Wits is the only Mind Attribute available to animals.

\subsubsection[Charisma]{Charisma \hint{magnetism, gravitas, glamour, friendliness, symmetry}}

Finally, a character's ability to speak with people, make friends, speak convincingly, lead a group or barter for cheaper goods are all covered under Charisma.
Charisma also covers characters' luck, and therefore some measure of their ability to avoid Damaged, because the gods seem to love a chancer.

\end{multicols}

\section{Skills}

\label{skills}

\begin{multicols}{2}

\noindent
A character's \glspl{skill} tell you what they do with most of their time.
A high Survival rating means they talk about wild plants, and probably hunt or fish, and a high Academics score means they read a lot, and communicate with other Academics.

\noindent
\begin{boxtable}[cL]

  \hline
  \textbf{Dots} & \textbf{Meaning} \\\hline

  1 & A novice, who makes occasional use of the Skill \\

  2 & A trained professional, who has spent years in these activities. \\

  3 & A master of the art, who spends weeks focussing on the minutiae of every aspect of the Skill. \\

\end{boxtable}

Each \gls{skill} pairs up with different \glspl{attribute} to show a multitude of different tasks.
A craftsman with great Dexterity may create beautiful and intricate items, but won't always craft with Speed, and if their Intelligence is poor, they may not be able to create new moulds well.

Have a look at this talented member of the \gls{guard}:

\toggletrue{allyCharacter}
\toggletrue{genExamples}
\Person{\npc{\F\Hu}{Grogfen}}%
  {{1}{0}{1}}% BODY
  {{-1}{-1}{1}}% MIND
  {
    \setcounter{Brawl}{1}
    \setcounter{Empathy}{2}
    \setcounter{Deceit}{3}
    \setcounter{Larceny}{1}
    \setcounter{Vigilance}{0}
    \shortsword
  }% SKILLS
  {}% KNACKS
  {50' rope, \rations}% EQUIPMENT
  {}% ABILITIES

Each Skill determines a wealth of different abilities.

\begin{description}
  \item[\roll{Strength}{Deceit}]
    allows her to intimidate with +4 to the roll.
  \item[\roll{Dexterity}{Empathy}]
    gives her +2 to dancing.
  \item[\roll{Dexterity}{Larceny}]
    gives her +1 to picking pockets.
  \item[\roll{Speed}{Larceny}]
    shows +2 to snatch-and-run attempts.
  \item[\roll{Intelligence}{Deceit}]
    gives +2 to planning a plausible lie.
  \item[\roll{Intelligence}{Empathy}]
    means a 0 penalty to negotiations.
  \item[\roll{Intelligence}{Larceny}]
    has her at +0 to plan a heist.
  \item[\roll{Wits}{Deceit}]
    gives her +2 to pull out a plausible lie on the spot.
  \item[\roll{Wits}{Empathy}]
    gives her +1 to see when someone lies to her.
  \item[\roll{Charisma}{Deceit}]
    gives her +4 to confidently sell someone on an idea.
  \item[\roll{Charisma}{Empathy}]
    gives her +3 when ingratiating herself with a new crowd.
\end{description}

Many pairings of \pgls{attribute} plus \gls{skill} will not come up often, but you should think of each likely pairing as an individual talent.
For example, a character with a Bonus to Academics and Vigilance has individual task Bonuses for \textit{forgery}, \textit{recall}, \textit{resisting enchantments}, \textit{storytelling}, \textit{keeping watch}, \textit{investigation}, and \textit{spotting illusions}.
It's only two \glspl{skill} on the sheet, but that's seven different ratings the character has.

See the table \vpageref{skillChart} for examples of how to view \glspl{skill} in multiple ways.

\bigSkillsTable

\subsubsection{Other \Glsfmtplural{skill}}
can be added if the default \glspl{skill} don't cover something.
Just run them by the \gls{gm} and discuss what kinds of tasks they cover.
When thinking up a new \gls{skill}, try to think about how it would work with each \gls{attribute}.

\subsubsection{Professionals}
usually take the \textit{Specialist Knack}, which gives them a +2 Bonus within that narrow field.%
\footnote{See \autopageref{specialist} for the Specialist Knack.}

As a result, \glspl{pc} will rarely match the abilities of \pgls{npc} who performs the same task daily.
A shepherd, for example, will gain a +2 Bonus for herding sheep, so even when their \roll{Intelligence}{Cultivation} only amount to a +2 Bonus, they will have a +4 Bonus in total to herding sheep.

\subsection{Academics}

The Academics Skill covers a love of learning facts, some of which may one day prove useful.
Academics study history, architecture, local politics, literature, and how to study more.
This `study of study', can involve reading, mnemonics, and teaching.

\sidebox{
  \begin{boxtable}[cX]

    \hline
    \textbf{\glsentrytext{tn}} & \textbf{Question} \\\hline

    7 & Simple \\

    10 & Standard \\

    13 & Obscure \\

    15 & Secret \\

    17 & Dangerous \\

  \end{boxtable}
}

\begin{description}
  \item[\roll{Strength}{Academics}]
    covers oration, as speaking to a large crowd requires strong lungs.

    A full auditorium can grant a +2 Bonus to the roll due to good acoustics, or a big hat could grant a +1 Bonus for drawing everyone's attention.
  \item[\roll{Dexterity}{Academics}]
    covers forgery, as the work needs a steady hand, along with an understanding of the meaning of every material and symbol upon coins, signet rings, and letters.

    To perform well, forgers need a large variety of materials, in order to select something which precisely mimics their target.
  \item[\roll{Intelligence}{Academics}]
    lets characters recall facts about an area's history.

    Libraries grant a bonus to the roll.
  \item[\roll{Wits}{Academics}]
    lets people resist spells of the Mind Sphere.
    Some think this protection comes from so years of thinking in terms of abstraction, while others say that reading too much twists the mind, making it less intelligible to anyone who hasn't gained the same books, and therefore hasn't gained the same world-view.
  \item[\roll{Charisma}{Academics}]
    covers storytelling.

    Travelling a little can grant a bonus, as people always like hearing new stories.
    Travelling a lot gives a penalty, as anyone too far removed from the storyteller struggles to understand the people, places, and even the values the story rests on.
\end{description}

\begin{exampletext}
  The \gls{jotter}'s purple veins throbbed, from his ears to his eyeballs.

  ``Third time.
  First, yous wouldn't help \pgls{village}, second yous were too good to clean up \pgls{bothy}, and now you've come back from the \gls{edge} without a single \gls{basilisk} egg.

  Everyone's demoted.
  Yous are now considered `Fodder', and you can leave the swords for the new recruits.
  Yous can go hunt bandits with sticks for all I care, and leave the armour too!''

  Hunting bandits without weapons effectively meant a death sentence.
  Luckily, Minkrash had been following the recent rulings in the \gls{court}.

  Minkrash sighed -- ``Could do\ldots or we could ask the \gls{warden} if he wants to eat his words''.

  ``The fuck are you talking about?!
  You think the \gls{warden} wants to listen to you?''

  ``Not exactly, but she'll listen to herself.
  Last month \pgls{jotter} sent out a bunch of new recruits, armed only with daggers.
  When they didn't return, the family summoned that \gls{jotter} to the \gls{court}, and \gls{warden} Carnyx agreed that he'd been negligent.
  He's still in the oubliette.
  It's a tiny room, barely enough room to turn around, and he never lies down.
  They say you can learn to sleep standing up after a couple of weeks, but then the spine never really recovers.''

  The \gls{jotter}'s pulsing veins changed direction, and he resolved to kill each of these miscreants one way or another.
  But at least they'd be well armed\ldots
\end{exampletext}

\subsection{Athletics}

This covers all manner of fancy movements, from somersaults and rolling to climbing and circus skills.


\begin{description}
  \item[\roll{Strength}{Athletics}]
    covers lifting and throwing heavy objects.

    Rope will help with some items, as does a good place to grab onto something.
  \item[\roll{Dexterity}{Athletics}]
    covers climbing.
    Rope is sometimes a requirement, but that does not make it a bonus.

    Climbers can gain a bonus from a solid plan -- many mountain faces are simply impossible without knowing the best route up.
  \item[\roll{Speed}{Athletics}]
    covers sprinting.
    Almost all of the \gls{guard} can sprint well, as the slowest members of any group tend to become something's lunch.

    Open roads may help someone fleeing pursuit, but only when the pursuer does not run on the same road (otherwise both have a bonus, and they cancel each other out).
  \item[\roll{Intelligence}{Athletics}]
    lets someone identify the best climbing route before they begin, which lets gain a bonus to actually climbing later.
  \item[\roll{Charisma}{Athletics}]
    covers stage tumbling and circus acrobatics.
\end{description}

\begin{exampletext}
  Climbing the mansion's ivy-straddled walls wouldn't challenge any child of the streets, but getting the crew up required real understanding.
  Coalgrit looked at the ivy climbing two stories up towards the rotten shutters.
  It wouldn't hold his first companion (a bulky gnome), never mind the mad thug who'd joined the \gls{guard} voluntarily (apparently just to crack bandit skulls).
  He ascended carefully, and found an impenetrable window.

  It was bronze mesh, filled with glass, hard set into the wall with deep nails.
  Nothing could get in, except Sunlight.
  He took out his chisel, rested his elbows on the window's sill, rummaged his foot along the ivy to find the best place to steady his body, then pulled a hammer from his tool-belt.

  A `clank' went out as he punctured a glass piece at the base of the bronze-meshed window.
  He looped the hammer's claw round to puncture another bit of glass from the inside, and let it land below.

  With his work done, Coalgrit dropped the tools, letting them land in the soft earth below with a timid `thunk', and spread his weight across different ivy strands once more.
  He climbed down the building's floors with a plan in mind.

  Collecting his tools and a single shared of glass from the ground, Coalgrit began walking home, and thought over the equipment for that night -- one knotted rope to tie through the openings in the bronze grate, and a saw to cut through the rotten window-latch.
  After that, everyone would follow up the rope without issue.
\end{exampletext}

\subsection{Caving}
\index{Caving!Skill}

Navigation, foraging, building basic structures, and plants, all need to be relearned ten steps into a cave.
Just as most humans understand some basic Cultivation, almost all dwarves and gnomes know a little of the Caving Skill.

\begin{description}
  \item[\roll{Strength}{Caving}]
    covers throwing stone out of the way to form a path.
    With a good pickaxe, cavers can mine out new paths.

    Of course, mining out paths becomes easier with the right kind of fire-starting equipment.
    Sufficiently hot rock-faces may become weak, or even shatter.
  \item[\roll{Dexterity}{Caving}]
    covers climbing across caverns, or navigating dangerous routes.
  \item[\roll{Intelligence}{Caving}]
    covers remembering all the twists and turns on a long journey, and remembering to bring the right supplies.
    It also helps one avoid foolish mistakes, such as lighting a fire underground.

    Of course, one can always coordinate easier with a map, or at least something to takes notes on.
  \item[\roll{Wits}{Caving}]
    covers spotting dangerous areas, or keeping track of one's altitude.
\end{description}

\begin{exampletext}

  ``And where do you think you're going, wearing that elf-getup?''

  ``It's not elvish, it's my Summer wardrobe'', said the \gls{seeker}.
  ``I don't like armour -- it makes me all sweaty.''

  ``And will this `wardrobe' protect you from the cold down there?
  And what the hell are \emph{you} wearing?''

  ``Full plate armour'', he smiled proudly.
  ``I thought dwarves invented plate armour''.

  ``Correct, but we invented it to protect perimeters, not to invade warrens.
  Can you fit through a goblin-sized tunnel in that?
  Will it make a noise when we try to creep up on any sentries?
  Will you be able to see the ceiling at \emph{all times}?
  And what the hell are you carrying?''

  ``You don't use torches underground?
  I don't know about dwarves, but humans need light to see.''

  ``And do you need air?
  Because if we light two of those in a narrow passage and don't get out within five minutes, we'll be rolling around, giggling like a bunch of juvenile princesses.''

  They all laughed.
  Apparently hypoxia was some kind of joke to these idiots.
  The dwarf just sighed, and regretted every decision that lead him into the \gls{guard}.
  \index{Hypoxia}
  
\end{exampletext}

\subsection{Craft}

The Craft Skill allows \glspl{pc} to make, fix, and occasionally break things.
Exactly what the character can craft depends on their other Skills.

\craftingReqList

\begin{description}
  \item[\roll{Strength}{Crafts}]
    lets characters kick down a door.
    Muscle can help, but it won't help that much when kicking the wrong spot on the door.
    \index{Doors!Busting}

    And of course, a heavy-duty battering ram can grant a +2 bonus.
  \item[\roll{Dexterity}{Crafts}]
    covers making any item the character can use.
    The character must have at least one level in the Skill they want to use for crafting.
    Crafting arrows requires Projectiles, and making backpacks requires Cultivation.

    All craftsmen require tools to work.
    Better tools, good schematics, and quality moulds can grant bonuses, while shoddy equipment can inflict penalties.
  \item[\roll{Intelligence}{Crafts}]
    lets characters make good moulds and plans, for themselves or others.
\end{description}

\begin{exampletext}
  The little contingent of \glspl{guard} woke to find the \gls{village} mourning their fletcher.
  \Pgls{woodspy} had unlocked his door with a clever tentacle, and entered his home.
  Nobody could loose an arrow at it, firstly because it had entered a house, and secondly because they had run out of arrows while defending their \gls{village}.

  ``Shame'', one said.
  ``We'll miss his arrows, especially this far out''.

  The \glspl{guard} knew their duty -- they would have to enter the house, and kill it, so they entered slowly, with swords out, and carefully stabbed at every bed, cushion, shelf, or any other medium-sized object they could stab.

  ``Real shame'', one of them said with a long sigh.

  ``Nobody knows that we cleared the road out of here, so the traders won't come up here until we return down that road, and tell them they can come up here safely.
  If we don't return, they'll assume we died, because of the same thing that killed everyone else coming up here.''

  By the time they'd finished stabbing at everything in the house, they realized that the \gls{woodspy} had long-since left, unnoticed, after eating the fletcher and his child.
  And with the morning entirely spent, they would have to return back along the lonely road quickly, if it wasn't too late already.

  Of course, that would leave the \gls{village} without any arrows, at night, which would mean that if anything approached, they would have to repel it with their weapons: two spears, a plough, and several flails.

  They \glspl{guard} stared at the fletcher's equipment, and thought of trying their hand at make-shift arrows, and whether the farmers would be better off with loosely-fitted arrow-heads and false-hope than they would be without.

  ``Damned shame'', they started to realize that the \gls{village} could fall before they returned with reinforcements, all for the want of their fletcher.
\end{exampletext}

\subsection{Cultivation}

This \gls{skill} means civilization.
It combines every practice where one handles the wilderness to make it into a tool.

\begin{description}
  \item[\roll{Strength}{Cultivation}]
    to erect wooden walls.
  \item[\roll{Dexterity}{Cultivation}]
    to weave clothes or rope.
  \item[\roll{Speed}{Cultivation}]
    to skin a corpse before rot sets in.
  \item[\roll{Intelligence}{Cultivation}]
    to predict and remember \gls{fenestra}'s strange \glspl{cycle}, from \gls{cOne} to \gls{cSix}.
\end{description}

\begin{exampletext}
  ``It was three years ago, just by the next \gls{village}, ten brigands came out of the snow, all carrying hunting bows.
  We were wounded from \pgls{basilisk} which we had fought just earlier that day, but we did not let our wounds show, and demanded that the brigands\ldots''

  ``During a snowstorm?'', one of the farmers asked.

  ``No, it wasn't a snowstorm, but the cold was bad, and so I said to them\ldots''

  ``It was a snowstorm'', the farmer insisted.

  ``Just after the eclipse, you said it was.
  Three days of snowstorm.
  Nobody was going anywhere, certainly not these `brigands', and the \gls{basilisk} you fought should have been hibernating.''

  ``Well'', the young \gls{guard} said with less gusto.
  ``Maybe it had trouble sleeping\ldots''.
\end{exampletext}

\needspace{8\baselineskip}

\subsection{Deceit}

Someone proficient at deception can make others see white as black by sheer confidence.

\begin{description}
  \item[\roll{Strength}{Deceit}]
    covers intimidation.

    Weapons can help with this endeavour just as well as they do with M\^{e}l\'ee, so an axe which grants +3 to hit in combat will also give +3 to intimidation attempts.
  \item[\roll{Dexterity}{Deceit}]
    covers fake spell-casting.
    People fear fake curses just as much as real ones!
  \item[\roll{Intelligence}{Deceit}]
    covers elaborate lies and ruses.
  \item[\roll{Wits}{Deceit}]
    covers on-the-spot lies, for when you just need to explain your presence in a rush.
  \item[\roll{Charisma}{Deceit}]
    means selling something -- an idea, a plan, an item, or a road.
    With enough confidence and gravitas, characters can sell anything.

    Of course having quality ideas and goods helps a lot, so they can add a bonus (or penalty) to the roll.
\end{description}


\begin{exampletext}
  Vanw\"e returned to the \gls{broch} with that creepy elven scowl that looks half-way between solving a puzzle, and getting ready to stab someone.
  \Gls{jotter} Coriolis began hopefully, before reorganizing her attitude.

  ``So what did you find in the city?
  How much gold can I put in the\ldots
  where are the others?''

  The scowl didn't move.
  ``At the tavern, in \emph{town}, all drinking.''

  ``Vanw\"e'', she said, trying to sound like an authority.
  ``You know that you can't go into towns.
  It's forbidden for the \gls{guard} to\ldots''

  ``Three days we searched.
  The place looks a century old.
  Not a bit of wood remains.
  It's pretty obvious that the place must have been raided ten times over by now, but we trusted you when you said it had valuables.

  We searched through the ruins for two days, and found nothing but acidic creatures, slowly inching towards us.
  And we could out-run them, but had nowhere to sleep.
  Did you send us out to die?''

  The \gls{jotter}'s attitude had finally turned full circle.
  She'd have to mark the abandoned city off her to-do list and her map if it really was that old and ravaged.

  ``I had no idea.
  I am sorry.
  Listen, I'll write you all a note, stating some business in town.
  Why don't you join the others, and you can all have a few nights in town to relax after the dangerous journey?''

  Vanw\"e held her vicious gaze silently, then took the note of permission, and went to town.
  And on the way, she nipped into the forest, to pick up the little bag of golden rings, rubies, and ancient alchemical books they'd found in the abandoned city.

\end{exampletext}

\subsection{Empathy}

The art of understanding people is practised by kind souls as well as malicious.

\begin{description}
  \item[\roll{Dexterity}{Empathy}]
    covers dancing, and the minstrels add a bonus or penalty.
  \item[\roll{Intelligence}{Empathy}]
    covers negotiating skills and puzzling out someone's likely motivations.

    Bonuses and penalties depend on leverage and information.
  \item[\roll{Wits}{Empathy}]
    helps spotting lies and judging someone's abilities.
  \item[\roll{Charisma}{Empathy}]
    covers making new allies.
\end{description}

\begin{exampletext}
  Nine or ten young men.
  Eleven or twelve eligible young ladies.
  One elf, polymorphed into a young noble.

  The \gls{seeker} requested a dance, and complemented the young lady's style.
  It wasn't her.
  She was strong, but hid her strength well.

  The second was clumsy -- she didn't know the dance.
  She repeatedly looked up at him with a little flecks of embarrassment.

  The third seemed disinterested, but still tried to lead the dance.
  Her fragile arms hadn't the muscle to properly telegraph her movements, and the dance soon became awkward.

  Bingo.
\end{exampletext}

\subsection{Larceny}

Theft, looting and arson all benefit from experience.
Of course the \gls{guard} have no use for any of those horrible things, but since most \glspl{guard} begin as criminals, most groups have a pick-pocket, cut-throat, or brigand, ready to employ their abilities if they find half a chance.

\begin{description}
  \item[\roll{Dexterity}{Larceny}]
    covers picking pockets.
  \item[\roll{Speed}{Larceny}]
    covers snatch-and-run jobs, which is what usually happens to a failed attempt to pick someone's pockets.
  \item[\roll{Intelligence}{Larceny}]
    covers picking locks, as someone has to understand the mechanism, without seeing it.
    \index{Doors!Unlocking}

    Picking locks always requires equipment, but someone can always attempt to make materials out of items they have to-hand, if they don't mind a sharp penalty to the roll.

    A tie generally indicates that the lock opens, but also breaks, leaving the entry obvious.

  \item[\roll{Charisma}{Larceny}]
    helps to create a distraction.
\end{description}

\begin{exampletext}
  ``This is fucking ridiculous.
  The painting alone should be worth fifty \glsentrylongpl{gp}.
  And we can't buy a sword, crossbow, or even a fucking bed with a back-pack full of treasures.''

  Gritsnatch slumped by the fire, and dumped the oversized canvas backpack full of more valuables than he could buy in a hundred lifetimes with a \gls{guard}'s wage.

  ``We need a fence'', Vanw\"e told him.

  ``Nah, a hedge won't hide it.
  Besides, the water will get in eventually.
  We need to sell it, but there's nobody -- I've asked all over town.''

  The rest moaned, hands on faces, but Vanw\"e continued her glass-eyed stare.

  ``A fence is a type of human who buys valuable things without regard to the laws of the other humans.''

  ``Oh, you mean a `fence'?
  Like I said, the town has none.
  I've asked.''

  ``You asked to sell.
  We should try to buy instead.
  We should ask to purchase art to decorate a small mansion.''

  ``Why in the blue fuck would I want to buy art when I have art, and have no fucking money and no kind of mansion?''

  Vanw\"e swivelled her stare to the other humans, which indicated that she hard started to count how long until they figured something out.
  This ranked among her top-three most annoying habits, but everyone started pondering anyway.
\end{exampletext}

\subsection{Medicine}

Medicine is a primitive but effective art, regrettably full of nonsense and superstition, but mandatory when it comes to keeping someone with a serious wound alive.

\begin{description}
  \item[\roll{Dexterity}{Medicine}]
    lets someone fix a twisted bone or nose.
    If someone has only 1~\gls{hp} of Damage, a medic can heal it, leaving them with only 1~\glsfmttext{ep} instead.
    A failed roll inflicts an additional \gls{hp} of Damage instead.
  \item[\roll{Intelligence}{Medicine}]
    covers making poisons, and figuring out which poison has affected someone.
    Find the systems for poisons \vpageref{poison}.
  \item[\roll{Wits}{Medicine}]
    covers stopping someone bleeding out.
    See \autopageref{death} for more on this.
\end{description}

\begin{exampletext}
  ``Blood-letting doesn't work on gnomes'', he protested.
  ``We need all our blood to work''.

  ``Not the bad blood'', she smiled.
  ``If you get an injury and fill up with `angry~blood' when the second moon is above those three stars you'll catch a fever, now sit still\ldots''
\end{exampletext}

\subsection{Performance}

This \gls{skill} covers acting, instruments, crowd control, and storytelling.
Those with Performance will pick up at least one instrument per level.

\begin{description}
  \item[\roll{Strength}{Performance}]
    covers long sessions performing.
    Minstrels with limited stamina never last long.

    The bonus depends on the quality of the instruments, with a standard range of -1 to +1.
  \item[\roll{Dexterity}{Performance}]
    covers lively performances, playing challenging pieces with a string-instrument, mime-acting and being understood by a whole crowd.
  \item[\roll{Intelligence}{Performance}]
    covers crafting new creations, such as plays or songs.

    Creators can gain a bonus if they have an unlimited supply of paper, ink and candles.
    When used responsibly, this ability can also provide a Bonus to spreading information.
  \item[\roll{Wits}{Performance}]
    to come up with an insulting rhyme.
\end{description}

\begin{exampletext}
  The elf looked sad.
  He had played for three hours, and he thought he had sung well.
  The notes were crystal-clear, his fingers delicately pulled twelve notes every breath he took.
  His songs had made the nobles who hosted the troupe cry, but here in the market the crowd remained three beggars and a dog.

  Fensoak smiled at her companion's incompetence.
  He still didn't really understand humans.

  Fensoak pulled all the thick smells of the marketplace into her lungs and began.

  ``Hoo-rah, up she rises!'',

  (she beckoned the elf to strum along)

  ``Hoo-rah, \emph{up} she rises!'',

  (she mimed to the elf to thrash the strings harder)

  ``Hoo-RAH, up she \emph{rises}!'',

  And half the market -- already her crowd -- sang the next line in response.

\end{exampletext}

\subsection{Seafaring}
\index{Sailing}

Sailors don't just sail, they typically know how to fish, coordinate reefs, work with others on larger boats, mend masts, sails and nets, and generally do a lot of sewing.

\begin{description}
  \item[\roll{Strength}{Seafaring}]
    holding the boom against a strong wind.
  \item[\roll{Dexterity}{Seafaring}]
    mending a sail in a storm.
  \item[\roll{Intelligence}{Seafaring}]
    navigating the open oceans.
  \item[\roll{Wits}{Seafaring}]
    noting a sudden storm brewing.
\end{description}

\begin{exampletext}
  ``Everyone wants to sail, nobody wants to build a boat''.
  Keelvore muttered the old phrase, as if cursing an enemy.
  Nobody could reliably cart stone from another land with two-man boats, and nobody could build a larger boat -- \gls{storm} destroys everything at every coast at the end of each \gls{cycle}.

  ``It's impossible!'', he shouted at his own schematics.

  ``What's `impossible'?'', a little voice asked, somewhere under table-height.
  She hopped up to stand on a stool, and examining the maps, and oversized boat-designs.
  Keelvore couldn't tell if she wanted to see the problem, or simply did not understand this one word.
  Either way, she was hooked.

  ``My cousin lives in a grotto, here'', she fingered a map.
  ``If you could take the boat's mast down, it could live safely inside, even if it were twenty \glspl{step} long''.

  ``Could little people really build such a boat?'', Keelvore tried to sound non-insulting.

  ``Would big people really want to sail a boat made with little hands?
  I couldn't build it, but I could orchestrate the building.
  How many humans do you have for labour?
  What shall I call you?''

  ``Maybe twenty?''

  ``Then we can do it -- `Maybetwen' -- stop drinking, we must plan with sober heads.''

  Keelvore suddenly had doubts, but it seemed somehow too late to back out, even if nothing had begun.
\end{exampletext}

\subsection{Stealth}

Stealthy movements can begin with pranking siblings, or abusive parents, and then most lose the knack as they grow up.
But a little practice and aptitude can let someone wander like a ghost.
In most cases, opponents resist with \roll{Wits}{Vigilance} to spot the character or spot the ruse.

While sneaking, players can say anything about what their character does, but any remotely in-character speech directed at another \gls{pc} means their character has communicated, someone spots them, and they lose \pgls{ap}.

\begin{description}
  \item[\roll{Strength}{Stealth}]
    to pull open a bale of hay, and hide inside.
  \item[\roll{Dexterity}{Stealth}]
    to move silently across squelching mud.
  \item[\roll{Intelligence}{Stealth}]
    to find the perfect spot to listen, without being noticed.
  \item[\roll{Wits}{Stealth}]
    to jump into hiding as someone unexpected walks up the stairs.
\end{description}

\begin{exampletext}
  Snow changes travel, but with enough cosy clothes, some pies, and a little gumption, \glspl{jotter} can still order the \gls{guard} to go anywhere, as if it were the height of Summer.

  Grogfen cursed the \gls{jotter} for the tenth time that journey.

  ``I hope she lies down with \gls{sable} until her fingers turn black and rot away.''

  ``Not all bad though'', she mused.
  ``The frost brings some safety, since the \glspl{basilisk} and \glspl{crawler} hibernate.
  I hear the \glspl{basilisk} go underground.
  What do you think happens to the \glspl{crawler}, Sootfilch?''

  ``Is that one there?''

  ``Nah, Soot.
  They're hibernating.
  But where do you think they hibernate?''

  ``Right here'', she said again.
  {\small
    ``That's the mouth-bits, sticking out.
    That's them in those snow-mounds, and the black bits at the top are the mouth-bits, so we should\ldots''

    ``\Gls{sylf}-crap, stop making that noise when you move.''
  }{\footnotesize
    ``We're all making noise when we walk.
    That's one of the back legs sticking out --- is it moving?
    Do you think they laid webs under the snow?''

    Grogfen stopped moving.
    Don't be stupid, webs can't be in snow\ldots can they?
  }{\scriptsize
    ``Wait'',
    Grogfen held up her hand as three long, black, legs slowly came out of the first snow pile.
    ``We don't have to move back.
    We can move forwards.
    They haven't woken yet, so we could just\ldots maybe\ldots
  }

\end{exampletext}

\subsection{Survival}

This \gls{skill} covers everything which lives in the green sea past the \gls{edge} -- the untamed and uncultured wilderness.
A survivalist understands \gls{navigation}, \gls{foraging}, and hunting.

\begin{description}
  \item[\roll{Strength}{Survival}]
    to stay steady in a storm.
  \item[\roll{Dexterity}{Survival}]
    to move through thorny undergrowth unharmed.
  \item[\roll{Intelligence}{Survival}]
    covers navigating a shorter route between two known locations.
  \item[\roll{Wits}{Survival}]
    to note rare and valuable plants in the wilderness.
\end{description}

\begin{exampletext}
  Mildrain slumped with her companions, while they complained about her fire.
  She needed a rest before completing the shelter.

  ``The forest will see us, Mildrain.
  We can't afford a fight, put out the fire.''

  They said this while shivering, and huddling next to the little fire-pit.
  Dug into the ground, it didn't give off much light, but the rain-sodden branches hissed and cracked like a whip.
  Then more crackles came from the edge of the clearing.
  \Pgls{crawler} announced itself, unashamed, walking as casually as a sheep coming its morning feed.

  ``Stand up.
  Stand-the-fuck-up'', she told her companions, while grabbing two bushy branches from the shelter, destroying it.
  And warriors stood, using their swords like walking-sticks to hold themselves up, shaking from cold and hunger.

  Then she held the branches out, like the wings of some ridiculous bird, and screamed as loud as she could, inching slowly forwards towards the face at the edge of the fire.
  Her companions joined the choir, trying to shout in a way that didn't sound like a wounded animal.

  On more step, and eyes pulled back, then fled through the wet woods.

  ``The thing to remember about \glspl{crawler},''
  Mildrain informed her companions,
  ``is that they're stupid''.

\end{exampletext}

\subsection{Vigilance}

Everyone in the \gls{guard} practices paranoia daily.
When the new \glspl{fodder} enter, they see their superiors flinching at every noise and staying awake all night with their backs to the fire, staring into the darkness wide-eyed, for hours; and they think `maybe that could be me one day'.

\begin{description}
  \item[\roll{Strength}{Vigilance}]
    lets one keep watch over a camp, despite a long day's march and a quiet fire.

    Staying up all night can \glsdesc{vigil}, so one \gls{pc} may receive \glsfmtplural{ep}, or two \glspl{pc} could take one each.

  \item[\roll{Dexterity}{Vigilance}]
    to feel out the right route in a lightless labyrinth.
  \item[\roll{Intelligence}{Vigilance}]
    to collect clues in a crime-scene.
  \item[\roll{Wits}{Vigilance}]
    to notice imminent danger.
  \item[\roll{Charisma}{Vigilance}]
    investigating a rumour.
\end{description}

\ifodd\value{r4}
\begin{exampletext}
  Goutfrak had raided a local tomb, but found it already ransacked, and was down to her last \glsentrylong{gp}.
  She examined the change from the bar.
  The \glsentrylongpl{sp} came from the time of Rex Dalyus -- making them at least two centuries old -- but they looked nearly new.

  It had to be the tomb raiders who had cleared out the nearby grave before she arrived.

  She hopped back to the bar for another pint.

  ``Another \glsentrylong{gp}!'',
  the barman exclaimed.

  ``I can't keep giving you change for this massive coinage!''.

  ``Sorry'',
  she shrugged, as the barman handed over yet another shiny coin with the Rex's.

  They were in here somewhere\ldots

\end{exampletext}

\else

\begin{exampletext}
  ``The beer arrived, fizzing but not frothing, and no head.
  Everyone in the troupe downed their drink but Ratfix.
  He stared, perturbed, mulling the problem, without a drink.

  Soon after he mentioned to the rest,
  ``I think we may have been drugged, and not in the way we paid for.''
\end{exampletext}
\fi

\end{multicols}

