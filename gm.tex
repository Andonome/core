\chapter{\glsentrylongpl{gm}}

\section{Basic Prep \& Play}

\begin{multicols}{2}

\noindent
The basic tools of the \glsentrylong{gm} must begin with with the obvious -- $4D6$ per player with multiple $D6$ colours so players can differentiate their Damage dice from their Action dice.
Remember pencils and a rubber, as players never bring their own.
Lastly, print out a load of character sheets.
This can be a lethal game, so players may need more than one.
They'll also need character sheets for any \glspl{npc} they bring into the game.

\subsection{Coins}

To helps players understand the tactical elements of the game, consider setting a central initiative track on the table, with the numbers 1-18.
Have everyone place a token, model, coin, or whatever, on their own Initiative number as soon as combat starts so that they can see the Initiative count moving slowly towards them.

As a \gls{gm}, it's always good to have at least 3 different types of coins.
Let's say you're orchestrating a battle with a hobgoblin leader, some hobgoblin troops and a goblin spellcaster.
Assign each one a coin and make a little mnemonic -- the spellcaster has dark magic so it gets the little copper penny.
The hobgoblins get the silver coin to represent their use of weapons, and the largest coin goes to the hobgoblin leader.
Don't worry about the players' Initiative -- they'll keep track of their own characters as you shout out where on the Initiative tree you are.

Coins should also be used when assigning the Combat Skill.
The character sheets contain a large space in the middle where players can add bonuses to their Combat Factors rather than attempting to remember where everything was placed.

Coins can even be used to keep track of \gls{fp} and \glspl{fatigue} as they change so often.
It'll help cut down on wear to the character sheet.

\subsection{Tracking Information}

Print out the \gls{gm} sheet at the end of the book for a little help handling all the information you'll need to keep track of during a campaign.
In particular, this is a good place to keep track of stats for all those \glspl{npc} that you need to make up on the fly.
Remember that it doesn't matter what you put for \gls{npc} stats, so long as those stats are consistent.

Long-standing \glspl{npc} should also have their \glspl{fp} listed next to the character, as \glspl{npc} gain \glspl{fp} at the end of each scene.
This helps beloved \glspl{npc} stay alive, as well as adding a little extra gravitas to any antagonists who encounter the \glspl{pc} multiple times.

\subsection{How to Make Rulings}

\subsubsection{Literal Interpretations}

If it's ever unclear how to resolve a situation, the first attempt should always be a strict interpretation of the rules.
For example, if a player says `If I charge round a corner, rather than a straight line, can I still use the Fast Charge knack?', the answer is `yes', because the rules as they stand don't prohibit going round a corner.

No rules will work all of the time, but by following a literal interpretation of the rules whenever possible, players feel better able to predict and navigate the world, and \glspl{gm} do not have to waste so much energy on making on-the-fly rulings.

Broadly, the \gls{gm} should consider themself bound by the rules as much as the players.
A good rule of thumb is to make as few decisions as possible, and let yourself focus on description and planning.

\subsubsection{Let Players `Ruin' the Adventure}

Encounters don't have to play through like you think they do.
If the players flood a dungeon, cast a fireball at the king, or raise their stats so high they can enchant every wild creature they will ever encounter, take a break, re-examine the situation, and go from there.

Perhaps dungeon has a low-point which isn't flooded, or perhaps it's flooded forever, and nobody will see that treasure again.
Perhaps the party have to become outlaws, and every future adventure has to take this into account.
Perhaps all encounters with bears, boars, and griffins turn into opportunities for new, temporary pets.

\subsection{\Glsfmttext{downtime}}

The most interesting \gls{downtime} happens when you skip it, and find out the details later.
Don't ask the players what their characters want to do, just jump straight to the next scene, a month, or even years later.
Short \glspl{downtime} should pepper a campaign to allow breathers.

Since people in BIND heal at the real-world rate, players will ask about stopping to heal often, and the default answer should be `yes'.

Once they reach a town or village, they can rest up, and you can mark weeks or months off your calendar.
\Gls{downtime} also provides the best opportunity for purchasing items, like expensive armour, or specialized adventuring equipment.

\sidebox{
\begin{itemize}

	\item
	Heal
	\item
	Apply cost of living
	\item
	Buy Traits
	\item
	Award \glspl{storypoint}

\end{itemize}
}

Of course, if they have active enemies, you can throw them in here, or just run a single encounter in town, to keep them on their toes.

Other \glspl{downtime} will last months, so that \glspl{pc} can purchase new specialized skills, such as Crafts.
Alchemists also need \gls{downtime} to purchase new sphere levels.

For any \gls{downtime} spanning over a year, you may want to award \glspl{storypoint}.
Spending them can later reveal what the character has done.
Perhaps they travelled and learnt a new language or found fantastic riches.

\sidebox{

\begin{rollchart}

	Years & \glspl{storypoint} \\\hline

	1-2 & 1 \\
	3-6 & 2 \\
	7+ & 3 \\

\end{rollchart}

}

Each year of downtime should cost 10\% of the character's wealth, or 10\gls{gp} (whichever is higher), to represent the money they've spent during this time.
Characters without any other means of sustaining themselves should default to spending 10\glspl{gp} per year.

\subsection{Slow Campaigns}

For a slightly slower campaign, consider removing the \glspl{xp} rewards for killing monsters.
The standard setup has characters swelling with new abilities every couple of sessions.
Removing the monster slaying focus should initially demand players spend \glspl{storypoint} for \glspl{xp}, before they start finishing Side Quests.

\end{multicols}

\section{Random Encounters}
\index{Encounters}
\label{encounters}

\begin{multicols}{2}

\subsection{Creating Encounters}

Whether you're in the middle of an adventure or the \glspl{pc} are just randomly wandering the world without any respect for local laws or plot, a random encounter can always add a sense of danger to a non-urban area.

Each time the players pass through a region, roll $3D6$ on the encounter table and create an encounter from the result.
You can make a unique encounter table for each region in your campaign to individuate them.
As an example, have a look at Gloomhaven's forests:

\begin{encounters}{Gloomhaven}

	Marshes & Forest & Result \\\hline

	\li & Elven fortress. \\
	\li & $2D6-1$ elven hunters. \\
	\li & $1D6+5$ Hobgoblins. \\
	\li \lii $3D6-2$ Ghouls. \\
	\li \lii $3D6-2$ Goblins. \\
	\li \lii $1D3$ Griffins. \\
	\li \lii $2D6$ Bandits. \\
	& \lii Bear. \\
	& \lii $2D6$ Wolves. \\
	& \lii $2D6-1$ Human traders. \\

\end{encounters}

The forest can be a dangerous place, but not nearly as dangerous as the marshes.
The entire Gloomhaven area is infested with ghouls, but they get much more common once one passes beyond the forest's edge and into the marshes.

Some encounters presented are fairly benign.
Wolves may try to steal the party's food, but they're not dangerous, and human traders simply provide an opportunity to gain news, and travel with a little more safety.
Despite the different tables, the overlap provides some cohesion to the area.

If you reach a result which is not listed, there is no encounter.
If you roll an encounter on trips (three of the same number) roll again, and if you get another encounter, combine the two.
If you get a griffin and a bandit, perhaps the players stumble upon bandits in the woods, attempting to pilfer griffin eggs for a patron.
If you roll wolves and a griffin, perhaps the players hear victorious howls in the distance, as wolves have caught a griffin.

You may want to set up your random encounter before the start of the session, allowing you to review monsters' stats and perhaps tie the encounters together, or integrate them with active characters from players' \glspl{storypoint}, or recent events in the campaign.

If you have a campaign book such as \textit{Adventures in Fenestra}, you'll find stats for creatures, suggested encounters, and random encounter tables for the different areas.

\subsection{Running Encounters}

\subsubsection{Direction}

Anyone the party meet on a road either comes in front of them, or behind.
Flip a coin, to find out if the encounter is \textit{head}ing towards them, or at their \textit{tails}.

On more open terrain, you can roll $1D6$ to pick a direction.
`1' means 'ahead', `2', means 'ahead from the right', `3' means 'behind from the right', `4' means `from behind', and so on.

\subsubsection{Distance}

Roll the encountered creature's Wits + Vigilance at \gls{tn} 7, minus the party's Wits + Vigilance.
This shows who spots whom first.

\begin{rollchart}

	Roll & Distance \\\hline

	10 & Griffins sees the party first, 40 squares away. \\

	9 & Griffins sees the party first, 20 squares away. \\

	8 & Griffins sees the party first, 10 squares away. \\

	7 & Party sees the griffin first, 10 squares away. \\

	6 & Party sees the griffin first, 20 squares away. \\

\end{rollchart}

The basic encounter distance is 10 squares in a dense forest, 40 on an open road, and 80 squares on a flat plane.
Each roll on the margin doubles this distance, and this works both ways.
For example, were the party (Wits + Vigilance = 1) to encounter a griffin (Wits + Vigilance = 4), the \gls{tn} would be 8, and the griffin would roll at +4.

If one side wants to sneak up on another, another roll can be made with Dexterity + Stealth, vs the target's Wits + Vigilance.
Those sneaking away gain a bonus equal to the previous roll's margin.

\subsubsection{Reactions}

\sidebox{
\begin{itemize}

	\item
	Direction
	\item
	Distance
	\item
	Reaction

\end{itemize}
}

Next, consider the other side's reaction.
A group of twenty goblins will obviously attack any small party of adventurers, but never a group of three.
That said, three goblins could follow the party for a while, hoping to see them lying down and vulnerable.

Bandits may react similarly, but can also show more intelligence.
They could demand the party pay them some gold in return for being left alone.

When in doubt, roll on the morale chart before combat can begin (page \pageref{morale}).

\subsubsection{Peaceful Encounters}

Peaceful encounters mostly make scenery.
If the players successfully hide from something nasty, it tells them about what kinds of creatures inhabit this area.
If they find a trader on the road facing towards them, he may share some gossip, or just assume they might rob him and try to his wares.

Almost any friendly people travelling in the same direction  as they party will want to join them for safety in numbers.

\end{multicols}

\section{Side Quests}\label{sidequests}

\begin{multicols}{2}

\noindent
Another way to add impromptu elements into your game is Side Quests.
These are short encounters which slowly feed elements into the background of your game.
They're good for foreshadowing without too much planning, and good for adding things to the path of players who simply want to run around in a sandbox, without the constraint of a full-on plot-arc.

Let's look at an example from a village area:

\begin{exampletext}

	Villagers have been cutting down trees near a spot sacred to the elves.
	Negotiations have failed, and now the elves intend to drive the humans out like vermin by burning down the human houses.
	Villagers start responding by attacking any elves, or magic users on sight (they associate all magic with elves).

\end{exampletext}

\subsubsection{Encounters}

\begin{list}{\Square}{}

\item[\CheckedBox]{(Villages) Villagers are burning a witch at the stake and will grab any known magic user or elf in the party.}

\item{(Villages) The party notice a group of elves sneaking up to a village. If they don't stop them, the elves attempt to set fire to various houses.}

\item{(Villages) Watchmen arrive in the villages, with orders to kill all magic users and elves on sight. Repeat.}

\end{list}

When the players enter the villages, you spring part 1 on them, so they see villages burning a witch at the stake.
The second time the players get a Side Quest in the villages, they might get part 2, where they see elves sneaking up to set fire to human houses.
Some Side Quests finish on a loop, so the players can repeatedly encounter watchmen in the village who will not take kindly to known magic users.

Notice that none of the encounters require the party to do anything.
If they don't want to engage in the plot, they can sit back and watch unless someone is actively trying to engage with them.

One more example:

\begin{exampletext}

A priest is using his ability to divine the future to capture criminals \emph{before} they commit crimes.

\end{exampletext}

\begin{list}{\Square}{}

\item[\CheckedBox]{(Villages) A local priest offers to tell the party their fortunes.  Combine this with the next encounter, then move it to Town.}

\item{(Town) The characters pass by men in stocks who keep shouting that they are all innocent, and were suddenly taken away by various guards after the local priest fingered them for a crime.  Move this encounter back to the villages.}

\item{(Villages) A dozen guards are tracking the characters. Repeat.}

\end{list}

The characters are now wanted by the guards who wander the villages, hunting for would-be criminals.

Notice that the first part combines with the encounter below it, meaning `whatever encounter is next on the list'.
This new encounter must always be from some other Side Quest, so that Side Quests merge together.
Exactly how these merged scenes play out rests in the hands of the \gls{gm}, but it's generally enough to simply run both encounters in quick succession.

\subsubsection{Random Side Quests}

In addition to story-based Side Quests, it's good to give each area a bunch of entirely random encounters.

\begin{list}{\Square}{}

	\item{(Forest) The party find a gnome attempting to sell them gemstones for his trip. Some are real and others are fake.}

	\item{(Forest) A dragon flies overhead.}

	\item{(Forest) A dead mage lies on the road. His books are valuable but should by law be returned to the mage's guild.}

\end{list}

This collection of non-quests serves two functions.
The first is to provide some short encounter when the time calls for it, but without getting the party wrapped up in yet another adventure.
If you already have five Side Quests happening at the same time, that's probably as much as the party want to handle.

The second use is in wrapping up a campaign.
If you have only two more plot-threads you want to wrap up, the rest of the world doesn't need to feel empty -- encounters can continue, but they needn't start more plot-threads.

\subsubsection{Summary}

Think of your campaign in terms of areas; a mountainous area by the sea might have `\emph{Underground}', `\emph{Mountains}', and `\emph{Coast}', while a deep forest might have `\emph{Elfwoods}', `\emph{Villages}', and `\emph{Swampland}'.

Each encounter is tied to an area, so when the players enter that area, they get the next encounter available there.
When the players enter the `\emph{Villages}', they encounter the next available a Side Quest.

Since Side Quests can leave the `Forest' area when the next part is in `Town', players will find themselves starting on a new Side Quest in the Forest, then returning to an old one once they enter Town again.
This format will soon have them engaged with multiple plot-arcs at the same time.
The party can often engage with these quests by seeking out a particular area, or going to preset locations, but if they choose to ignore any plot hooks then that's fine -- the plot will march on and conclude one way or another without their input.

If you want to run Side Quests as a secondary part of your game, you can just run them any time the group doesn't get a random encounter.

If you want them to be the primary mover in your campaign, you can run a Side Quest every time the group enters a new area.
You can also make one plot line the \emph{primary} quest by making it longer than the others.

Putting the above Side Quests together, if a party were moving from the villages, to the forest, then back to the villages, and finally to town, they would 
encounter Villagers burning a witch at the stake.
In the forest, they would find a gnome attempting to sell them half-faked goods.
If the party took some gems from the gnome and wanted to sell them in town, they'd have to return through the villages.
The encounter with the priest would combine with seeing elves sneaking about at night, so perhaps the priest travels with them, and that night he and the party all see the elves attempting to burn down houses.
Finally, once everyone reaches town, the party would find those men in stocks, put there by the prophecies of the priest.

However you run Side Quests, players should each receive 4 \gls{xp} for completing a Side Quest for each part the party engaged with.
A 2 part Side Quest grants 8 \gls{xp}, while a 4 part Side Quest grants 16 \gls{xp} to each party member.

\subsubsection{Anatomy of Side Quests}

Side Quests often begin with an example to introduce the players to the scene.
This example won't work for every group in just any situation, but provides a starting point to picture how things might play out.

\begin{boxtext}

	As you sit down to write your first Side Quest, you are assaulted by a blank white page!

\end{boxtext}

After that, you'll find details such as the \glspl{npc}, with their stats and motives.

After the Side Quests have finished, you'll find details of any locations relevant to the Side Quests.

Side Quests should never require characters going to a specific location, since they are something which happen \emph{to} the party, but Side Quests can still reference an area, such as the local priest's church, or the sacred lake which the elves guard.

\subsubsection{Preparation}

Rolling up Encounters and Side Quests beforehand can really get a game rolling, and you'll have more opportunity to integrate those encounters together.
You'll find space on your \gls{gm} sheet (back of the book) to write down a couple of Encounters and Side Quests per area.

Once a Side Quest becomes available, tick the box next to it in the miniature table of contents (the first one is ticked by default).
Once you have completed a part, mark it with an `X' then tick the next part to show it's ready to play once the party have entered the area.

\subsubsection{Holes in the Map}

Some encounters include places (not as attached locations, but as the actual encounter).
For example, one Side Quest might include finding `the Elven Citadel', while traversing the swamps.
From that point on, the Elven Citadel is in the swamps.

This has unintended side effects for maps, as some areas cannot be placed on any map until they occur.
Of course, all the set locations attached to side quest parts can be placed onto the map, so no map needs to be barren.

If you have paper to spare, designating a party cartographer can bring a map to life by filling in the areas as you go.

\subsubsection{Writing a Side Quest}

Consider this standard fantasy plot-hook:

\begin{exampletext}

	Bandits stole an alchemist's magical item, and he wants the party to retrieve it.

\end{exampletext}

Now let's expand with some foreshadowing:

\begin{enumerate}
	\item
	Merchants enter town without their wares, having been robbed.
	\item
	An alchemist asks the characters to find the items bandits stole from him.
\end{enumerate}

Now let's add a decoy -- the bandits must know adventurers often come after them, so they can send someone to lead the party into a trap.
And finally, we might add another lead to make things interesting.
Perhaps the bandit leader began as an alchemist who left his circle because they began practising Necromancy, and they want him dead so he doesn't spill the beans.

\begin{enumerate}
	\item
	(Town)
	Posters go up for a local alchemist, wanted `DEAD on sight'.
	\item
	(Town) 
	Merchants inter town without their wares, having been robbed.
	\item
	(Town)
	An alchemist asks the characters to find the items bandits stole from him.
	\item
	(Villages)
	A man offers to help the party find the bandits, but in fact wants to lead them into a trap.
\end{enumerate}

The first two encounters can combine with whatever else the players want to do.
If they want to drink, they drink and notice the posters.
If they want to buy weapons, they find prices have risen since shipments of iron and various other components have been raided by bandits, leaving the merchants with nothing but their skins.

If the characters skip town to find the bandits, we don't need the mage with the mission - they can simply encounter the guide in the village.

Once we've finished with those encounters, you might leave the rest to the areas: the bandits' hide-out, and then a local tower where alchemists secretly learn about Necromancy.
Or perhaps another part might slot into the town, or forest, involving an alchemist, and lengthen the Side Quest into a meatier tale.

Breaking things up like this allows different smaller plots to meld into one.
Perhaps characters go looking for the bandit alchemist immediately, and find some other encounter in the forests.
And maybe when the bandit spy joins the party, he'll end up in an unrelated encounter where the party find themselves trapped in a giant web, and he takes the opportunity to finish them off.

\end{multicols}

\section{The Undead}

\begin{multicols}{2}

\noindent
Undead creatures have certain properties in common.
Firstly they imperceptibly feed from the souls of the living.
This is not performed with the mouth but by merely being close to dying things and absorbing them before they can wander to the next realm.

Undead eyes generally do not work, instead they `see' the souls of people shining outward.
Inanimate objects such as books, or even fellow undead, are not so clearly seen; the undead can avoid bumping into these objects but have great trouble reading anything or working fine machinery.
However, they can operate in complete darkness and even fight without penalty, using the light of living people's souls to see them.
They can also see living beings from a great distance due to the soul-light they emit.

Undead also feel no pain and suffer little from scrapes and bruises.
As a result, they automatically have a \gls{dr} of 2 which stacks with armour in the usual way.%
\footnote{See page \pageref{stackingarmour}.}
This counts as Complete armour, but not Perfect -- shots through their eyes or attacks which sever muscles still debilitate them.

The undead do not tire -- they take no \glspl{fatigue}.
They can walk or dig or fight endlessly, without complaint.
They enjoy feeding on souls, but it is not required for them to continue moving.
Each has an Aggression score of +2.

When the undead are newly created, they are clumsy, as they are not used to their own bodies, and suffer a -2 penalty to Dexterity.
Shortly afterwards, rigour mortis sets in, and then decay.
Any undead more than a few hours old gain a -2 penalty to their Speed Bonus, but lose the Dexterity penalty.

Ageing corpses -- even those that age fairly well -- lose their ability to speak entirely.
Any ghast who wishes to speak will have to resort to either magic, writing, or some other system, because a dead tongue and dead lungs can never articulate things properly.
This can really get in the way of spell casting when it comes to the precise tones required by alchemical spells, but has little effect when it comes to other forms of magic, where intention outweighs precision.

\subsection{Mana}

The undead do not regain mana over time.
Rather, intelligent undead who use magic must kill to regain mana.
Every dead creature within their vicinity regains them 1 \gls{mp}, plus the creature's Intelligence Bonus (if positive).

The `range' of this ability is equal to five squares, plus five squares for each Wits Bonus of the undead thing consuming the soul (again, if positive).
Ties go to whichever of the dead has the highest Intelligence Bonus, then Wits Bonus.

\end{multicols}

\section{Combat}

\begin{multicols}{2}

\subsection{Fast Initiative \& Good Pacing}

You can give a good pace to combat by hollaring the Initiative count.

\begin{quote}

``Twelve! The gnolls ready their weapons''

``Eleven, ten! They move forward, bearing their yellowed teeth.''

``Nine! Snarls abound as they speed up to a rush.''

\end{quote}

Nothing has actually happened by this point, but it sets the scene nicely.

\begin{quote}

``Nine'', one of the players shout.  ``I'm going at nine.  I move to protect Max.''

``Two gnolls go for you, another two go for Amelia.  Roll to defend at TN 11.''

\end{quote}

The initiative continues down quickly at all times, and the count always provides a sense of urgency.
If players don't notice it's their turn when you're shouting, that's 1 Initiative point lost.
Do it once, and they'll never make the same mistake again.

\subsection{Speaking}

You may have noticed that speaking comes with an Initiative cost of 2.
It's important that this doesn't become a `gotcha' for any kind of speech.
Players shouting `charge', do not merit a penalty.

The cost for speaking exists to add a tactical decision.
When in battle, players may want to turn to a spell-caster and ask them to cast a \textit{Curse} upon an enemy, or request that someone guard them.
This speech helps the party tactically, so it has a cost.
When a player has a great idea about the whole group moving backwards to avoid enemy attack, they should think of the simple proposal as a tactical manoeuvre, and consider the cost of proposing it.

\subsection{\glsentrytext{npc} Fights}
\label{npcfights}

Add a few too many \glspl{npc} to a fight and you can end up either being a stumped \gls{gm} or having players wait for you to roll an awful lot of dice on your own.
That's no fun for anyone.

\begin{exampletext}

	``The goblin platoon start throwing more spears at you, but then from the side, the garrison of guards burst into the cavern's entrance to join you.''

\end{exampletext}

If you need a quick approximation of a massive battle between \glspl{npc}, just have each \gls{npc} deal its own \gls{xp} value in Damage each round (ignoring \gls{dr}).
A guard worth 10 \gls{xp} who fights alongside the characters deals 10 Damage, which could mean killing a single creature with 10 Damage, or could mean finishing off 2 creatures the characters have already wounded, by dealing each one 5 Damage.

\begin{exampletext}

	The \gls{gm} thinks for a moment.
	That's 30 goblins and 12 guards.
	The twelve guards are worth 10 \glspl{xp} each, so they deal 10 Damage each, killing 10 goblins.
	Then the 20 remaining goblins, worth 4 \glspl{xp} each, deal 40 Damage, killing 4 guards.

\end{exampletext}

If two \glspl{npc} fight, whichever individual is worth the most \glspl{xp} deals Damage first.
So if ten soldiers worth 10 \gls{xp} each fight a basilisk worth 24 \gls{xp}, the basilisk would deal 24 Damage, killing 2 soldiers.
On the next turn, the 8 remaining soldiers would deal 80 Damage, killing the basilisk.

\begin{exampletext}

	``The guards spill in, massacring the goblin horde.
	You see some surrounded, and spears driven into them, but the rest keep fighting.''

\end{exampletext}

Obviously, this system is not going to represent anything with much accuracy, but it's better than halting a game so you can roll dice for twenty minutes alone.

\subsection{Illusions}

Whether players are attempting to use illusions in combat, or trying to attack your \glspl{npc}'s illusions, the same rules apply; everyone attacks on the same initiative click.
If the players are attempting to attack the illusion of an armoured knight, the (illusory) armoured knight gets a low initiative counter, and any players acting at a particular step attack him.
If they hit (and they probably will), the illusion is vanquished, and the players are left with a wasted action.

Similarly, if a player attempts to cast an illusion of a strong man, and the horde of twenty goblins are acting on initiative 5, then each of them will attach the knight, and each of them pay the Initiative cost for attacking.

\subsection{Tactics}

Nobody like an opponent who's always letting them win.
A \gls{gm} pulling out three basilisks on new \glspl{pc} is bad form, but it's even worse when the players are allowed to win by poor tactics.

\subsubsection{Focus}

Basic tactics include two things: it's best to focus all attacks on single targets, and it's good to flank opponents whenever possible.

If the \glspl{pc} have left their anterior side exposed, enemies should spend initiative points to move to their side and allow half the group to flank the \glspl{pc}.
Don't parcel up opponents in a fair and even-handed way -- they're there to destroy the \glspl{pc}, so set them all against one, and if that player wants their character to survive, they'd best move back, or the other \glspl{pc} had better guard them.

If the \glspl{pc} want to survive, they'll need to take start stepping back at the right time, guarding each other, and killing faster.

\subsubsection{Throw}

Next up, remember the use of ranged weapons.
Everyone from thieves to goblins can throw spears, and if no spears are available, they can throw rocks.
If the creatures have no Projectiles Skill, they'll have a -1 penalty, but twelve goblins throwing rocks can still cause some damage.
Players will have to choose between spending Initiative on Keeping Edgy%
\footnote{See page \pageref{edgy}}
then evasion, or taking the hits and rushing forwards.

\subsubsection{Duel Wield \& Kick}

When you get down to 5 or 6 Initiative, making a single attack with a Medium weapon will put you below Initiative 1, so that would be your last attack.
Another option here is use to use a dagger, or simply kick someone.
This costs only 4 Initiative, putting you on 1 or 2, so you still get a chance to make your another attack later.

This secondary attack is often a bad idea for obvious reasons, but sometimes it's a great idea, and it's a good way to surprise players.

\subsubsection{Manoeuvres}

So you have twenty goblins facing off against four of the \glspl{pc}, but the \glspl{pc} have plate armour, a round shield, and a bad attitude.
They're invincible.
Their total Evasion Bonus is +7.
The battle looks hopeless, despite the goblins' tenacity, hunger, and greater numbers.
Now is the time to think tactically.

First, have the goblins attack with the \textit{Blind Rage} manoeuvre,\footnote{\nameref*{blindrage}.
Page~\pageref{blindrage}.} while making a \textit{Charge}.%
\footnote{\nameref{charge}. Page~\pageref{charge}.}
The first couple may die, but only one needs to hit.
They jump at one character's face and attempt to wrestle them.
It's not hard to pull a single goblin off, but while the goblin is grappling them, they'll count as being grappled, allowing the others to attack as a Sneak Attack.

Once multiple goblins are attacking as a Sneak Attack, they'll each receive +4 to attack, so the \gls{tn} to avoid them could move from 10 to 14, and any that hit will gain +2 Damage.

Remember to attack whichever \gls{pc} is the strongest in order to lower their initiative.
Once a \gls{pc} has run out of initiative, keep attacking so that the \gls{pc} gets a penalty for defending while below 0 initiative.

\subsubsection{Magical Chicanery}

At some point, the party spellcaster may put together a \textit{Massive, Range, Sentient, Murder-Spell}, and destroy anything in their path.
This isn't an abuse of the rules -- it's just intelligent use.

Firstly, remember that \glspl{xp} is limited to the first ten creatures killed in a single go.%
\footnote{See \nameref{xpCreatureMax}, page \pageref{xpCreatureMax}.}
Secondly, it's time to up the ante and miss out encounters.
If the party mage can handle 30 goblins with a single spell, just gloss over goblin encounters with a brief roll of the dice.
If the party mage somehow fails the spell-casting roll, run the encounter; otherwise, move on.
Focus on the encounters that still matter.
There are always creatures big enough to challenge a caster of any magnitude.

\end{multicols}

\section{The Players}

\begin{multicols}{2}

\subsection{Roll Before You Roleplay}

It's hard to play `the social character'.
You put all your \gls{xp} into a high Charisma score because you want to build alliances and understand people, then the \gls{gm} asks you to roleplay the encounter and all that comes out is your natural stutter.

It's also hard playing a non-social character.
You have been lumped with a character with a Charisma Penalty of -4 and by all the gods you intend to roleplay it, so it's time to ask the town master which lady he stole his robe from and then wipe your mouth with the tablecloth.
But the other players are not impressed; all they can see is someone intentionally ruining the encounter rather than the fun-loving, amazing improviser that you are.

Consider the following solution: tell the players that if they wish to speak, they must roll Charisma plus Empathy or Wits plus Whatever, then set the \gls{tn} for the encounter.
Getting information from the drunken patron of a temple of \gls{joygod} might be \gls{tn} 4 while getting a noble to stop and give everyone a hand might be \gls{tn} 10.
The player should not declare the result but make a mental note of the roll's Margin.
If the Margin is high, they should confidently roleplay someone saying just what the situation appears to demand.
On the other hand, if the roll was not only a failure but had a high Failure Margin, they should attempt to roleplay the worst kinds of insults -- perhaps because the character is genuinely mean-spirited, perhaps because they are making persistent, accidental faux-pas.

This method of players rolling before roleplaying to indicate their roll gives value to the social characters' Traits and legitimacy to the antics of more socially clumsy players saying all the wrong things.
The roll of the dice also acts as a way of saying `I am about to speak', so people can pace conversation without interruption.

\subsection{Damage, Death \& Dismemberment}

\subsubsection{Damage}

Losing \gls{hp} is a massive, screaming deal in BIND.
It's easy to take habits over from other games where losing one's liver is all part of a normal Tuesday afternoon but here \glspl{pc} should lose \glspl{fp}, then attempt to flee and only in the most dire situations should they start to bleed.
Damage which doesn't hit home can be brushed over with a brief note about `avoiding the swing' but if anyone loses a single Hit Point the \gls{gm} should grind the description and combat to a halt to emphasise exactly how eyeball poppingly, knee-cap shatteringly painful and side-splittingly debilitating a knife can be.
Take your time.
Make the words secrete congealed blood.
If the \glspl{pc} start to lose \glspl{hp} and don't realise how serious this situation is they might perish where they otherwise would have run away to fight another day.

\subsubsection{Death}
\label{pcdeath}
\index{Death}

Don't fear \gls{pc} death (or \gls{npc} death for that matter).
Character creation should be relatively easy, and no main plot-line should rely on a particular character.

If a \gls{pc} dies, the player should be slotted into the adventure at the next available opportunity as a known \gls{npc} from one of the \glspl{storypoint}.
This character is introduced as per the story \nameref{oldnpc} (see page \pageref{oldnpc}), so they will begin with half the total \glspl{xp} of whichever party member has the highest \glspl{xp} total (minimum 50).

If no \glspl{npc} have been established, anyone in the part can establish one immediately.
If none of the party have any \glspl{storypoint} left, the new character is established for free, with the same rules as if a \gls{storypoint} had been spent.

Players, rather than characters, keep their unspent \glspl{xp}, so any time a character dies, any unspent \glspl{xp} should be immediately given to the new character.
\Glspl{xp} received from spending \glspl{storypoint} do not reset, so if the old character had spent 4 \glspl{storypoint}, the new one would not receive any more \glspl{xp} from \glspl{storypoint} until they had spent 4.
In this way, the entire group should have a constant maximum number of points they can receive from \glspl{storypoint}.

\subsubsection{Dismemberment}

If a \gls{pc} is totally out of commission, with 1 \gls{hp} left, 4 \glspl{fatigue} from being bled dry, and an inexplicable curse, consider letting them play an \gls{npc} and letting them keep all \glspl{xp} gained during this time.


\subsection{\Glsentrytext{pc} Creation}

For a slightly more even spread of pluses and minuses across the party, consider rolling Attributes in pairs when making a character.

As you roll up Strength, you might select Intelligence as its opposite, and any gains in one become losses in the other.

	\begin{rollchart}

	Result & Attribute Bonus \\\hline

	2 & Strength -3, Intelligence +3 \\

	3 & Strength -2, Intelligence +2 \\

	4-5 & Strength -1, Intelligence +1 \\

	6-8 & Strength 0, Intelligence 0 \\

	9-10 & Strength +1, Intelligence -1 \\

	11 & Strength +2, Intelligence -2 \\

	12 & Strength +3, Intelligence -3 \\

	\end{rollchart}

For each Attribute you roll, you can select any as its opposite before rolling.

\end{multicols}

\section{Skill Use Cases}
\label{skill_uses}

\begin{multicols}{2}

\noindent
Below are some suggested uses for skills.

\subsection{Academics}

\paragraph{Area knowledge } -- Intelligence + Academics.
The character recalls local information about important sites.
Cities are TN 6, Towns are 8, and villages are 12.

\paragraph{Forgery} -- Dexterity + Academics, TN 8 for a signature (vs the interpreter's Wits + Academics).

\label{magicidentification}
\paragraph{Identifying Items} -- Intelligence + Academics, TN 10 (for Pocket Spells), 12 (for Talismans), or 14 (for Artefacts).
Magical items which do not come with instructions often remain enigmas.

A successful rolls allows someone to identify how to activate an item, but the roll requires a Margin of 2 to understand its effects.
Therefore, rolling a 13 when trying to understand a talisman means one understands how to activate it, but not what the talisman will do.

Specializations apply to this roll as usual, so someone with the Alchemy specialization receives a +2 bonus to the roll.

\paragraph{Storytelling} -- Charisma + Academics.

\subsection{Athletics}

\paragraph{Climb a wall} -- Speed + Athletics.

\paragraph{Planning the best climb up a mountain} -- Intelligence + Athletics.
A successful roll can lower the TN for others scaling a mountain equal to a third of the roll's Margin.

\subsection{Beast Ken}

\paragraph{Calm an animal} -- Charisma + Beast Ken vs animal's Wits + Aggression.

\paragraph{Taming a Horse} -- Intelligence + Beast Ken vs Horse's Wits + Aggression.

\paragraph{Wrestling a pig} -- Strength + Beast Ken, vs pig's Strength + Aggression.

\subsection{Crafts}

\paragraph{Breaking in a door} -- Strength + Crafts, \gls{tn} 10.

\paragraph{Crafting a sword} -- Strength + Crafts, TN 11.
This requires equipment, such as moulds, and a long night.

\paragraph{Creating a weapon mould} -- Intelligence + Crafts, TN equals 7 plus 2 per Initiative Bonus.

\subsection{Deceit}

\paragraph{Intimidating someone into backing off} -- Strength + Deceit vs the target's Strength + Empathy.

\paragraph{Quick thinking lies} -- Wits + Deceit, TN 10.

\paragraph{Well planned lie} -- Intelligence + Deceit, TN 7.

\subsection{Medicine}

\paragraph{Crafting a poison} -- Intelligence + Medicine, TN 4.

Each Margin inflicts 1 \gls{fatigue} on the target by the end of the scene.

\paragraph{Bandaging a wound} -- Wits + Medicine to stop someone bleeding, TN 7 plus the Damage which caused the bleeding.
Each Margin stops 1 point.
For example, someone stabs a man, inflicting 4 Damage, which then starts to bleed.
This could cause 4 \glspl{fatigue} in bleeding, and is TN ($7 + 4 = $) 11 to stop.
A healer rolls a grand total of 12, which stops one point of bleeding, so the man only gains 3 \glspl{fatigue}.

\paragraph{Curing a poison} -- Wits + Medicine, TN 10.

Each margin cures 1 \glspl{fatigue} caused by poison by the end of the scene.

\subsection{Larceny}

\paragraph{Picking a pocket} -- Dexterity + Larceny, TN 8 plus the target's Wits + Vigilance.

\paragraph{Snatch and run} -- Speed + Larceny TN 7, vs the target's Speed + Vigilance.

\subsection{Performance}

\paragraph{Complex recital} -- Dexterity + Performance.

\paragraph{Creating a new piece} -- Intelligence + Performance, TN 8.

\paragraph{Slow recital} -- Charisma + Performance, TN 11.

\paragraph{Rap battle} -- Wits + Performance, vs opponent's Wits + Performance.

\subsection{Stealth}

\paragraph{Ambush} -- Intelligence + Stealth, TN 10 for villages, 12 for a town, and 8 for a forest.

\paragraph{Finding a hiding spot} -- Wits + Stealth.

\paragraph{Planning a hidden route into a castle} -- Intelligence + Stealth.

\subsection{Survival}

\paragraph{Building a shelter} -- Intelligence + Survival, TN 11.
Each point on the Margin allows an additional person to sleep inside the shelter.

\paragraph{Navigation} -- Intelligence + Survival.
\index{Navigation}
\label{marching}
\begin{itemize}

	\item
		Mountains are \gls{tn} 9.
	\item
		Forests are \gls{tn} 12.
	\item
		Marshes are \gls{tn} 13.

\end{itemize}
Each failure margin adds 2 Miles to the journey time, so when trying to find a particular house somewhere in a forest, 10 miles away, the \gls{tn} would be 12.
If the roll is an 8, the actual journey would be 18 miles.

\paragraph{Climbing a tree} -- Speed + Survival, TN 8.

\paragraph{Gathering food} -- Intelligence + Survival, TN 9.
Each margin grants an additional day's food for one person.

\subsection{Tactics}

\paragraph{Planning an open battle} -- Intelligence + Tactics, TN 7 vs opponent's Wits + Tactics.

\subsection{Vigilance}

\paragraph{Keeping watch over the camp through the night} -- Strength + Vigilance, TN 7.

\paragraph{Finding a small opening in the dark} -- Dexterity + Vigilance.

\paragraph{Scouting the forest for an enemy camp nearby} -- Speed + Vigilance, TN 9.

\paragraph{Finding a hidden message in a book} -- Intelligence + Vigilance TN 7, vs opponent's Intelligence + Academics.

\end{multicols}

\section{Magic}

\begin{multicols}{2}

\subsection{Flavour}

The Mage Armour spell only has one effect -- it creates \glspl{SP} -- but it can have multiple \textit{flavours}.
\Gls{justicegod} protects his followers with radiant shields, while alchemists cast crackling energy-barriers arranged in precise, geometric forms.
Elves can naturally let their inner light flow out, while alchemists exchange darkness for light, and song mages translate radiant lyrics into brilliant performances.
A priest of \gls{justicegod} instils bravery in their men by commanding them to fight, and have them continue fighting until they die, while elves will trick a man into dancing, and then command him to dance until he passes out; both are casting \textit{Focus}.

It's important to remember the descriptions for spells, but especially important the first time someone casts one -- whether this is a \gls{pc} or an \gls{npc}.
Every initial casting of a spell should be a remarkable event.
Describe that first casting well, and the players can feel what you mean for the rest of the game.

People who exist `in the game', know full well that different paths of magic can lead to the same effects, but they still think of the `Blessing' spell in different ways if it's cast by a priest or a dragon.

\subsection{Which Things are Things?}

We know that enchantment spells target people, but others spells don't have such clear boundaries.
Conjuration spells cannot transform a person's  head without the rest -- whole people only.
\Glspl{miracleworker} cannot cast an illusion over a window; a spell must target the entire house or nothing.

Whenever boundaries become unclear, think of a word for the largest continuous object.
Houses make a town, but they have breaks between them.
A wall, on the other hand, cannot gain the Mage Armour spell unless the entire wall receives it.

\subsection{Magical Items}

\noindent
Spells may seem simple, but the way they can be chained together to create powerful magical items, \emph{isn't}.
With that in mind, here are a couple of examples of spells, and how a mage might cast them.

\magicitem{Finger of Undeath}% NAME
	{Extinguish}% SPELL
	{Divinity (\gls{deathgod})}% PATH
	{Instant}% DURATION
	{Pocket Spell}% TYPE
	{2}% Potency
	{5}% MP

The priest of \gls{deathgod} begins by creating a Mana Stone at level 2 from the finger-bone of a fallen warrior, which then stores 4 \glspl{mp}.
While the mana stone is still fresh, the mage casts `Pocket Spell' costing 5 \glspl{mp}, bringing the total casting cost to 8.
The spell infused is a \textit{Wide, Enervated, Ghoul Calling}, meaning it will be able to raise a group of people from the dead.
The caster's Intelligence Bonus of +2 means the spell will be able to raise up to 5 creature with 7 or fewer \glspl{hp} (the spell is level 3 in total, and the item's Potency is 2).

Casting this level 3 spell into the item brings the total cost up to 11.
Such a high cost will almost always demand a steady supply of \glspl{mp}, either from others helping with the spell, or from Mana Stones.


\magicitem{The Silent Forest}% NAME
	{Empathic, Trans-Species Animal Transformation}% SPELL
	{Divinity (\gls{naturegod})}% PATH
	{4 Scenes}% DURATION
	{Massive, Potent, Sentient, Talisman}% TYPE
	{3}% Potency
	{3}% MP

A powerful druid, with +3 Intelligence, begins with a \textit{Massive, Mana Stone}.
She casts Mana Stone at 2nd level, so 2 \glspl{mp} will be locked as long as the spell lasts.
However, she decides to make this a \textit{Massive} spell, so it stretches across 5 entire areas of a forest -- the tended grove, the river, the edge of the lake, the boggy patch, and the local hill.
As this entire area is her Mana Stone, she spends 7 \glspl{mp} to cast Talisman, making this a \textit{Massive Talisman}.
Finally, the \textit{Empathic, Trans-Species, Animal Transformation} spell is laid into the area.
The total cost is 14 so far, which puts her near the absolute limit, given the Mana Stones she already has (a pet raven, and a pet frog).

At this point, the magical `item' (forest area) would normally be usable by anyone inside who uses the command word.
But that's not what she wants -- instead, she makes the spell \textit{Sentient}, for a total cost of 15 \glspl{mp} (her absolute limit).
This means the forest will target anyone it likes (so most people), especially if it means something which \gls{naturegod} would approve of.

The spell rolls with a +3 Bonus against TN 7, so it will mostly succeed.
It can only work on one person at a time, but that's enough of a threat to keep everyone in the area silent (except for alchemist \glspl{miracleworker}, who don't need articles in their spells).

\iftoggle{verbose}{
	
\begin{exampletext}

  ``Do you think the village on the other side of the mountain is safe to visit?'', asked Sean with raised eyebrows.
  
  His companions did not really want to hear that question, but they had.
  It was impossible to tell from this distance if the hobgoblins had settled there already.
  
  ``Ah dinnae ken, laddie''
  
  \pic{Boris_Pecikozic/dwarves_meet}{\label{boris:meet}}

  Hugi is resolved to just enter the next area, stoically, but his player is no stoic.
  It is decided that now is the time to expand Hugi's backstory.
  He wants a place to rest, he wants more of an idea of what is happening here.
  He decides to spend a Story point to specify that a single dwarven outpost has a single person still there.
  
  ``Why is just one person in an outpost?'', the \gls{gm} asks.
  
  ``Well, it's my cousin.
  She was inside at the time. When the scouts returned from watching the side of the mountain, they all got eaten by hobgoblins.
  Only after that she managed to escape, helped by the men-dwarves. So she's alone in the outpost''
  
  ``So this is a safe space story?'', the \gls{gm} asks.
  
  ``No. No I just want to spend one Story point and get someone with a normal place to stay, and knows a little about what's going on, and maybe some knowledge of Medicine''.
  
  Hugi's player marks off a single \gls{storypoint} and starts telling his story.
  
  ``There's an outpost over there'', Hugi remarked.
  ``It looks mostly like the mountain but you can see a little dark bit that's too straight-cut.
  They're little windows.''
  
  Entering the building, Hugi found his cousin, Magda.
  Sean expected them to hug after the ordeal, but Hugi just bowed low.
  Apparently he was proud of the honour of gathering news from her on account of their shared blood.
  
  As luck would have it, she was a proficient medic, and helped patch Hugi back up, safely removing the arrow.
  
While all the players are thinking about the next move, the \gls{gm} adds up their \gls{xp}. They defeated 6 hobgoblins and 1 ogre. Hobgoblins and ogres are worth 7 each. There were seven in total, so that means 49 \gls{xp} in total, minus one \gls{xp} per member of the group. The final result is that each character receives 15 \gls{xp} and one more \gls{xp} is left in the pot for later (because 46 cannot evenly be divided by 3). After that, each player wants a little additional \gls{xp} for following their own God or codes.

Arneson follows the Goddess, \gls{naturegod}.
He receives 3 additional \gls{xp} because the hobgoblins are particularly hated enemies for him - followers of \gls{naturegod} believe they are either unnatural, or that their presence in the human realm is unnatural.
Hugi, meanwhile, follows the Code of the Tribe; what's important to him is his dwarvish clan's honour.
In coming here he has defended his tribe's honour and claims 3 \gls{xp} for coming to the rescue of dwarves in the name of his own tribe.
He is additionally helping a particular member of the tribe whom he has met a long way from home.
That's another 2 \gls{xp}.
He believes his arrival has saved this cousin.
The \gls{gm} thinks this is plausible, since his cousin Magda was previously stranded with little food.
This grants him another 5 \gls{xp}.
That's a total of 7 \gls{xp}.

Hugi decides to spend his 7 on his first Knack, and selects `Chosen Enemy (Goblins)'.

Meanwhile, Arneson purchases Dexterity +1 with his 15 \gls{xp}. The group is a little older and wiser, and are more confident about meeting danger in the future.

Hugi was filled with pride to the point of forgetting about the pain when Magda pulled out the arrow which had so deeply penetrated his shoulder. He was almost caught smiling when Magda bandaged up the ogre's teeth-marks on his face - it would make a good scar.

The band took only a couple of hours before they set off again, hoping to find that village, somewhere beyond the mist. What had happened to that bard, they could only guess, but there seemed little chance of finding him in that village.

\end{exampletext}


}{}

\end{multicols}
