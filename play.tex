\chapter{How It Goes}

\begin{multicols}{2}

\sideBySide{
  Scheduling becomes much easier when you just run a game with anyone who's there.
  So BIND has a little `restart' at the start of each session.

  \Gls{fenestra} has twelve seasons (which map to our twelve months) so everyone can feel time marching on.
  And it marches on regardless of who's at the table that week -- even if nobody's there, \gls{fenestra} keeps moving.

  This copy of BIND's core rules was created during one \showTemperature\ \showSeason.
}{
  \begin{description}
    \item[\gls{gm}:]
    Fenestra is currently experiencing the freezing season of Qualmea, when the trees shed their leaves, letting sunlight shine on shining, white roads.

    \item[Mat:]
    Wasn't it raining last time? Are we jumping time again?

    \item[\gls{gm}:]
    Yes - three weeks passes every session, so we've actually started a new 'cycle'.
  \end{description}
}

\sideBySide{
  Character creation's fast, so don't worry about your \gls{pc} dying.
  Most will begin as members of the \gls{guard} -- the place they put society's unwanted; the useless miscreants, bastards, and political agitators.

  The character creation rules are in the book of stories%
  \iftoggle{stories}{, \autopageref{character_rolls}.}{.}
  This book just has a lot of notes on Traits, and details on resolution
  (it's mostly good for reference).
}{
  \begin{description}
    \item[Nina]
    Do I get a character?
    \item[\gls{gm}:]
    You sure do.
    Take the book of stories and roll $2D6$ on these charts.
    \item[Nina:]
    Okay, I'm a human\ldots called\ldots `Sootfilch'.
    What's with the name?
    \item[\gls{gm}:]
    People in \gls{fenestra} believe that the gods takes the best in life, so it's bad luck to name their children after good things.

    You can roll the stats up next.
    There's a guide on the character sheet as well as the players' book.
  \end{description}
}

\sideBySide{
  Starting equipment and beliefs all come wrapped together with the core concept in character creation.
  This lets new people get up and running quickly.

  Some of them will grow to like their characters, but if they don't, they can always build their own character, with everything they want, after the first.

  Starting pay for the \gls{guard} is roughly nothing.
  This forces the \glspl{pc} into some side-jobs and creative thinking, if they want to get their hands on reasonable weapons and armour.
}{
  \begin{description}
    \item[\gls{gm}:]
    Low Dexterity and high Intelligence, so we'll look up what that says.
    \item[Nina:]
    It says I'm a `Loner'.
    \item[\gls{gm}:]
    That makes sense.
    Society and social ties are very important in \gls{fenestra}, so people can end up in the \gls{guard} just because they don't have ties to an employer.
    \item[Nina:]
    This equipment looks a bit rubbish.
    Can't I get something better?
    At least some proper armour?
  \end{description}
}

\sideBySide{
  The `role-playing' aspect works as usual (i.e. you do you).
  However, there's a light suggestion, especially for new players, that they should roll first and interpret the dice.

  The \gls{gm} doesn't have an exact reference for asking for armour, but since \glspl{jotter} tend to be harsh, he sets the \gls{tn} to 10.
}{
  \begin{description}
    \item[\gls{gm}:]
    Sure -- you're in a \gls{broch}, so you can ask the \gls{jotter} here.
    She does all the paperwork for the \gls{guard}, and decides who gets weapons.

    Roll $2D6$, then add your \roll{Charisma}{Empathy}.
    \item[Nina:]
    \ldots that's an `11' in total?
    Do I pass?
    \item[\gls{gm}:]
    Sure, but how would a loner ask for equipment?
    What would they do?
    \item[Nina:]
    He's probably ask her when she's alone.
    Knock politely, and just explain he's not been given the tools to survive.
  \end{description}
}

\sideBySide{
  The \gls{gm} forgot to prepare anything for the session, but that's okay.
  He's already determined the season, so he flips to the encounter charts and rolls $3D6$.
  The result is: `a woodspy, with thunder, after 3 \glspl{interval}'.
  The \gls{gm} notes that down on his sheet, then looks up the mission.

  Rolling $3D6$ again, he finds the troupe must go to two nearby \glspl{village} and cut away at the perimeter.
}{
  \begin{description}
    \item[\gls{gm}:]
    \Gls{jotter} Cartpike agrees to give you the armour, as long as you head out on the new mission, immediately.
    Sootfilch, Grimesinge, and Mossboke will journey out to two neighbouring \glspl{village} -- that means a `walled village' -- and cut around the perimeter.
    \item[Mat:]
    We're bush-whacking?
    \item[\gls{gm}:]
    Yes.
    If the forest grows too dense, and approaches \pgls{village}, it means predators can get close, and attack the farmers there.
    One lies eight miles North, the other is 10 miles south, with \pgls{bothy} half way along.

    \Gls{jotter} Cartpike gives everyone a day's rations (hardened cheese and a pie), and lets you pick your poison.
  \end{description}
}

\sideBySide{
  Travelling times are easy to work out.
  Players can decide their characters travel any number of miles, but every two miles adds \pgls{fatigue}.
}{
  \begin{description}
    \item[Mat:]
    May as well do the distant one first.
    Can we walk ten miles in an afternoon?
    \item[\gls{gm}:]
    Sure, that's just 5 \glspl{fatigue} along the road.
    And an extra 1 \glspl{fatigue} for the freezing weather.
    
  \end{description}
}

% Encounters
% RP moment with dice
% Travel
% Combat
% Stories
% 

\end{multicols}
