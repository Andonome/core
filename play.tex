\chapter{So It Goes}

\begin{multicols}{2}

\noindent
This is how a game might go.
\glsadd{guard}
Think of this as a skeleton of a game, or just a standard introduction.
Once you get into the swing of this simple cycle -- march, encounter, actions, repeat -- the simple resolution system can fill days or weeks of in-game time quite easily.

How much the troupe actually engage with their \gls{guard} missions depends on the players -- they may lean into them, and feel pressures on all sides as they juggle a chaotic world with their duties, or may end up abandoning them, and wandering \gls{fenestra} as free agents or bandits (and who can tell the difference?).

\sideBySide{
  \begin{description}
    \item[\gls{gm}:]
    Fenestra is currently experiencing the freezing season of Qualmea, when the trees shed their leaves, letting sunlight shine on roads which are slowly turning white, with little specks of snow.

    \item[Matt:]
    Wasn't it raining last time? Are we jumping time again?

    \item[\gls{gm}:]
    Yes - three weeks passes every session, so we've actually started a new `\gls{cycle}' in \gls{fenestra}.
    And as usual -- we'll start in \pgls{broch} -- one of the \gls{guard} towers which sit between \glspl{village}.
    \glsadd{broch}
    \glsadd{village}

    Mark off any rations or \glspl{ingredient} from last session -- they've all been eaten or gone rotten.
  \end{description}

}{
  Scheduling becomes much easier when you just run a game with anyone who's there.
  So BIND has a little `restart' at the start of each session.

  \Gls{fenestra} has twelve seasons (which map to our twelve months) so every time we enter January, \gls{fenestra} enters \gls{Qualmea}.
  And it marches on regardless of who's at the table -- even if nobody's there, \gls{fenestra} keeps moving.

  This helps keep the table open, so Soibhan -- a new player -- can jump right in.
}

\sideBySide{
  \begin{description}
    \item[\gls{gm}:]
    The \gls{broch}'s \Gls{jotter} wants a report from last week.
    \glsadd{jotter}

    What did you do?
    \item[Alice:]
    Well, Grogfen helped a trader getting to town\ldots
    (but I don't want to tell her about entering a town, or the \glspl{sp} she paid us)
    \item[\gls{gm}:]
    So how will you explain how long the mission took?
    Remember, she has a record of when Grogfen left the \gls{jotter} who gave her the mission.
    \item[Alice:]
    With lies\ldots?
    Can I just roll this one?
    I'll still tell her about the strange markings on all the trees.
    \item[\gls{gm}:]
    Sure.
    You just need to keep it consistent with what she knows.
    Roll \roll{Intelligence}{Deceit} at \glsentrytext{tn} 8.
    \item[Alice:]
    \dicef{7}
    Okay, that's an `8', so\ldots
    \item[\gls{gm}:]
    That's a tie, I'm afraid.
    You got what you wanted (the money), but \gls{jotter} Cartpike thinks you're lying.
    The two of you end up in an argument, with accusations, but she can't prove anything for now.
  \end{description}
}{

  Whenever the characters try to accomplish something dangerous, the player rolls:

  $$2D6 + \underbrace{Attribute + Skill} ~vs~ \glsentrytext{tn}$$

  When players roll above the \gls{tn}, their character succeeds.
  When they roll below, the character fails, and the danger occurs.
  When they roll a tie, then both occur -- or neither.

  Sometimes this depends on what makes sense.
  If both make sense then the player can choose to accept the danger and their goal, or neither.

  In this case, Grogfen has a -1 Penalty to Intelligence, and +2 Deceit.
  So in total, the roll is:

  $$2D6 + 1 ~vs~ 8$$

  Sometimes a piece of equipment or a social contact can add to the roll, so the total Bonus can become very high.
  A \gls{tn} of 14 is not unreasonable in some situations.

  The full formula for an action is:

  $$2D6 + \underbrace{Attribute + Skill + Equipment} ~vs~ \glsentrytext{tn}$$
}

\sideBySide{
  \begin{description}
    \item[Soibhan]
    Do I get a character?
    \item[\gls{gm}:]
    You sure do.
    Take the Book of Stories and roll $2D6$ on these charts.
    \item[Soibhan:]
    Okay, I'm a human\ldots called\ldots `Sootfilch'.
    What's with the name?
    \item[\gls{gm}:]
    People in \gls{fenestra} believe that the gods takes the best in life, so it's bad luck to name their children after good things.

    Roll the Attributes next.
  \end{description}
}{

  Character creation's fast, so don't worry about \glspl{pc} dying.
  Most will begin as members of the \gls{guard} -- the place they put society's unwanted; the useless miscreants, bastards, and political agitators.

  The character creation rules are in the \textit{Book of Stories}%
  \iftoggle{stories}{, \autopageref{character_rolls}.}{.}
  This book focusses on resolution mechanics, details of Traits, and spells.
}

\sideBySide{
  \begin{description}
    \item[\gls{gm}:]
    Low Dexterity and high Intelligence, so we'll look up what that says.
    \item[Soibhan:]
    It says I'm a `Loner'.
    \item[\gls{gm}:]
    That makes sense.
    Society and social ties are very important in \gls{fenestra}, so people can end up in the \gls{guard} just because they don't have anyone arranging a safe position for them.
    \item[Soibhan:]
    This equipment looks a bit rubbish.
    Can't I get something better?
    At least some proper armour?
  \end{description}

}{

  Starting equipment and beliefs all come wrapped together with the core concept in character creation.
  This lets new people get up and running quickly.

  Most players end up making more interesting characters with some random input, but if players really want to decide every facet of their character, they can use the `point-buy' character creation system.

  Starting pay for the \gls{guard} is roughly nothing.
  This forces the \glspl{pc} into some side-jobs and creative thinking, if they want to get their hands on reasonable weapons and armour.
}

\sideBySide{
  \begin{description}
    \item[\gls{gm}:]
    Sure -- you're in a \gls{broch}, so you can ask \gls{jotter} Cartpike.

    Roll $2D6$, then add your \roll{Charisma}{Empathy}.
    \item[Soibhan:]
    \dicef{9}
    \ldots that's an `11' in total?
    Do I pass?
    \item[\gls{gm}:]
    Sure, but how would a loner ask for equipment?
    What would they do?
    \item[Soibhan:]
    He's probably ask her when she's alone.
    Knock politely, and just explain he's not been given the tools to survive.
  \end{description}

}{
  The \gls{gm} doesn't have an exact reference for asking for armour, but since \glspl{jotter} tend to be harsh, he sets the \gls{tn} to 10.

  Some people like to `roll for Charisma', because they want their characters to succeed, rather than make a performance.
  Instead of asking for acting talent, I've found it's best to have people roll, then interpret that roll.
  This lets players plan for their characters abilities (as usual), and opens the field for failures, without demanding that players fail in the party's rolls because `it's what my character would do'.
  Or rather, it provides mechanical justification for `what my character would do'.
}

\sideBySide{
  \begin{description}
    \item[\gls{gm}:]
    \Gls{jotter} Cartpike agrees to give you the armour, as long as you head out on the new mission, immediately.
    Sootfilch, Grimesinge, and Mossboke will journey out to two neighbouring \glspl{village} -- that means a `walled village' -- and cut around the perimeter.
    \item[Matt:]
    We're bush-whacking?
    \item[\gls{gm}:]
    Yes.
    If the forest grows too dense, and approaches \pgls{village}, it means predators can get close, and attack the farmers there.
    One lies eight miles North, the other is 10 miles south, with \pgls{bothy} half way along.

    \Gls{jotter} Cartpike gives everyone a day's rations (hardened cheese and a pie), and lets you pick your poison.
  \end{description}

}{
  The \gls{gm} forgot to prepare anything for the session, but that's okay.
  He's already determined the season, so he flips to the encounter charts and rolls $3D6$.
  The result is: `4 trader caravans, and biting winds, after \pgls{interval}'.
  The \gls{gm} notes that down on his sheet, then looks up the mission.

  For the mission, he rolls $3D6$ again, and finds the troupe must go to two nearby \glspl{village} and cut away at the perimeter.

  Finally, he rolls up a single \gls{village} while two players explain how armour works to Soibhan.
  The \gls{village} has bear traps surrounding it (that might make expanding the perimeter challenging!) and rumours about the local swamp-hag who occasionally eats people.
  Since it's near a marsh, he names the \gls{village} `Marshwall', and notes it down on his expanding map.

  The various encounter charts and random tables shouldn't be considered `hard rules' -- they exist to support the \gls{gm}.
  You can find the random rolls in the \textit{Book of Judgement}%
  \iftoggle{judgement}{, \autoref{encounters}, \autopageref{encounters}}{}.
}

\sideBySide{
  \begin{description}
    \item[Matt:]
    May as well do the distant one first.
    Can we walk ten miles in an afternoon?
    \item[\gls{gm}:]
    Sure, that's just 5 \glspl{fatigue} along the road.
    Sootfilch and Grogfen are human, so they can ignore the first \gls{fatigue}.
  \end{description}
}{
  Travelling times are easy to work out.
  Players can decide their characters travel any number of miles, but every two miles along a road adds \pgls{fatigue}.

  If they want to hard-march 20 miles in an afternoon, they can do so, but they'll receive 10~\glspl{fatigue}, which means serious penalties.
}

\sideBySide{
  \begin{description}
    \item[\gls{gm}:]
    So everyone heads off to Marshwall, and on the road traders try to sell you torch pitch and rope for 100~\glspl{cp} each.
    \item[Matt:]
    Nah we're\ldots actually, that'd be really handy.
    I'll take both.
    \item[\gls{gm}:]
    Roll \roll{Wits}{Crafts} to determine the quality of the goods (\tn[9]).
    \item[Matt:]
    \dicef{7}
    Already?
  \end{description}
}{
  Purchases can get tricky, as tricksters are everywhere.
  A bad roll can leave the buyer holding useless goods, or buying services from a chancer.

  Most of the `secondary skills' (like Crafts and Wyldcrafting) can come in useful for trades.
}

\sideBySide{
  \begin{description}
    \item[\gls{gm}:]
    Snow's started to fall, thicker than before.
    By evening, the road is white, but you can see Marshwall's high wooden walls ahead, and an archer waves at you from his perch.
    \item[Matt:]
    Time to rest.
    \item[\gls{gm}:]
    Wait a minute\ldots
    just checking for encounters.
    \item[Matt:]
    Crap\ldots
    \item[\gls{gm}:]
    Who's in the lead?
    Grogfen has the highest Speed, so I think she would be.
    Can you roll \roll{Wits}{Vigilance} (\tn[13])?
    \item[Alice:]
    \dicef{3}\dicef{2}
    Crap\ldots
  \end{description}
}{
  When two people (or creatures) act against each other, the player rolls at \tn[7], plus their opponent's score.

  $$2D6 + \underbrace{-1 + 0} ~vs~ 7 + \underbrace{2 + 2 + 2}$$

  The creature's total Bonus is +6 (with a snow-camouflage Bonus), so Alice must roll at \gls{tn} 13.
}

\sideBySide{
  \begin{description}
    \item[Alice:]
    I rolled a `4', so what happens to me?
    \item[\gls{gm}:]
    A bleach-white tentacle grabs you by the neck, another around your left leg, then the great woodspy rises.
    \item[Soibhan:]
    The what?
    \item[\gls{gm}:]
    ``Woodspy'' -- a great land-octopus, able to camouflage, with a load of tentacles.
    It grabs her and starts to slither away, yanking its way through the trees while holding Grogfen tight above its head.
    \item[Soibhan:]
    What do we do?
    Run after it?
    \item[\gls{gm}:]
    If you want to, spend \pgls{ap}.
    You can roll \roll{Speed}{Athletics} at \tn[11].
    \item[Soibhan:]
    Okay, \dicef{7} \ldots sorry, Alice\ldots
  \end{description}
}{
  Everyone starts combat with 3~\glspl{ap} plus their Speed Bonus.
  Having more \glspl{ap} lets you act before others, and do more.

  Most combat actions resist an opponent, so spending \pgls{ap} usually forces an opponent to spend one too.
  These forced-expenditures can push characters into negative \glspl{ap}.
  A character on negative \glspl{ap} feels seriously flustered, and takes the negative as a penalty to their actions.

  For example, a character on -2~\glspl{ap} would take a -2 penalty to all actions (and could only act in response to something acting against them).

  Using \glspl{ap} means you don't need to be too precious about which order peole act in -- every character receives the actions they're due by the end of the round.
}

\sideBySide{
  \begin{description}
    \item[Matt:]
    Wait, Earth magic covers snow, doesn't it?
    Can I make the snow solid around the woodspy's tentacles to stop it getting away?
    \item[\gls{gm}:]
    Yes -- it's still \tn[11], but you can roll your \roll{Charisma}{Earth} to cast.
    \item[Matt:]
    \dicef{9} Got it!
    \item[\gls{gm}:]
    What does he say?
    \item[Matt:]
    `Solid frost, and woodspy tomb'?
    \item[\gls{gm}:]
    Well the snow responds, freezing solid, and holding a couple of tentacles tight.

    Soibhan -- one \roll{Wits}{Vigilance} roll, please (\tn[11]).
  \end{description}
}{
  Spells work like any other roll, including when making a Resisted roll -- the spellcaster uses their Bonus to resist the opponent, or players roll their \gls{pc}'s Bonus against the \gls{npc}'s.

  Characters cast spells with Charisma plus an elemental Skill, as they literally speaks to the elements.
  Most spells come with a suggested statement, but the player can fill in anything (or nothing).
}

\sideBySide{
  \begin{description}
    \item[Alice:]
    I'm rolling to attack!
    \item[\gls{gm}:]
    Roll Brawl, \tn[13].
    \item[Alice:]
    \dicef{9}
    \Glsentrylong{tn} what?
    This is hopeless\ldots
    \item[\gls{gm}:]
    Yes, but then again, you and the woodspy both spend \pgls{ap} when you struggle.
    \item[Alice:]
    Okay, I spend every \glspl{ap} I have.
    \dicef{3}\dicef{2}
    All of them.
    \dicef{8}.
  \end{description}
}{
  Everyone starts combat with 3~\glspl{ap} plus their Speed Bonus.
  Having more \glspl{ap} lets you act before others, and do more.

  Most combat actions resist an opponent, so most involve both combatants spending \pgls{ap} as the same time.
  Characters can even go into the negative, but having -2~\glspl{ap} means a -2 Penalty to all rolls.
}

\sideBySide{
  \begin{description}
    \item[Soibhan:]
    \dicef{5}\dicef{1}
    Does a `5' pass?
    \item[\gls{gm}:]
    A bear trap, hidden in the snow, leaps up dealing\ldots\dicef{6} 6 Damage.
    You can remove your 5~\glspl{fp}.
    \item[Soibhan:]
    And the last Damage?
    Does my armour get it?
    \item[\gls{gm}:]
    Not with a bear-trap I'm afraid.
    Remove \pgls{hp}.

    However, since the woodspy couldn't move, you're standing in front of it, with one leg bleeding from the frozen, iron jaws.
  \end{description}
}{
  \Glspl{fp} measure a character's distance from death and (in some sense) their courage.
  Once they run out, all further damage makes a real wound -- Sootfilch might carry this damage for the remainder of the session.
  However, after \pgls{interval} the troupe will regenerate \glspl{fp}, so the wounds won't leave them near-death for the entire session -- they can persevere while wounded, and rely on their luck.

  BIND doesn't have much in the way of healing magic, but it has plenty of Fate spells.
}

\sideBySide{
  \begin{description}
    \item[Soibhan:]
    I'll stab the woodspy.
    \item[\gls{gm}:]
    Okay -- spend \pgls{ap} to take out your sword.
    \item[Soibhan:]
    Spent.
    I'll stab the woodspy!
    \item[Alice:]
    \dicef{11}
    I'm free, but out of \glspl{ap}.
    \item[Matt:]
    I'll go for another binding spell, can I make it trapped so Sootfilch can stab it easier?
    \item[\gls{gm}:]
    Sure, and it's at \tn[10] this time.
    \item[Soibhan:]
    Can I not stab the woodspy?!
    \item[\gls{gm}:]
    Snow freezes around its every grounded tentacle, while Grogfen pulls away.
    Roll \roll{Dexterity}{Combat} at \tn[7].
    \item[Soibhan:]
    \dicef{9}
    That's 8, how do I Damage?
    \item[Matt:]
    $1D6$!
    \item[Soibhan:]
    \dicef{6} that's an 8 in total, with the shortsword.
    \item[\gls{gm}:]
    The shortsword enters, and its snow-white skin splits, blue blood runs down the wound and the skin writhes, turning red, black, then mottled-brown.
  \end{description}
}{
  The woodspy would usually flee at this point, but being bound by ice, and trapped, the troupe will destroy it in a moment.

  Once it dies, Laiquon will pull its beak out, to create \pgls{talisman} for a Water spell, then the troupe can sell the rest of the body to the farmers in the \gls{village} for a few copper and a couple of favours, like a nice place to rest.

  The \gls{gm} hasn't introduced the real hooks yet -- the \gls{sq} \glspl{segment}.
  Last session saw several little plot-\glspl{segment} which provided foreshadowing for the swamp-hag's plans.
  The next \gls{segment} in the forests will be the reckoning, but if the players decide to stick to their mission, on the roads, the \gls{gm} will have to start a new \gls{sq} to throw at them.

  Running multiple \glspl{sq} isn't a problem.
  The old plot can wait, and new \glspl{guard} will hear about the important background from the \glspl{jotter} records.
}

\end{multicols}
