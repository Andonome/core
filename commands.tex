\makeindex[name=spells,title={Spell Summaries},columns=2]

% for CS callout boxes

\newcommand{\initiativechart}{

  \begin{nametable}[Xc]{\Gls{ap} Costs}

  \textbf{Standard Actions} & \textbf{\Gls{ap}} \\
  \hline

  Attacking      & by weapon \\

  Drawing weapon & 1 \\

  Guard Someone & 1 \\

  Ram & 3 \\

  \hline
  \hspace{3em}\textbf{Projectiles} & \\
  \hline

  Crossbow & 1 \\

  Improvised projectile & 2 \\

  Longbow   & 4 \\

  Reloading & 1 \\

  Shortbow & 2 \\

  Thrown weapon & 4 \\
  \hline

  \hline
  \textbf{Other Actions} & \\\hline

  Cast a spell & $1 + Lv - Wts$ \\

  Moving & 1 \\

  Speaking & 1 \\

  \end{nametable}

}

\newcommand\improvisedWeaponsChart{

  \begin{nametable}[XXXXX]{Improvised Weapons}

  \textbf{Name} & \textbf{Attack Bonus} & \textbf{Damage Bonus} & \textbf{\Glsfmtshort{ap} Cost} & \textbf{\Gls{weight}} \\\hline

  \showWeapon{\boulder} \\

  \showWeapon{\chair} \\

  \showWeapon{\club} \\

  \showWeapon{\cudgel} \\

  \showWeapon{\firepoker} \\

  \showWeapon{\skillet} \\

  \showWeapon{\knife} \\

  \showWeapon{\Log} \\

  \showWeapon{\rock} \\

  \showWeapon{\stick} \\

  \showWeapon{\woodaxe} \\

  \end{nametable}

}

\newcommand{\weaponsChart}{
  \begin{nametable}[XXXXX]{M\^{e}l\'{e}e Weapons}

  \textbf{Name} & \textbf{Attack Bonus} & \textbf{Damage Bonus} & \textbf{\Glsfmtshort{ap} Cost} & \textbf{\Gls{weight}} \\\hline

  \showWeapon{\Dagger} \\

  \showWeapon{\glaive} \\

  \showWeapon{\greataxe} \\

  \showWeapon{\greatsword} \\

  \showWeapon{\javelin} \\

  \showWeapon{\longsword} \\

  \showWeapon{\maul} \\

  \showWeapon{\poleaxe} \\

  \showWeapon{\quarterstaff} \\

  \showWeapon{\shortsword} \\

  \showWeapon{\spear} \\

  \end{nametable}
}

\newcommand\esotericWeaponsChart{
  \begin{nametable}[XXXXX]{Esoteric Weapons}

  \textbf{Name} & \textbf{Attack Bonus} & \textbf{Damage Bonus} & \textbf{\Glsfmtshort{ap} Cost} & \textbf{\Gls{weight}} \\\hline

  \showWeapon{\greatclub} \\

  \showWeapon{\giantsword} \\

  \showWeapon{\rapier} \\

  \showWeapon{\warhammer} \\

  \showWeapon{\whip} \\

  \end{nametable}
}

\newcommand{\shieldchart}{
  \begin{nametable}[LYYY]{Shields}

  \textbf{Name} & \textbf{Defence Bonus} & \textbf{\Glsfmtshortpl{ap} Cost} & \textbf{Weight} \\\hline

  \bucklar \\
  \roundshield \\
  \kiteshield \\

  \end{nametable}
}

\newcommand{\armourchart}{

  \begin{boxtable}[Lccc]

  \textbf{Armour} & \textbf{\Glsentrytext{dr}} & \textbf{Covering} & \textbf{Weight} \\\hline

  \ifnum\value{r4}=3
    \armour[\addtocounter{weight}{-1}]{Elvish Ceramic}{2}{2} & \arabic{armourDR} & \arabic{covering} &  \arabic{weight} \\
  \fi

  \showArmour{\armour[\addtocounter{weight}{1}]{Padded Armour}{2}{2}} \\

  \showArmour{\armour{Partial Leather}{3}{2}} \\

  \showArmour{\armour{Complete Leather}{3}{4}} \\

  \showArmour{\armour{Partial Chain}{4}{2}}  \\

  \showArmour{\armour{Complete Chain}{4}{4}} \\

  \showArmour{\armour{Partial Plate}{5}{2}} \\

  \showArmour{\armour{Complete Plate}{5}{4}} \\

  \end{boxtable}

}

\newcommand{\chasechart}{

  \begin{nametable}{Chase Chart}

  Total & Result \\\hline

  11+ & The characters immediately escape their pursuers. \\

  10 & The characters escape their pursuers after travelling through 2 areas. \\

  9 & The characters escape their pursuers after travelling through 3 areas. \\

  8 & The characters are chased through 3 areas and reroll. \\

  7 & The characters are chased through 2 areas and reroll. \\

  6 & The characters are chased through 1 area and reroll. \\

  {\textless}5 & The characters are immediately caught. \\

  \end{nametable}

}

\newcommand{\huntchart}{

  \begin{boxtable}

  Total & Result \\\hline

  10+ & The characters immediately capture their quarry. \\

  9 & The characters chase their quarry through 3 areas, then capture them. \\

  8 & The characters chase their quarry through 2 areas before catching up with them. \\

  7 & The characters chase their quarry through 1 area and then catch up with them. \\

  6 & The characters chase their quarry through 3 areas, then lose them. \\

  5 & The characters chase their quarry through 2 areas before losing them. \\

  4 & The characters chase their quarry through 1 area, then lose them. \\

  {\textless}3 & The characters immediately lose their quarry. \\

  \end{boxtable}

}

\newcommand{\projectilesChart}{
\begin{boxtable}[XXXXX]

  \textbf{Projectile} & \textbf{\Gls{ap}} & \textbf{Damage} & \textbf{Weight}  & \textbf{Reload Time} \\\hline

  Crossbow &  1 & $2D6$ & 1 & 5 - Str \glspl{round}  \\

  Longbow &  4 & varies & -4 & 4 - extra Str \glspl{ap} \\

  Shortbow &  1 & $1D6-1$ & -5 & 1 \gls{ap} \\

  Throwing knives & 2 & $1D6$ & -5 & 1 \gls{ap} \\

\end{boxtable}
}



\newcommand\larcenyChart{
  \begin{nametable}[cccL]{Larceny Roll}
    \textbf{Village} & \textbf{Town} & \textbf{City} & \textbf{Result} \\
    \hline
     17 & 15 & 14 & $2D6 \times 20$ \gls{cp} from a noble's servant. \\
     16 & 14 & 13 & $2D6 \times 15$ \gls{cp} from a traveller. \\
     15 & 13 & 12 & $2D6 \times 10$ \gls{cp} from a trader. \\
     14 & 12 & 11 & $2D6 \times 5$ \gls{cp} from an old lady. \\
     13 & 11 & 10 & No good targets found \\
     12 & 10 & 9 & Caught red handed! -- roll a `snatch and run'. \\
     11 & 9 & 8 & Caught red handed and surrounded! \\
  \end{nametable}
}

\newcommand\gatheringChart{
  \begin{nametable}[ccX]{Gathering Table}
    \textbf{Tundra} & \textbf{Forest} & \textbf{Result} \\\hline
    11  & 10+ & Food for one, +1 per margin. \\
    10  & 9 & Nothing found. \\
    8-9 & 8 & Lost: make a navigation roll (below), or wander in the wrong direction. \\
    7   & 6-7 & Accidental foxglove: gain 3 \glspl{fatigue} due to vomiting. \\
    6   & 5 & Creature encounter -- the \gls{gm} rolls $2D6 + 6$ on the local encounter table. \\
    5   & & Snake bite: gain $1D6+4$ \glspl{fatigue}. \\
    4   & 4 & Wrong mushroom: gain 3 \glspl{fatigue} at the end of the interval. \\
        & 3 & Snake bite: gain $1D6+2$ \glspl{fatigue}. \\
    < 4 & < 3 & Slowburn ivy: gain 2 \glspl{fatigue} each interval until you find a cure (Intelligence + Medicine, \tn[8]). \\
  \end{nametable}
}
