\makeindex[name=spells,title={Spell Summaries},columns=2]

\newcommand\bigSkillsTable{
  \begin{figure*}[t!]
  \small
  \begin{nametable}[L||L|L|L|L|L|L]{Skill Table}
  \label{skillChart}

  & \textbf{Strength} & \textbf{Dexterity} & \textbf{Speed} & \textbf{Intelligence} & \textbf{Wits} & \textbf{Charisma} \\\hline\hline
  \textbf{Academics} & Orating to a massive crowd & Forgery & Courier Runs & Recalling facts & Resisting an enchantment spell & Storytelling \\\hline
  \textbf{Athletics} & Lifting heavy loads & Climbing & Sprinting & Finding the easiest route to climb & Identifying optimal climbing conditions & Stage acrobatics \\\hline
  \textbf{Deceit} & Intimidation & Feigning an injury & Spreading a rumour across an entire town & Crafting a plausible lie & Making a quick excuse & Implausible lies \\\hline
  \textbf{Stealth} & Hiding in a hay bail & Moving quietly & Escaping into a crowd & Identifying the best hiding spot & Quickly hiding & Slipping into a party uninvited \\\hline
  \textbf{Survival} & Wrestling a~boar & Moving through dense undergrowth & Fleeing a stampede & Planning a new, hidden trail & Foraging for a quick meal & Selling information about the woods \\\hline
  \textbf{Vigilance} & Keeping watch all night & Feeling for an exit in the dark & Searching a full forest for a particular tree & Investigating a crime scene & Spotting an illusion spell & Finding the best con target at a banquet \\

  \end{nametable}
  \end{figure*}
}

\newcommand\craftingReqList{
  \begin{nametable}[l|X]{Craft Requirements}

    Academics & Books, scrolls, and bookcases. \\

    Caving & Rope, carts, pulleys, and alcohol lanterns. \\
    
    Combat & Making swords, armour, scabbards, maces, and polearms. \\

    Cultivation & Fences, rope, houses, blankets, practical clothing and saddles. \\

    Empathy & Jewellery, fancy clothing, and paints. \\

    Deceit & Disguises. \\

    Medicine & Bandages, casts, and fake eyes. \\

    Performance & Instruments, auditoriums, and theatres. \\

    Projectiles & Bows, crossbows, arrows, quivers, and bolts. \\

    Seafaring & Boats, sails, rope, and anchors. \\

  \end{nametable}
}

\newcommand{\coveringchart}{
  \begin{boxtable}[lLL]

    \textbf{Roll} & \textbf{Result} & \textbf{Damage} \\\hline

    $\leq$ \gls{tn} - your Covering & Opponent scores \gls{vitalShot}! & You take full Damage \\

    < \gls{tn} but > \gls{tn} - your Covering & Opponent hits you & Your \gls{dr} reduces Damage \\

    > \gls{tn} but < \gls{tn} + opponent's Covering & You hit the opponent & Opponent's \gls{dr} reduces Damage \\

    $\geq$ \gls{tn} + opponent's Covering & You score \gls{vitalShot}! & Opponent takes full Damage \\

  \end{boxtable}
}

\newcommand\armourExample[1][t]{
  \boxPair[#1]{
    \begin{exampletext}
    \Pgls{sunGuard} stands with a flail, and a cocky attitude.
    His basic Attack Score is 10, so rolling a 10 means a tie.
    His full plate armour has \pgls{covering} of 5 and \gls{dr} 5.

    To inflict \pgls{vitalShot}, \pgls{pc} needs to roll at \mbox{$\tn[10] + 5 = 15$}.
    Rolling 9 or less means the \gls{pc} is hit, unarmoured.

    \begin{boxtable}[cLc]
      \textbf{Roll} & \textbf{Result} & \textbf{Margin} \\
      \hline
         $\leq 9$ & \gls{pc} is hit, taking full Damage! & -1! \\
        10 & \emph{Draw} (\gls{dr} applies) & 0 \\
        11 & \Glsentrytext{npc} is hit, but \gls{dr} applies & \textbf{1} \\
        12 & \Glsentrytext{npc} is hit, but \gls{dr} applies & \textbf{2} \\
        13 & \Glsentrytext{npc} is hit, but \gls{dr} applies & \textbf{3} \\
        14 & \Glsentrytext{npc} is hit, but \gls{dr} applies & \textbf{4} \\
        $\geq 15$ & \emph{\Gls{vitalShot}!} -- full Damage to \gls{npc} & \textit{5!} \\
    \end{boxtable}

    You might think of each potential number you can roll as a location on the body, with armour adding \gls{covering} to certain numbers.
    In this case, the \gls{pc} rolls a 15, so he hits for 6 Damage, and the knight loses 6~\glspl{hp}.
    \end{exampletext}
  }{%
    \begin{tikzpicture}
        \node[anchor=south west,inner sep=0] (image) at (0,0) {\pic{Roch_Hercka/vitals_shot}};
        \begin{scope}[
            x={(image.south east)},
            y={(image.north west)}
        ]
            \foreach \mNum/\mX/\mY in {%
              {\huge$\leq 9$!}/22/25,
              10/54/20,
              11/30/63,
              12/37/45,
              13/52/63,
              14/31/70,
              {\huge 15!}/42/38,
            }{
              \mapLegend{\outline{\mNum}}{\mX}{\mY}{\large}
            }
        \end{scope}
    \end{tikzpicture}
  }
}

\newcommand\larcenyChart{
  \begin{nametable}[cccL]{Larceny Roll}
    \textbf{\Gls{village}} & \textbf{Town} & \textbf{City} & \textbf{Result} \\
    \hline
     17 & 15 & 14 & $2D6 \times 20$ \gls{cp} from a noble's servant. \\
     16 & 14 & 13 & $2D6 \times 15$ \gls{cp} from a traveller. \\
     15 & 13 & 12 & $2D6 \times 10$ \gls{cp} from a trader. \\
     14 & 12 & 11 & $2D6 \times 5$ \gls{cp} from an old lady. \\
     13 & 11 & 10 & No good targets found \\
     12 & 10 & 9 & Caught red handed! -- roll a `snatch and run'. \\
     11 & 9 & 8 & Caught red handed and surrounded! \\
  \end{nametable}
}

\newcommand\calcFallingDamage[2]{%
  \setcounter{Strength}{#1}%
  \setcounter{damage}{#2}%
  \roundUp{damage}
  \setcounter{enc}{#2}%
  \addtocounter{damage}{\value{Strength}}%
  \addtocounter{damage}{-4}%
  \setcounter{tn}{7}%
  \addtocounter{tn}{#2}%
  \setcounter{track}{#2}%
  \multiply\value{track} by 2%
  \addtocounter{track}{7}%
  \arabic{Strength} & \arabic{enc}~\ifnum\value{enc}>1\glspl{step}\else\gls{step}\fi & \dmg{damage} & \arabic{tn} / \arabic{track} \\
}

\newcommand\retreatCommentary[1][b]{
  \playCommentary[#1]{
    The \glspl{sunGuard} have caught the \glspl{pc} red-handed, so the \glspl{pc} flee.

    \begin{description}
      \item[\Glsentrytext{gm}:]
      Gimme a roll of \roll{Speed}{Athletics} for the group.
      It's \glsentrylong{tn}~8 to run, but you need 11 to get away clear.
      \item[Player 1:]
      Got it!
      \twoDice{7}
      I've got a +4~Bonus, so the total is~11.
      \item[\Glsentrytext{gm}:]
      She's out the door, faster than the wind!
      \item[Player 1:]
      Can I stay behind with the rest?
      \item[\Glsentrytext{gm}:]
      Sure -- let's see where she ends up later.
      You can always run slower.
      \item[Player 2:]
      \composeHumanName\ has 9 in total, so does he get away?
      \item[\Glsentrytext{gm}:]
      No, but you can lead the \glspl{sunGuard} where you want to run, and change one of the \glspl{trait}; you're using \roll{Speed}{Athletics}, so change one of those two.
      Can I use Empathy?
      \item[\Glsentrytext{gm}:]
      How does empathy help you run away?
      \item[Player 2:]
      We were at the market recently.
      I'll run into the marketplace, and pick the stalls I think will let me go through, or run past animals that get spooked, so they slow down anyone following me.
      \item[\Glsentrytext{gm}:]
      That makes sense.
      Next \gls{round}, roll \roll{Speed}{Empathy}.
      For now, take \pgls{ep}.
      \item[Player 3:]
      I have an 8 in total, so nothing happens, right?
      \item[\Glsentrytext{gm}:]
      Right -- \composeHumanName\ flees down the street, \pgls{sunGuard} breaks from the group to follow, and you take \pgls{ep}.
      \item[Player 4:]
      My total is 5.
      This is not a fast gnome.
      \item[\Glsentrytext{gm}:]
      \Pgls{sunGuard} breaks from the group to grab you.
      He's grabbing you, and the \glsentrylong{tn} is~11.
      What do you do?
      \item[Player 1:]
      I jump in, and move to guard her.
      You said just one \gls{sunGuard}, right?
    \end{description}
  }{
    So the fastest \gls{pc} decided to stay and help the slowest member fight, while the rest of the \glspl{sunGuard} chase the others.

    Once combat resolves (probably a fast fight), the other two players will need to roll again.
    Changing \glspl{trait} helps represent how people flee, and prompts players to use the environment to their advantage; but it also resolves chase-scenes faster, as people will pick \pgls{attribute} or \gls{skill} which gives them a decisive advantage.
  }
}

\newcommand\throwGoblinCommentary[1][t]{
  \playCommentary[#1]{
    \begin{description}
      \item[Player 1:]
      I grab the goblin, then throw it at the far goblin with the javelin.
      Is there a rule for throwing goblins?
      \item[Player 2:]
      I don't think there's rules for throwing goblins.
      \item[\Gls{gm}:]
      Well there's a grab rule.
      Grab him with \roll{Dexterity}{Brawl} at \glsfmtlong{tn}~9.
      \item[Player 1:]
      \twoDice{7}
      Got him!
      I grab his skinny neck!
      \item[\Gls{gm}:]
      And there's also a rule for \glspl{projectile}.
      But first, he has five~\glsfmtlongpl{hp}, so his \gls{weight} is 15.
      \item[Player 1:]
      So\ldots minus two penalty?
      \item[\Gls{gm}:]
      Right, and another minus two penalty for using \pgls{impromptuThrownWeapons}, since goblins are not designed for throwing.
      \item[Player 2:]
      Do the rules actually say that goblins are `impromptu'?
      \item[\Gls{gm}:]
      Goblins are nothing if not impromptu.
      Ask anyone.

      Roll \roll{Dexterity}{Projectiles} at \glsfmttext{tn} 8.
      \item[Player 1:]
      \twoDice{10}
      That hits, and the Damage is
      \dicef{4}\ldots
      what's the Damage Bonus on a Goblin?
      \item[\Gls{gm}:]
      I think `bugger all', seems generous, but that's enough to kill the target anyway.
      The skinny, white mass of bone and teeth is flung forward, skull-first into the goblin with the javelin.
      Both lie motionless.

      Now the goblin had \pgls{weight} of fifteen, so spend 15~\glsfmtplural{ap} for throwing him.

      Two more goblins aim javelins at you.
      \item[Player 1:]
      Fifteen!?
      I'm at negative fourteen \glsfmtplural{ap}.
    \end{description}
  }{
    No rules can cover every situation, but breaking \pgls{action} down into little bits often works out well enough.
  }

}
