\chapter{Established Ways}
\label{coreRules}

\settoggle{genExamples}{true}

\section{Resolution}
\label{basicaction}

\begin{multicols}{2}

\newcommand{\TNChart}{

  \noindent
  \begin{boxtable}

    \textbf{\glsentrytext{tn}} & \textbf{Task} \\\hline

    2 & Automatic \\

    4 & Trivial \\

    6 & Easy \\

    8 & Serious \\

    10 & Professional \\

    12 & Specialist \\

    14 & Extreme \\

    16 & Epic \\

    18 & Legendary \\

    20 & Implausible \\

  \end{boxtable}
}

\subsection{Basic Actions}
When \glspl{pc} attempt something dangerous and difficult, the \gls{gm} states the \gls{tn}, and the players try to beat it by rolling $2D6$ plus any bonuses.
The more difficult the action, the higher the \gls{tn}.

Players add one of their character's Attributes and Skills to the roll, and sometimes a bonus for equipment.

\begin{multicols}{2}
\begin{itemize}

  \item
  If the player rolls above the \gls{tn}, they succeed!
  \item
  If they roll below, they fail, and the danger occurs.
  \item
  If they roll equal to the \gls{tn}, the \gls{gm} can give them the choice to succeed, as long as they accept the danger.

\end{itemize}

\columnbreak

\noindent
\TNChart

\end{multicols}

\begin{exampletext}
  Hugi listens carefully at the keyhole, trying to figure out what the elves on the other side are plotting.
  The \gls{tn} is 10, and he matches it exactly.

  ``One of the elves sounds like he's wandering closer to the door, but you think you almost heard a familiar word'', the \gls{gm} says.

  ``I'll stay and listen'', Hugi's player says.

  ``They use your name, though you can't understand the elvish -- just your name -- then one opens the door, saying `well here he sits, or stands, I can never tell with dwarves'.
  The other elves stand up quickly.''

\end{exampletext}

All actions are assumed to have a \gls{tn} of 7 unless your \gls{gm} states otherwise.
Don't ask -- just roll!%
\footnote{\Glspl{gm} never forget to state the \gls{tn}, so if you're told to roll, but don't hear any \gls{tn}, then the \gls{tn} \emph{must} be 7.
This is legally binding.}

These standard rules should cover any situation, with the right interpretation.
However, for suggested interpretations with more detail, see \nameref{skill_uses}, \autopageref{skill_uses}.

\subsection{One Roll Only}

Players only make one roll per action.
If the player wants to attempt to re-try an action, the result remains the same unless circumstances change.

When many characters are trying to do the same task, a single roll is made, and they all consult the same results.

\begin{exampletext}
\begin{itemize}
  \item
  Everyone wants to kick in the door, the \gls{tn} is 10, and the roll uses Strength + Crafts.
    \begin{itemize}
    \item
    Pikerudd's player rolls the dice. Her Strength + Crafts Bonus is 0, so she fails.
    \item
    Snowblight's Strength + Crafts total is +2, so his total is 9, and he also fails.
    \item
    Chatrik's Strength + Crafts total is +4, so her total is 11 -- she succeeds.
    \end{itemize}
  \item
  These look like the famous Catacombs from the national anthem, but how do the words go?
  Did the hero take two lefts and a right, or two rights and a left?
    \begin{itemize}
    \item
    Water is filling the area, so time will be lost if they get muddled.
    \item
    Everyone rolls, and the highest result is Dzo's -- but he only equals the \gls{tn}.
    \item
    Dzo's player decides to guess left, rather let his companions think about the song for another moment.
    They fail, but avoid wasting time.
    \end{itemize}
\end{itemize}

\end{exampletext}

If the troupe are all attempting the same action, then they only make one roll, while adding different attributes to obtain their individual result.

\subsection{\glsentrytext{restingaction}}\label{restingactions}

Difficult, but safe actions allow players to repeat the same task until they get it right or give up.
In these cases, the player sets one die to a `6', and rolls only the other die.

\begin{exampletext}
\begin{itemize}
  \item
  The group want to sneak into a noble's house, and have plenty of time to plan the heist.
  The \gls{gm} says they can wander past without suspicion, but it will take a week to plan and gather all the information they need, the Bonuses are Intelligence plus Stealth, and the \gls{tn} is 12.
    \begin{itemize}
    \item
    The players accept, and roll a single die, achieving a `4'.
    \item
    With the other die automatically on `6', their roll is `10', and the total is `13' -- a narrow success.
    \end{itemize}
  \item
  When the group find a powerful, but mysterious artefact.
  \begin{itemize}
    \item
    Without any danger reading, the \gls{gm} lets them study it for a month as a \gls{restingaction}.
  \end{itemize}
  \item
    The troupe really need a favour from a local warden, but he has no time to meet with commoners.
  \begin{itemize}
    \item
    The \gls{gm} disallows this as a resting action.
    \item
    After the first failure, the noble tells his servants not to pay them any more attention -- they can no longer succeed via the official channels.
  \end{itemize}

\end{itemize}
\end{exampletext}

\subsection{Resisted Actions}
\index{Resisted Actions}
\label{resistedactions}

When \pgls{pc} and \pgls{npc} act in opposition to each other, the player add their characters Attribute + Skill bonuses as usual, and rolls against \tn[7] plus the opponent's Trait-pair.%
\footnote{In general, only players roll.}

\begin{itemize}
  \item
    A villain wants to sneak up on a \gls{pc} while they go shopping for armour, and corner them alone.
    Her \roll{Intelligence}{Stealth} gives a +3 Bonus, so the \gls{gm} asks for a \roll{Wits}{Vigilance} roll at \tn[10].
  \item
    The troupe need to convince the locals not to trust a particular psychotic, lying, bastard.
    Despite him not being present, the \gls{gm} notes that he his \roll{Charisma}{Empathy} grants a +5 Bonus, so he states the \gls{tn} is 12.
  \item
    A witch begins a curse against a member of the \gls{guard}, so he decies to stab her.
    If the witches spell works, he will fumble his attack, and lose some \glspl{fp}, so her \roll{Charisma}{Fate} adds to the roll, making a \gls{tn} of 9.
    The guard will have to use his \roll{Dexterity}{Combat} to resist the curse, but if he wins, he will inflict Damage (covered in \autoref{combat}).
\end{itemize}

\subsection{Teamwork}
\label{teamwork}
\index{Teamwork}
\index{Group actions}

Some tasks lend themselves to working with others. Others can be difficult or impossible to do with companions. Some tasks, such as fleeing or sneaking, do not benefit at all from having a load of friends right behind you.

When acting as a group provides no benefit, one player rolls the dice and the same result counts for everyone.  If that player rolls a 9, then everyone's score is 9 and they add their own bonuses and penalties.

If, on the other hand, working together can benefit a situation, one character takes the lead, and up to three other characters can add up to half their bonus (rounded up).
Two companions with a +3 bonus would add a total of a +2 bonus.


\begin{exampletext}
  Example Team Actions include:

  \begin{itemize}

  \item Getting a broken cart down a hill without damaging it.
  \item Tracking down a local thief in a large city.
  \item Spotting danger in the wild.

  \end{itemize}
\end{exampletext}

\subsubsection{Stacking}
\index{Stacking}
\label{stacking}

In general, whenever you want to see how something stacks, add the second lot as half its usual value.
If two people are pushing with Strength +2, they count as having a total Strength of +3.
If others want to join, add any third items as worth a quarter, then an eighth, and so on.

\begin{exampletext}

Convincing the townsfolk that they need to rebel against the baron, and could easily succeed, the troupe work together on a Charisma + Tactics roll.

\end{exampletext}

\noindent%
\begin{footnotesize}%
  \begin{boxtable}[Y |cccc]
                      & Pikerudd & Snowblight & Chatrik & Drake \\
  \hline
  Charisma + Tactics: &  +3     & $+\frac{3}{2}$      & $+\frac{2}{4}$      & $+\frac{1}{8}$    \\
  Roll Bonus:         &  +3     &     +2               &       +1             &  0 \\
  \hline
  Running Total:      &  +3     &     +5              &  +6     & +6    \\
  \hline
    Grand Total: & & & & \textbf{6} \\
  \end{boxtable}
\end{footnotesize}

\subsection{Margins}
\index{Margins}
\index{Failure Margin}
\label{margin}

If you ever need detail on how well an action went, look at how many points above the \gls{tn} the dice show.
With a \gls{tn} of 12, rolling 14 means a margin of 2.

The \gls{gm} might use a Margin for some variable, for example a bard attempting to charm a crowd into giving him money might gain $2D6$ copper pieces plus the Margin, so if the Margin is 3 then he would get $2D6+3$ copper pieces.
Margins might also be used to gain bonuses on later rolls.
Someone attempting to impress a noble court might roll Charisma with the Tactics Skill; the bigger the Margin the more troops they will be trusted with.

\subsection{What the Dice Mean}

You might think of the dice as representing random chance in the environment. Just how irritated is that person you're trying to question, and how creative is that craftsman feeling today? Dice are never re-rolled for different results on the same action because once the dice have told you what the situation is, the situation stays put.

Such a do-over still suggests initial failure; it just means that the character is trying over and over again until a better result is obtained.
Actions cannot be attempted multiple times with rerolls unless the situation has changed notably.

\end{multicols}

\section{Weight \& Encumbrance}
\index{Weight}
\index{Encumbrance}
\label{weight}

\begin{multicols}{2}

\noindent
We measure weight in broad terms, just in case characters need to lift a wounded companion (or corpse), or try to put thirty shortswords in the backpack.
Players should take care to note how much their character carries any time they pick up another item.

\input{config/rules/weight.tex}

The increased weight from carrying an object in one hand generally makes dual-wielding axes about as practical as sprinting with a human sitting on your shoulders.

\subsubsection{Encumbrance}
\index{Encumbrance}

Encumbrance can become normal, if character want to maintain a good store of rations,%
\footnote{A day's rations has a \gls{weight} of 1.}
and character may simply decide to put up with the penalties.
The only real danger lies in unexpected combat.
In these situations, characters can spend \pgls{ap} to remove their backpack with a \roll{Dexterity}{Wyldcrafting} check.
The basic \gls{tn} is 10, but it depends on the backpack.
\index{Backpacks}

\pic{Roch_Hercka/dwarf_encumbrance}

\end{multicols}

\pagebreak
\section{Time \& Space}

\begin{multicols}{2}

\subsection{Time}
\label{time}

\subsubsection{Setup}
Each session has a small setup, and then the day starts to spin.

\input{config/rules/start.tex}

\subsubsection{Rounds}

When everyone wants to talk and act at the same time, time switches to \glspl{round} (usually during combat).
\Pgls{round} itself can then be further divided into \glspl{ap} (covered \vpageref{actionPoints}).
All that matters is that \pgls{round} is a period of time in which players roll dice, then another \gls{round} occurs.

\subsubsection{\Glsfmtplural{interval}}
\label{intervals}

\input{config/rules/interval.tex}

\subsubsection{\Glsentrytext{downtime}}

Players have a limited amount of \gls{downtime} to employ.
Every week in our world, three weeks pass in \gls{fenestra}.
So a weekly session leaves the players with no more than three weeks for the \glspl{pc} to rest.

\subsubsection{Healing}
\label{healing}
\input{config/rules/healing.tex}
So someone on -3~\glspl{hp} would recover 2~\glspl{hp} after a week, then reach 0~\glspl{hp} on week 2, 1 \gls{hp} on week 3, then 2, 3, 5, and so on.

When a \gls{pc} has been reduced to 1~\gls{hp}, the player should consider taking a different character -- from their \gls{characterPool} or any relevant \gls{npc} -- for the rest of the session.

\subsection{Space}
\label{space}
\index{Space}

\subsubsection{\Glsfmtplural{step}}
\index{Steps}

BIND tracks space with \glspl{step}.
A \gls{step} is just any unit of space within the battlefield.
If you are using a battlemap which has squares or hexagons marked out on it, then those tiles are the size of a step.
A step might be ten metres wide as each one covers an entire house when the battlefield is a large town, or it might be just two yards wide when moving through a detailed map of a labyrinth.

The precise distances represented do not matter, just so long as they consistently balance one character's ability to run away with another's ability to hit someone with a projectile.
To track longer distances, without resorting to unnatural language like `250 metres', a couple of systems use broad relational terms:

\begin{itemize}
  \item
  `throwing distance'
  \item
  `yelling distance'
  \item
  `on the horizon'
\end{itemize}

\subsubsection{Areas}
\index{Areas}

\Pgls{area} is just any place which looks different from another.
While traipsing through a labyrinth, each room or passage might be thought of as \pgls{area}.
When gallivanting through open plains one \gls{area} might be a copse of trees, another a lake, and then the next area, \pgls{village}.

\subsubsection{Region}
\index{Regions}

Regions encompasses a full forest, a town, or a collection of \glspl{village}.
Each region has its own set of likely encounters, such as tradesmen in the \glspl{village}, cut-throats in town, and elves in the forest.

\end{multicols}

\settoggle{genExamples}{false}
