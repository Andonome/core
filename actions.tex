\chapter{Established Ways}
\label{coreRules}

\settoggle{genExamples}{true}

\section{Resolution}
\label{basicaction}

\begin{multicols}{2}

\newcommand{\TNChart}{

  \noindent
  \begin{boxtable}[cX]

    \textbf{\glsentrytext{tn}} & \textbf{Task} \\\hline

    2 & Automatic \\

    4 & Trivial \\

    6 & Easy \\

    8 & Serious \\

    10 & Tricky \\

    12 & Professional \\

    14 & Specialist \\

    16 & Extreme \\

    18 & Legendary \\

    20 & Implausible \\

  \end{boxtable}
}

\subsection{Basic Actions}
When \glspl{pc} attempt something dangerous and difficult, the \gls{gm} states the \gls{tn}, and the players try to beat it by rolling $2D6$ plus any bonuses.
The more difficult the action, the higher the \gls{tn}.

Players add one of their character's Attributes and Skills to the roll, and sometimes a bonus for equipment.

\begin{multicols}{2}
\begin{itemize}

  \item
  If the player rolls above the \gls{tn}, they succeed!
  \item
  If they roll below, they fail, and the danger occurs.
  \item
  If they roll equal to the \gls{tn}, the \gls{gm} can give them the choice to succeed, as long as they accept the danger.

\end{itemize}

\columnbreak

\noindent
\TNChart

\end{multicols}

\begin{exampletext}
  Hugi listens carefully at the keyhole, trying to figure out what the elves on the other side are plotting.
  The \gls{tn} is 10, and he matches it exactly.

  ``One of the elves sounds like he's wandering closer to the door, but you think you almost heard a familiar word'', the \gls{gm} says.

  ``I'll stay and listen'', Hugi's player says.

  ``They use your name, though you can't understand the elvish -- just your name -- then one opens the door, saying `well here he sits, or stands, I can never tell with dwarves'.
  The other elves stand up quickly.''

\end{exampletext}

Assume all actions have \pgls{tn} of 7 unless your \gls{gm} states otherwise.
Don't ask -- just roll!%
\footnote{\Glspl{gm} never forget to state the \gls{tn}, so if you're told to roll, but don't hear any \gls{tn}, then the \gls{tn} \emph{must} be 7.
This is legally binding.}

These standard rules should cover any situation, with the right interpretation.
However, for suggested interpretations with more detail, see \nameref{skill_uses}, \autopageref{skill_uses}.

\subsection{One Roll Only}
\index{Group actions}

Players only make one \gls{natural} per action.
If a player wants to re-try an action, the result remains the same unless circumstances change.

When many characters are trying to do the same task, one player rolls, and all players consult the same results.

\begin{exampletext}
\begin{itemize}
  \item
  Everyone wants to kick in the door, the \gls{tn} is 10, and the roll uses Strength + Crafts.
    \begin{itemize}
    \item
    Pikerudd's player rolls the dice. Her Strength + Crafts Bonus is 0, so she fails.
    \item
    Snowblight's Strength + Crafts total is +2, so his total is 9, and he also fails.
    \item
    Chatrik's Strength + Crafts total is +4, so her total is 11 -- she succeeds.
    \end{itemize}
  \item
  These look like the famous Catacombs from the national anthem, but how do the words go?
  Did the hero take two lefts and a right, or two rights and a left?
    \begin{itemize}
    \item
    Water is filling the area, so time will be lost if they get muddled.
    \item
    Everyone rolls, and the highest result is Dzo's -- but he only equals the \gls{tn}.
    \item
    Dzo's player decides to guess left, rather let his companions think about the song for another moment.
    They fail, but avoid wasting time.
    \end{itemize}
\end{itemize}

\end{exampletext}

If the troupe are all attempting the same action, then they only make one roll, while adding different attributes to obtain their individual result.

\subsection{\Glsfmtplural{restingaction}}
\label{restingactions}

Difficult, but safe actions allow players to repeat the same task until they get it right or give up.
In these cases, the player sets one die to a `6', and rolls only the other die.

\begin{exampletext}
\begin{itemize}
  \item
    The troupe really need a favour from a local \gls{warden}, but he has no time to meet with \gls{guard}~\glspl{fodder}.
  \begin{itemize}
    \item
    The \gls{gm} disallows this as \pgls{restingaction}.
    \item
    After the first failure, the noble tells his servants not to pay them any more attention -- they can no longer succeed via the official channels.
  \end{itemize}
  \item
  The group want to sneak into a noble's house, and have plenty of time to plan the heist.
  The \gls{gm} says they can wander past without suspicion, but it will take a week to plan and gather all the information they need.
  The roll is \roll{Intelligence}{Stealth} at \tn[12].
    \begin{itemize}
    \item
    The players accept, and roll: \dicef{4}.
    \item
    With \dicef{6}~\dicef{4} on the table, the dice show `10', and the total is `13' -- a narrow success.
    \end{itemize}
\end{itemize}
\end{exampletext}

\subsection{Resisted Actions}
\label{resistedactions}

When \pgls{pc} and \pgls{npc} act in opposition to each other, the player adds their characters Attribute + Skill bonuses as usual, and the \gls{tn} equals \tn[7] plus the \gls{npc}'s \gls{attribute} + \gls{skill}.%
\footnote{In general, only players roll for actions.}

\begin{exampletext}
  \begin{itemize}
    \item
      A local cutthroat wants to sneak up on \pgls{pc} while they go shopping for armour, and corner them alone.
      Her \roll{Intelligence}{Stealth} gives her a +3 Bonus, so the \gls{gm} asks for a \roll{Wits}{Vigilance} roll at \tn[10].
    \item
      The troupe need to convince the locals not to trust a particular psychotic, lying, bastard.
      Despite him not being present, the \gls{gm} notes that he his \roll{Charisma}{Empathy} grants a +5 Bonus, so he states the \gls{tn} is 12.
    \item
    \Pgls{witch} begins a curse against a member of the \gls{guard}, so he decides to stab her.
    Her \roll{Charisma}{Fate} total is +2, so the \gls{tn} is ($7+2$) 9.

    The guard uses his \roll{Dexterity}{Melee} to resist the curse, so if he wins, he will inflict Damage (covered in \autoref{combat}).
    But if the \gls{witch}'s spell works, the \gls{guard} will fumble his attack, and lose some \glspl{fp}.
  \end{itemize}
\end{exampletext}

\subsection{\Glsfmtplural{bandAct}}
\label{teamwork}
\label{banding}
\index{Teamwork}

Some tasks lend themselves to working with others.
Others can be difficult or impossible to do with companions.
When the troupe want to work together to get a broken cart down a hill safely, track down a thief, or spot danger, they can benefit from \pgls{bandAct}.
But when sneaking through an area, navigating, or understanding history, Banding together will not help.

When characters can work together, one person rolls with their Bonus, then adds half the second Bonus, a quarter of the next Bonus, and so on.%
\footnote{Always round up on half, so $4 + \frac{1}{2} = 5$, but $5 + \frac{1}{4} = 5$.}

\begin{exampletext}
  Convincing the townsfolk that they need to rebel against \pgls{warden}, the troupe work together on a \roll{Charisma}{Combat} roll.
  Pikerudd has the highest Bonus, so he adds the full Bonus, then Snowblight adds half of his, and Chatrik, a quarter.
\end{exampletext}

\begin{boxtable}[L|ccc]
                          & Pikerudd & Snowblight      & Chatrik         \\
\hline                                                                   
\raggedright
\roll{Charisma}{Combat}  &  +4     &     +2           & +2              \\
Multiplier                &   1     & $\frac{1}{2}$    & $\frac{1}{4}$   \\
Value:                    &   4     & $1$              & $\frac{1}{2}$   \\
Running Total:            &   4     & $5$              & $5+\frac{1}{2}$ \\
\hline
  Grand Total: & & & \textbf{6} \\
\end{boxtable}

\subsubsection{Generalized \Glsfmtplural{bandAct}}
means all accumulated Bonuses work the same way.
If two people are pushing with Strength +2, they count as having a total Strength of +3.
You'll find a few different rules are just the \gls{bandAct} rule applied to a different area (for example, \nameref{bandingArmour}, \vpageref{bandingArmour}).

\subsection{Margins}
\index{Margins}
\index{Failure Margin}
\label{margin}

If you ever need detail on how well an action went, look at how many points above the \gls{tn} the dice show.
With \pgls{tn} of 12, rolling 14 means a margin of 2.

The \gls{gm} might use a Margin for some variable, for example a bard attempting to charm a crowd into giving him money might gain $2D6$ copper pieces plus the Margin, so if the Margin is 3 then he would get $2D6+3$ copper pieces.
Margins might also be used to gain bonuses on later rolls.
Someone attempting to impress a noble court might roll Charisma with the Combat Skill; the bigger the Margin the more troops they will be trusted with.

\subsection{What the Dice Mean}

You might think of the dice as representing random chance in the environment. Just how irritated is that person you're trying to question, and how creative is that craftsman feeling today? Dice are never re-rolled for different results on the same action because once the dice have told you what the situation is, the situation stays put.

Such a do-over still suggests initial failure; it just means that the character is trying over and over again until they obtain a better result.
Characters cannot roll dice for a different result unless the situation has changed notably, and generally not during the same day.

\end{multicols}

\section{\Glsfmttext{weight}}
\label{weight}
\index{Encumbrance}

\begin{multicols}{2}

\noindent%
Everything in the game has \pgls{weight}, and when characters carry more than they should, ever point in excess inflicts a Penalty to all actions.

\pic{Roch_Hercka/dwarf_encumbrance}

\subsubsection{On the Back}
characters can hold a total \gls{weight} equal to their current \glspl{hp} without Penalty.
After that, every spare shortsword, quiver, and cheese-wheel increases the Penalty by its own \gls{weight}.

\index{Backpacks}
\begin{exampletext}
  Skidvein has an impressive +3 Strength Bonus, and the Knack `Unstoppable', giving him 11~\glspl{hp} in total.
  He decided to wear complete leather armour (\gls{weight} 2), take a longsword (3), then stuff his backpack with three days' worth of food (3), and a small tent (1).

  His companions decided he could add their larger tent (3), medical supplies (1), and bagpipes (2).
  The \gls{weight} of 15 puts him well over his 11~\glspl{hp}, so the penalty is -4.

  However, if a fight breaks out, he could just remove the backpack by spending \pgls{ap}, and rolling \roll{Dexterity}{Survival} (\tn[10]).
\end{exampletext}

All creatures have \pgls{weight} equal to their total \glspl{hp},%
\footnote{If \glspl{pc} try to use goblins as a throwing weapon, remember that improvised weapons receive penalties because goblins are actually not made for throwing.}
so if a gnomish companion receives a serious injury, the troupe may be able to lift them up once the danger has passed, and march them back to civilization from beyond the \gls{edge}.
If a large human receives a serious injury, their companions may have a serious question on their hands.

\subsubsection{In the Arms}
characters can carry a total \gls{weight} equal to half their \glspl{hp} in their arms.
Someone with 8~\glspl{hp} who picks up a bronze statue, with \gls{weight}~12 would receive a -8~Penalty to all actions.
Mounting it on their back reduces the Penalty to~-4.

\subsubsection{To Hand}
means using only one hand, and puts the limit to a quarter of their \glspl{hp}.
If \pgls{pc} with 6~\glspl{hp} lifts a poleaxe with \gls{weight}~4, they would receive a -2 Penalty to attack.
If the character actually hits, the Damage works as usual, so sometimes people find that picking up a weapon twice their height feels like the best option they have.

\subsection{\Glsfmtlongpl{ep}}
\label{ep}

\Glsentryfullpl{ep} are \pgls{weight} you cannot drop.
They build up slowly as characters exert themselves, threatening quietly, then suddenly present an unfair choice --- should the \gls{pc} drop the last of their food (risking more \glspl{ep} later), stay and rest (in a forest full of curious \glspl{monster}) or endure (and listen to their croaking lungs degrade)?

Everything from combat, to swimming through freezing water, or climbing a cliff, should add \pgls{ep}, but only once the character has stopped, and the adrenaline has worn off.
While mid-combat, characters should only gain \glspl{ep} from spells, not from the fight itself.

\Glsentrylongpl{fp} cannot mitigate \glspl{ep}.
Characters with enough luck to avoid arrows and dragon-fire can still collapse after a long~run.

\subsubsection{Special Categories}
of \glspl{ep} will not disappear with \pgls{interval}'s rest.
Starvation and sleep deprivation inflict \glspl{ep} which only return with food and sleep.
Likewise, poison and some wicked diseases accrue \glspl{ep} until healed with \pgls{ingredient} from the correct \gls{sphere}.

\index{Coins!\Glsfmtlongpl{ep}}
The character sheet holds a special place for a stack of coins to represent \glspl{ep}, below the space for \glspl{hp} coins.
Using different coins to represent different kinds of \gls{weight} and \glspl{ep} helps to track all the changes -- you can feel the weight pinning the character sheet down, as the character's burden increases.

\end{multicols}

\pagebreak[0]
\section{Time \& Space}

\begin{multicols}{2}

\subsection{Time}
\label{time}

\subsubsection{\Glsfmtplural{round}}
\glsentrydesc{round}.


\subsubsection{\Glsfmtplural{interval}}
\glsentrydesc{interval}.

A central pool of coins makes giving and tracking \glspl{fp} and \glspl{mp} faster than writing and unwriting over the same scuffed square on the character sheet.
\index{Coins!\Glsfmtlongpl{fp}}
Players have a space on their character sheets to track \glspl{fp}, just below the \glspl{hp} area.

\subsubsection{\Glsentrytext{downtime}}
\glsentrydesc{downtime}.
So a weekly session leaves the players with no more than three weeks for the \glspl{pc} to rest, train, and plan.
Find suggested \gls{downtime} actions \vpageref{downtimeActions}.

\label{healing}
When \pgls{pc} has been reduced to 1~\gls{hp}, the player should consider taking a different character for the rest of the session, from their \gls{characterPool} if possible.

The wounded \gls{pc} can go to rest in the nearest \gls{healersGuild}.

\subsection{Space}
\label{space}
\index{Space}

\subsubsection{\Glsfmtplural{step}}
\index{Steps}
are yards if you like yards, or metres if you like metres.
Or just \glspl{step}.

\subsubsection{Longer Distances}
use these standard approximations, to save the \gls{gm} from having to say `250 metres North'.
In order, they are:

\begin{multicols}{2}
  \begin{enumerate}
    \item
    \toggletrue{Distant}
    \setcounter{spellCost}{2}
    \setRange\stepcounter{spellCost}
    `\spellRange'
    \item
    \setRange\stepcounter{spellCost}
    `\spellRange'
    \item
    \setRange\stepcounter{spellCost}
    `\spellRange'
    \item
    \setRange\stepcounter{spellCost}
    `\spellRange'
    \item
    \setRange\stepcounter{spellCost}
    `\spellRange'
  \end{enumerate}
\end{multicols}

%\subsubsection{\Glsfmtplural{area}}
%\glsentrydesc{area}.
%
%\subsubsection{\Glsfmtplural{region}}
%\glsentrydesc{region}.

\end{multicols}

\settoggle{genExamples}{false}
