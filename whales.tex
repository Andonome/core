\section{The Bodies of Giants}

\begin{multicols}{2}

\noindent
In every RPG you kill monsters.
It's a staple of the genre.
But why would you kill monsters?
An actual reason, not just the video game logic of ``kill enemy = number go up''.
What would be the point?

Because monsters are fuel, and always have been fuel.
Humanity runs on monsters.
The gears of industrial revolution were lubricated by whale-oil.

You can call me Ishmail, because I'm gonna tell you about whales.

Whaling has always existed, for as long as humans could hold a spear, but it was the boom in whaling in the North Sea that partially facilitated the industrial revolution.
A single whale carcass provides a lot of resources, all ripe to be utilized by numerous scientific discoveries of the time.

In the early days, whaling was a brutal, dirty job, done by the dregs of society (remember ``Moby Dick''?).
You had a sailing ship, a row boat, a harpoon and your two hands.
And once a whale was harpooned, you had to tug it along to your ship, and quickly extract it before it sinks.%
\footnote{Whales, especially big ones, sink pretty quickly once dead.
It's actually kind of a big deal for the ocean floor and its dwellers, and it's called `whale fall'.}
Peel its skin like an orange, carve out blubber to be rendered\ldots
For sperm whales,%
\footnote{Sperm whales -- like \textit{Moby Dick}, or \textit{Pearl} from \textit{Sponge Bob}.
If you snicker at the name, some might protest that its name doesn't actually mean sperm.
Which is true, its named after the thick waxy liquid sloshing inside its angular head -- spermaceti.
Which in turn means `whale sperm'.
Funny how that works.
It is in fact not sperm, it's probably something related to buoyancy and diving, but we cannot confirm because sperm whales are smart and also hate us now.}
crack its skull and extract the liquid within (for sperm-oil), for baleen whales,%
\footnote{Whales are broadly divided into `toothed whales' and `filter-feeding baleen whales' (like the whale that swallowed Pinocchio and probably Jonah of Biblical fame).
Baleen comes from Latin \textit{balaena}, related to Greek \textit{phalaina}, all of which mean `whale', which makes `baleen whale' the `whale whale'.
The Platonic ideal of whale.}
snap off the curtain-like teeth.

Imagine every time you wanted to fill up your gas tank, someone else had to fist-fight a dragon for it.

Whale by-products were truly in everything.
You lit your house with whale-oil gas lamps and candles, put it on bread (as margarine), washed yourself with whale-oil soap, lubricated your watch, loom, sewing machine, wore ambergris%
\footnote{Ambergris -- fossilized bowel secretions of a whale, probably a reaction to irritation (again, we can't study this because whales hate us now).
Highly valued ingredient of perfumes and sometimes food.
Valuable enough that some guy lied about the Bahamas being the mother lode of ambergris and got it colonized by the British.
Technically not a by-product of whaling, but nonetheless illegal in order to discourage the usage of whale by-products.
Still illegally used worldwide nonetheless.}
perfume, perhaps ate it as a delicacy, wore whalebone%
\footnote{Whalebone -- in fact not a bone, but the sort-of-teeth of a baleen whale.
Closer to hair or nails in structure than to bone.}
in your corset or bustle, and carried it as an umbrella or a basket.
And you most certainly used things made with newfangled industrial processes, all of which used sperm-oil as fuel and lubricant.

Whale persisted as a resource until the 70s, long after the discovery of petroleum.
At this point, whaling was a much easier process, with steam and diesel-powered ships, harpoon cannons tipped with explosives, and air-pumps to keep the carcass afloat.
No longer a gruelling fight of man against monster, now merely an impersonal industrial mining process.

But \gls{fenestra} has no industry, whaling or otherwise.
So it's on you to find a weapon, enter the forest, and slay a basilisk without damaging the hide.

\end{multicols}
