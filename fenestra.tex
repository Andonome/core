\chapter{Monsters \& Malthus}

\begin{multicols}{2}

\subsection{Walking with the \glsentrytext{guard}}

\begin{exampletext}

  Welcome to the war for civilization.
  We'll be going first, to the \gls{shatteredcastle}, then on to Arthur's Wing, then out to find out what destroyed Greenwall.
  Don't look so worried -- there are eleven of us, and the barracks on the other side have crossbows.

\end{exampletext}

\subsubsection{\Gls{shatteredcastle}}
\index{The Shattered Castle}

\begin{exampletext}

  This isn't a detour.
  \Gls{shatteredcastle} just allows the best route to Mt Arthur.
  You go in one end of it, and out you come in another land.
  The traders will gladly pay in blood if they have to -- \gls{shatteredcastle} goes everywhere.

  This clever setup started by making an alchemical portal from the Rex's castle in the Shale to a hidden location in Whiteplains.
  Then his other castle in Eastlake got a portal to the Whiteplains hub, and another, until all six castles were joined into one.
  Each one sits in a different region -- the Pebble Islands, Quennome, Mount Arthur -- but the alchemical portals mean they are also one castle.
  His castle surrounds Fenestra.
  We've already passed through the Whiteplains central heart in one of those corridors.
  Do you feel that cold?
  Nobody knows exactly where his father hid the castle's heart, but we have -- without a doubt -- entered it; meaning we are now some hundreds of miles from where we once were.

  Follow the signs towards the portal leading to Mount Arthur.
  It will appear as nothing more than a normal corridor to us, but when you feel the warmth, you will know we have reached Mount Arthur.

  We could be wandering through here a while, so do not become lost.
  Other portals lead to other lands, such as Quennome, or the Pebble Islands.
  I have heard that some lead to stranger lands in the world, where the trees grow upside down and gold grows out of the walls.

  \Gls{shatteredcastle} surrounds and joins Fenestra.
  It ensures nobody challenges \gls{king}, as he can gather his army -- meaning us -- to any realm instantly.
  I imagine nobody will challenge \gls{king} any time soon, given that he has outlawed the creation of alchemical portals, or any other portals for that matter.
  So remember, if you ever witness illegal magics occurring, you must report it straight to me, or the nearest soldier.

  That smell of fish in the air\ldots probably leads off to the Pebbles islands.
  We should turn left ahead to exit to Mount Arthur.

\end{exampletext}

\subsubsection{The Towns}

\begin{exampletext}

  If you grew up in a quiet village on the edge of a small town, I suppose you've never seen a big city like Arthur's Wing before.
  No monsters live in here, so everyone can rest easy, aside from the cutthroats and thieves, who of course have to worry about the likes of me dragging them into our merry little crew and our glorious mission.
  Look at that pathetic beggar over there, asking for food.
  He can clearly walk, but refuses to sign up with us and fight for the crown.
  Remember that even if you get mauled by some creature in the forest, that still leaves an opening for your companions to get a hit against the beast that killed you.
  Everyone dies a hero in the \gls{guard}.

  Most places just have a town master or city master, but being such a big place, Arthur's Wing has one of the seven \textit{area} masters.
  If they ever give you an order, remember that they're in charge, unless \emph{I} give you different orders, because I am always in charge.
  You can feel free to entirely ignore the demands from the guild masters, even if they happen to be priests.

  The various guild-temples obtained their monopolies long before the current Rex.
  Laiqu\"e has always had a monopoly on farmers, and just about anything that grows, and the priests of V\'er\"e have always taken care of the court houses.

  Of course the towns, and even cities, have the \gls{guard} patrolling them with good reason.
  Besides the cutthroats -- whom I regularly drag into the Guard -- cities can be attacked.
  Sometimes a great, stinking, basilisk manages to get through the outer rings of villages.
  Sometimes the nura overrun the landscape.
  I became a captain fifteen years ago during a massive siege of ogres in another city.
  They were still intelligent enough to construct basic ladders, and started coming over the walls.
  We pulled in everyone we could from the villages outside, and eventually they starved, and we emerged to pick them off when they were weak.

  Let us get some rest.
  We must leave for the forest at dawn.

\end{exampletext}

\subsubsection{Villages}

\begin{exampletext}

  We will take the higher road so everyone can get a better look at the lands we came to protect.
  Only a minority of villages look like the nice place you grew up in -- no walls, or patrols.
  Look over at that island -- people live there for safety, then go over to work the fields during the day.
  Creatures come out of the forest to take their animals, and then they shout and shoot crossbows.
  I like the idea of living on the safety of a little island, but I do not think they can make it work -- just look at how few sheep they have left.

  In the distance, you can see a more practical -- and standard -- solution: walls.
  Walls cannot provide much protection on their own, but when every able-bodied man in the village owns a bow, it adds up to a lot.
  Whether a bear or a woodspy crawls up to that wall, a dozen men will ready within minutes to fend it off.

  Look up ahead -- a trader's caravan.
  Shame it travels the other way.
  Remember -- always travel with as many people as possible, the beasts could appear anywhere, especially the woodspies.

  I do not feel surprised you have never seen a woodspy -- they can change their skin to look like a tree, or grass.
  You will find them by the cries of nearby animals, or during the dawn's light where they never seem to get their colouration on point.
  The chitincrawlers instinctively stay in the shadows to hide their sable bodies.
  Mostly they come out at night.

  Now if you ever smell a basilisk, just run for the nearest walled village, and ready the men.
  You will know how they smell by the time one is half a mile away.
  Once the archers have readied, you shoot until the basilisk runs.
  Don't expect to kill one -- we can only hope to make the forest more hospitable than the sheep.
  On the bright side, if you get eaten, a basilisk typically wanders away, sated by the meal.
  Like I said, everyone dies a hero in the \gls{guard}.

  Most roads stay between villages, making them a little safer, but this road ahead stretches to the last village.
  We'll be there soon, and we can start the report on what happened to it.

  When something destroys a village, you can usually get the report from those that flee.
  Mostly it happens slowly.
  A few too many young men die in something's jaws, and everyone has to relocate to nearby family, or the city, where we can give them a sword and put them to work taking back their lost holdout on the edge of civilization.
  Sometimes monsters take too many animals -- a horde of chitincrawlers or a couple of basilisks come, and soon enough nobody has enough food for a cold season, and they relocate.

  We need to investigate this place personally, as there were no survivors -- not one returned.
  The last group found the village empty, and tracks leading away.
  It could be a small army wandering away, or perhaps the inhabitants themselves turned undead, or were enchanted by some black magic.

  There! You can see up ahead the archers' stations, fifteen in total.
  The report missed that detail out.
  This means that whatever attacked this place dealt with many archers who had a clear shot to whatever was below.
  Look here at the ground -- you can already see dozens of marks in the earth where the arrows hit, but no arrows remain in the ground.
  Someone pulled them out

  That makes another win for the forest, and another loss for civilization.
  You can already see the grass growing inside.
  The forest now considers this space her own, so we should sleep outside so we can see farther -- any number of creatures could have made their nests besides the dead hearths in here.
  Even creatures not here now might return at night for a secluded area to sleep.

\end{exampletext}

\subsubsection{The Primordial Forest}

\begin{exampletext}

  The forest wants to eat you.
  If you do not want to be eaten, stay alert, but do not get too curious.
  Every disgusting, and hungry creature you have ever heard of lives out here, and then some more.
  Most of the world sits in darkness, just like this.
  Most of the world lacks roads, beer, beds, and everything that makes life worth living.
  For this reason, we exist, to push back the darkness, and make way for more civilization.

  On other trips, the \gls{guard} enter first, and clear a path for people to cut timber down, make walls, lay rocks for a road, and then begin houses inside.
  In this way we civilize our world, but today we can only do damage control on what we have lost.

  On a more positive note, we will not have to worry about rations out here.
  Fruit and vegetables, mushrooms and roots, and all manner of things grow out here.
  So long as you see no snow, you can usually pick up what you need as you go.

  Unfortunately, this propensity for verdant growth makes the forests a standard place for lawless people to hide.
  Black Magic sorcerers, bandits, heretics, and worse all live outside the law, and outside civilization.
  Many die to a basilisk's bite within a month, but smarter groups can subside for years, or even decades, and we can only route them out by entering the forest and reading the tracks.

\end{exampletext}

\subsection{Circles of Civilization}

While maps of Fenestra, made by men, focus on cities and villages, fields and coppiced trees; the truth is that the world is mostly made of dangerous terrain, running wild with dangerous creatures.

\subsubsection{The Outer Darkness}

Throughout most of Europe's history, we were the nastiest, scariest things around.
We could wander freely, and use the land as we saw fit.
We tamed forests by cutting down the under brush so we could hunt game more easily, and farm any land we could.
Fenestra, by comparison, sustains a much smaller population per square mile.
The continent remains wild, and only little blotches of civilization exist, with rare roads running between them.

These regions of outer darkness, placed on maps using guesswork and rumours, form the larger part of the world.

\subsubsection{Lonely Roads}

Roads which lead from town to village, or between villages, provide easy walking for people and horses.
But the roads which connect two great cities by cutting through a large, wild, forest, present far more danger.

Sometimes these lonely roads break.
Whenever people wander down one, they might not return.
However, if too many parties go down one but do not return, the people know that the lonely road has been closed.
Typically a  road's closure is resolved by a band of armed warriors who go to clear and cleanse whatever lies on the road to eat people.
If they don't return, then the road lies closed for good, and people have to take another road to civilization, or forge a new one through the wild forest.

\subsubsection{Chaos at the Edge of Civilization}

Exactly what lies in wait for people outside the small civilized lands depends upon the area.
Mount Arthur has bears, giant arachnids, griffins and more.
The frozen Eastlake area in the North tends to have a lot of undead.
Quennome has every creature one can name, in addition to strange monsters which defy classification.

Long roads, connecting different civilizations, wander through the forest for many miles.
These long roads are only taken by the suicidal, or by groups of armed men.
The exact number of soldiers depends upon the area, but typically six to twelve can keep themselves safe if they take turns at watch during the night.

The forests hold so much edible material -- fruits, vegetables, roots, and game -- that people could live easily within them were it not for the creatures which hunt them.
For this reason, outlaws commonly make little liveable spots, either in a self-made shelter, an abandoned stone building in the forest, or anywhere else they can put up enough of a wall to stay safe.
In this way, any group can keep themselves fed until the food in the local area runs out, or until a cold season hits.
In general, such groups do not have the organizational skills to survive, so they either die one by one, as the forest eats them, or they turn to banditry, and someone comes for their heads.

From the point of view of civilization, the greatest dangers come from any element which can organize the creatures of the forest.
Sometimes this is a necromancer, able to summon the dead, and intent on taking out villages.
At other times, an old elf has become irritated with humanity's encroachment on a nice forest, and decides to organize the creatures of the forest to attack, and trees to grow tall and reclaim the land.
These `forest masters', or `beast masters' pose such a danger that local lords must send specialized hunters after them.
Sometimes a full army will go, but smaller teams are often preferred.
When necromancers kill large armies, the lot can be turned undead, and when priests of the forest sing enchantment spells over a wide area, the extra numbers offered by an army do nothing to help the battle.

\subsubsection{Villages \& Walls}

Villages have numerous ways to stay safe, from staying on small islands to building massive wooden walls.
Many build massive moats around their lands to keep their animals safe, while others keep their animals in barns, and post watchmen with bows to guard them through the night.
Predatory creatures do not always come out during the night, but during the day people have less to fear because they can see danger a long way off, and fire arrows before it arrives.

Villages almost universally cut down all vegetation in the area to give themselves better visibility.
Farther afield, villagers allow trees to grow so they can grow them into the correct shapes for quarterstaffs, or use them to make arrows.
Massive orchards can be left safely outside, as the animals of the forest already have plenty of fruits to eat.

Villages typically surround a town in every direction, meaning that those close to a town or city can rest easy;
any creatures wandering from the forest will typically encounter trouble with those in the outer layers before getting anywhere near the inner circle.
Meanwhile, those poor villages in the outer circle can see a dark, primordial forest every day.

If a village defends itself well, it can grow, and one day may create another village farther out, pushing further into the deep forest.
However, this push-and-pull game does not always go so well for people.
When a village has too many young archers die, or too many livestock stolen to feed itself properly, it can no longer defend itself, and the remaining inhabitants must flee to neighbouring villages, or into a town, where most will have to join the \gls{guard}.

\paragraph{Dwarves} tend to live underground, with tight fortifications, and almost always maintain a direct, safe, tunnel to some nearby dwarvish city.

Despite their relative safety, dwarvish parties must still venture out in order to hunt for more seams, or establish fertile mushroom gardens.

\paragraph{Elves} build small villages almost exclusively.
Each one needs only one or two powerful spellcasters and the rest can remain safe.
The exact magics employed vary from village to village, but they might include a spellcaster who can sense any nearby dangers and incinerate them, or someone who can bless all other villagers with luck when they leave.

\paragraph{Gnomes} tend towards hidden villages, but a few cities remain within Fenestra.
They rely extensively on traps both underground and above ground.

\paragraph{Gnolls} keep plenty of fierce guard dogs around their area to alert them to wandering monsters.
Every gnoll in a village knows they must run and hunt at the first sign of danger.
Gnolls welcome such incursions more than any other race, as they enjoy meats of any creature.

\subsubsection{Towns, Guilds \& Temples}
\label{guilds}

Within every town in Fenestra, divine monopolies are officially enforced.
People must seek legal rulings in a temple of V\'{e}r\"{e}, swords are only sold from the temples of Ohta, and every tavern, at least in some official sense, is a temple to Alass\"{e}.

\paragraph{Alass\"{e}} governs `the ale guild', and all manner of taverns.  Officially, all taverns are temples to Alass\"{e}.

\paragraph{C\'{a}l\"{e}'s} temples doubles as paper-producing guilds, and provided all townmasters and areamasters with seneschals to count up the lord's holdings and due taxes.
Many are now being replaced with the \gls{king}'s own accountants.

\paragraph{Laiqu\"{e}} was at one point in charge of grain supply and various tasks related to farming.
The temple have since abandoned all such activities, and mostly abandoned any buildings they once held in towns.
The priesthood have stated their intention to work purely on theological matters; as a result they hold the highest portion of efficacious miracle workers.

\paragraph{Ohta} rules over warfare and many call the temple `the Sword Guild', as it has exclusive jurisdiction over the sale of all weapons.

\paragraph{Qualm\"{e}} does not deal with much beside funerals, and once dealt with death-payments, made when a murderer must make pay the expected value of the victim to the victim's family.
These services were not popular, and death-payments were soon taken over by the Verean temples.
After that, the church had borrowed too much so the temples were sold or abandoned.
A few remote priests decided to pass into undeath and remain in their abandoned monasteries in a sad, robotic, and bitter state.

\paragraph{V\'{e}r\"{e}} has become central figure of `the Justice Guild'.
People approach the temples of V\'{e}r\"{e} for marriages, court rulings, and to make public business deals.

\subsubsection{\Gls{shatteredcastle}}

\Gls{king} rules the land completely from every area at once, with the single exception of Liberty.
Not long ago, these lands had separate rulers, and in those times each lord of a land -- large or small -- created a personal militia to deal with problems.
Nowadays, all personal armies have been made illegal.
People may hold weapons, but nobody may have a standing army.

In place of the local militias, \gls{king} has created the \gls{guard} to look after the realm.
Anyone unable to find proper work, goes to work in the \gls{guard}.

\end{multicols}


