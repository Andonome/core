\chapter[Codes of Belief]{Codes}
\label{gods_codes}
\index{Gods}
\index{Codes of Belief}

\begin{multicols}{2}
\noindent
Players can receive additional \gls{xp} for following their beliefs.
While anyone is free to give offerings to any of the gods, most people have a primary god they worship, suggested by their birth but decided in adulthood based on shared values.
Others follow no god but have a code of some type, guiding their actions.
These codes are not formal beliefs, written as law and discussed at meetings but rather a set of aspirations which some have.

Each god has a holy day marking its favourite time of year.
During the holy day, anyone can earn \gls{xp} by following the edicts of the god, even those who follow others.
The day of \gls{wargod} is a day to remember war and settle disputes by fist or steel, the day of \gls{joygod} is one of joy, to be celebrated with pranks and presents.

The gods are most popular with humans and gnolls. Most dwarven settlements have a temple of some kind but it is not something all dwarves take much interest in except during odd times when they want to pay for a blessing. Gnomes' interactions with the gods mainly consists in chronicling legends about them and debating the nature of divinity, but not actively worshipping them. Elves, it is said, do not have the humility to worship anything.

The gods presented here are the most important -- they are the ones featured in the larger tales and who have the most prominent holy days. There are, however, many more. Each region or individual tribe has its own little god. Players are encouraged to create their own.

Each god has a holy day marking its favourite time of year.
During the holy day, anyone can earn \gls{xp} by following the edicts of the god, even those who follow others.
The day of \gls{wargod} is a day to remember war and settle disputes by fist or steel, the day of \gls{joygod} is one of joy, to be celebrated with pranks and presents.

Those without a dedicated deity often dedicate themselves to some informal code instead.
The codes might be thought of as attitudes or philosophies for life.
Followers of similar codes may well get along together but they will not recognise each other as members of a similar organisation.
Those with a code as their primary motivator may also sacrifice to gods or even occasionally worship and donate to temples, but their ultimate aims lie with themselves.
It is said those who do not fully dedicate themselves to any god must wander the afterlife without aid or guidance -- such spirits always provide the most bizarre and contradictory accounts of death and can prove difficult to summon with Necromancy.

Those with a personal code can never walk the Path of Divinity.

\end{multicols}

\section{Rulings}

\begin{multicols}{2}

The \gls{gm} decides how much \gls{xp} to give out for any given task -- each path has a number of suggestions but the list should be understood as open-ended and entirely at the whim of the \gls{gm}.

Players can only gain each reward once per session, and only for the greatest reward of any type,
so a follower of \gls{deathgod} can receive 5 \glspl{xp} for  \glspl{hp} only once per session, and would not also gain 1  \gls{xp} for losing a single \gls{hp}.

Some codes give a reward for donating or gaining gold.

\noindent Warden to all oaths, lord of ten thousand holy warriors, leader of armies, the giver of vengeance and punishments -- \gls{justicegod} is a popular god.
He is invoked during wedding vows and business deals.
His followers are found among the politically influential and can be some of the most zealous of religious followers.
He values obeying the law, making fair deals, being a good host and supporting the poor.

His holy day is during the second season of the second cycle.
It is considered extremely good faith to make an oath on this day, and mortally bad luck to break such an oath.

\subsubsection{Spheres}

\noindent \Glsentrytext{justicegod}'s clerics can access the enchantment and Force spheres.
They use enchantment to gain followers, dazzling them with the glory of the purity and strength of their god, while force is used to protect the innocent and faithful.
Their spells appear in a shimmer of gold and the sound of a gong.

\subsubsection{Mana Stones}

\Gls{justicegod}'s mana stones are always people who are followers of \Gls{justicegod}.
Those believers alone can activate any spells which are stored inside them.
Priests of \Gls{justicegod} often gift their followers with single-use magical powers, such as the ability to call upon a blessing or the ability to protect themselves with armour.
If the people who are being used as mana stones are given spells then they can activate those spells at will with a short prayer by spending 4 \glspl{ap}.

\end{multicols}
