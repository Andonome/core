\chapter[The Arena]{Combat}
\index{Combat}
\label{combat}

Don't roll for initiative!
Instead, each character's \textit{Speed} determines how many \glsentrylongpl{ap} they start with. 
Characters attack with any number of Resisted rolls, but generally use their \textit{Dexterity} and a M\^{e}l\'ee Skill.
With a standard attack, the winner of the Resisted roll then uses their \textit{Strength} to deal Damage.%
\footnote{Having one roll for a `clash' saves a lot of time.  Instead of rolling for `initiative, attack, dodge, attack, damage', players can just spend \pgls{ap} and attack.}

Combat will generally only last a couple of `rounds', during which the \glspl{pc} could take one action, or six, or even more, depending on circumstances and their Traits and equipment.

\section{Raw Combat}

\begin{multicols}{2}

\subsection{Attacking}
\label{attack}

Characters generally fight with a Resisted \roll{Dexterity}{Combat} roll, but any kind of Resisted roll works, as long as it makes sense.
In fact, a round of combat could easily go like this:

\begin{exampletext}
  Cleftrank charges towards the elven enchanter as the enchanter points behind Cleftrank, and asks `what is that thing, crawling over your shoulder?'.
  The player insists his character would ignore this and keep on charging, so the \gls{gm} allows him to resist with a charge, as he already began the assault.

  \small{(The \gls{gm} adds the enchanter's \roll{Charisma}{Mind} to the \gls{tn}, for a total \gls{tn} of 12; Cleftrank's player resists with his \roll{Speed}{Athletics}, and wins the roll)}

  My character swings his axe through the elf's jaw to shut him up.

  \small{(The elf continues attempting the enchantment spell, so the player rolls at \gls{tn} 12 again, but this time he uses \roll{Dexterity}{Combat}, and the axe costs 3 \glspl{ap} to use)}

  `Cleftrank looks behind himself for a moment, then checks his other behind, and somewhere in his confusion he spots the enchanter beside him, taking out a dagger.'

  \small{(Cleftrank loses the roll, so the enchantment spell strips him of 2 \glspl{ap}, which puts him at -1 \glspl{ap}; he cannot run away)}

  The enchanter drives the dagger down!

  \small{(If Cleftrank defends with the axe, it will cost him 3 more \glspl{ap}, putting him down to -4.
  He cannot deal with a -4 penalty, so he decides to use the Brawl Skill instead)}

  (The enchanter's \roll{Dexterity}{Combat} puts the \gls{tn} to 10.
  Cleftrank rolls his \roll{Dexterity}{Brawl} with a -1 penalty\ldots and wins)

  \small{(The elf buckles in pain from the kick, then turns to run)}

\end{exampletext}

When combat begins, roll your \roll{Dexterity}{Combat} against \tn[7], plus the enemy's \roll{Dexterity}{Combat}.
If you win, roll to see how much Damage you deal to the opponent.
If your opponent wins, the \gls{gm} rolls to see how much Damage your opponent deals you.
Finally, on a tie, you can decide to both take \emph{and} deal Damage, or neither.
In the former case, you and your opponent hit each other at the same time.
In the latter, blades clash but nobody gets hurt.

\begin{figure*}[t!]
  \stackingDamageChart
\end{figure*}

\subsection{Damage}
\index{Damage}

If you hit, roll $1D6$ plus your Strength Bonus to determine Damage.
The Damage is then taken off the enemy's \glspl{hp}.
Everyone has a number of \glspl{hp} to withstand Damage.
When they reach 0 \glspl{hp}, they fall over.

\subsubsection{Stacking Damage}
\index{Combat!Stacking Damage}
\label{stackingDamage}

Damage Bonuses cannot extend forever.
Replace every +4 Damage Bonus by an additional $D6$.
It continues through all Damage Bonuses, so $1D6+9$ Damage becomes $3D6+1$ Damage after conversion.

\subsubsection{\Glsfmtlongpl{hp}}

Each character has a number of \glsentryfullpl{hp} equal to 6 plus their Strength Bonus.
Small gnomes typically have 4 \glspl{hp} while big, strong humans typically have 7.
Losing even a single \gls{hp} means the character has suffered serious Damage.
A long fall might have broken the character's bone.
A dagger could have slashed open several veins.
Characters do not have many \glspl{hp} so losing even one is a serious matter.

\subsubsection{Death}
\index{Death}
\label{death}
Once a character reaches 0 \glspl{hp}, they fall over, and don't get up.
Anyone can attempt to save them by bandaging up their wounds, or staving off a concussion, with a \roll{Wits}{Medicine} roll.
The \gls{tn} is 7 plus the number of \glspl{hp} the character has fallen below 0, so someone at -3 \glspl{hp} would need a roll at \gls{tn} 10 to save.

\paragraph{If the healer rolls the \gls{tn} exactly,}
the character has survived, but with a permanent wound.
The players must select one Attribute, and give it a penalty equal to $1D6$.
If the Attribute falls below -5, the character dies.

A Charisma penalty might suggest a partly broken jaw, leading to a permanent speech impediment.
An Intelligence penalty might represent a brain-injury.

\paragraph{If the healer fails the roll,}
anyone else can have another attempt, but with a -1 penalty (bad healers can make the problem worse).
If nobody succeeds in the roll, the character dies.

\paragraph{A successful check}
means that the character is unconscious for the remainder of the interval, but still alive.
At this point, the rest of the troupe will have to carry their fallen comrade back to safety -- if they can.%
\footnote{See page \pageref{weight} for \nameref{weight}.}

\subsection{\Glsfmtlongpl{ap} \& Initiative}
\label{actionPoints}
\index{Initiative}

Everyone begins combat with 3 \glsentryfullpl{ap} plus their Speed Bonus.%
\footnote{Anyone with a Speed Bonus of -3 can act on Initiative 0, but only once everyone else has reached Initiative 0.
Those with a lower Initiatives must wait one cumulative round extra, before acting.}
Every action requires spending some number of \glspl{ap}.
Once someone reaches 0 \glspl{ap}, they cannot initiate any actions, but they can still engage in any Resisted actions if attacked.

\subsubsection{Negative \Glsfmtlongpl{ap}}
Anyone overspending \glspl{ap} enters \emph{negative} \glspl{ap}, and receives a penalty to all actions equal to their negative score.

\begin{exampletext}
  A character on 2 \glspl{ap} who attacks with a greatsword (which costs 4 \glspl{ap}) would then go to -2 \glspl{ap}.
  If someone attacked them, they would have to respond with a -2 penalty to their action, then reduce to -6 \glspl{ap}.
  Using big weapons gives big bonuses, but they bring their own dangers!
\end{exampletext}

\subsubsection{Combat Order}
\index{Initiative}

During combat, anyone may take actions without a set sequence\ldots until two characters vie for priority.
At that point use the following method to determine who goes first:

\begin{enumerate}
  \item
  Whoever currently has the most \glspl{ap}.
  \item
  Whoever is spending the \emph{least} \glspl{ap}.
  \item
  Whoever has the highest Speed Bonus.
  \item
  Whoever has the highest Wits Bonus.
  \item
  Dice roll! ($1D6$ each)
\end{enumerate}

\subsubsection{\Glspl{quickaction}}

\Glspl{quickaction} require spending an \gls{ap} immediately to react to a situation.
This includes getting attacked by someone or moving away from someone who approaches you.

Resolve priority upon simultaneous action with the Combat Order rules above.

\paragraph{Guarding}
\label{guarding}
allows any character to move up to 1 \gls{step}, position themselves in front of another player, and receive all attacks from their front.
Anyone attacking a guarded character must first make a standard combat roll against the guardian, and if that attack succeeds they deal no Damage, but have the option to make a second attack, as a \gls{quickaction}, against the guarded character.

If a guarded character moves, they lose the benefits of their guardian.

\paragraph{Moving}
\label{moving}
lets the character travel up to 3 steps plus their Athletics Skill.

\paragraph{Speaking}
requires 1 \gls{ap} if any player tells another to act, stop, or guard them.
During combat, everyone should focus on the task at hand, and communicate sparingly, only when they need to say something vital.

\end{multicols}

\section{Equipment}

\begin{multicols}{2}

\subsection{Weapons}

\noindent
Weapons are a great way of inflicting additional Damage, but they are an equally excellent way of defending oneself.
Having a longsword to keep scary opponents at bay is always better than trying to nimbly dodge about.
Longer weapons grant an Attack Bonus, allowing someone to hit the enemy before the enemy hits them, and heavy weapons tend to deal more Damage.
However, both of these come at the cost of extra \emph{heft}; they take more time to swing, and so cost more \glspl{ap} to use.

Each weapon has the following properties:

\begin{itemize}

  \item
  \textbf{The Attack Bonus:} adds to the Attack roll, representing reach and manoeuvrability.
  \item
  \textbf{The Damage Bonus:} adds to the Damage of a successful Attack roll.
  This might represent sharpness in a dagger, or just sheer weight in the case of a war hammer.
  \item
  \textbf{The \Gls{ap} Cost:} shows how many \glspl{ap} the player spends after engaging in an Attack roll (whether attacking or being attacked).
  It represents a weapon's inertia (and hence difficulty in pulling it back from a swing), and allows enemies with lighter weapons to `close the gap'.
  \item
  \textbf{The \glsentryname{weight}:} means that your character must have at least double this number of \glspl{hp}, or they will struggle to use the weapon.
\end{itemize}

\weaponsChart

\bigLine

\label{weaponschart}
\index{Weapons}

\subsection{Shields}
\index{Shields}

Shield allow attacks to be blocked with ease.
Characters with a shield may use it in lieu of their weapon in order to defend against an Attack, but a successful roll only indicates that they have received no Damage, and do not deal Damage.

Characters typically use shields when overwhelmed, allowing them to defend against attacks with a lower \gls{ap} cost than most weapons.

Shields can also be used like weapons.
Their Attack Bonus is 0, their Damage Bonus is equal to their \glsentryname{weight}, and their \gls{ap} cost is 1 higher than normal.

\shieldchart

\boxPair{
  \subsubsection*{Example}
  An armoured knight stands with a flail.
  His basic Attack Score is 10, so rolling 10 means a draw.
  His full plate armour has \pgls{covering} of 5,
  so to hit him, a \gls{pc} will need to roll the $\gls{tn} + 5$.

  Rolling 9 or less means the \gls{pc} is hit, and since he has no armour, that won't go well!

  \begin{boxtable}[cLc]
    \textbf{Roll} & \textbf{Result} & \textbf{Margin} \\
    \hline
       $\leq 9$ & \gls{pc} is hit, taking full Damage! & -1! \\
      10 & \emph{Draw} (\gls{dr} applies) & 0 \\
      11 & \Glsentrytext{npc} is hit, but \gls{dr} applies & \textbf{1} \\
      12 & \Glsentrytext{npc} is hit, but \gls{dr} applies & \textbf{2} \\
      13 & \Glsentrytext{npc} is hit, but \gls{dr} applies & \textbf{3} \\
      14 & \Glsentrytext{npc} is hit, but \gls{dr} applies & \textbf{4} \\
      $\geq 15$ & \emph{\Gls{vitalShot}!} -- full Damage to \gls{npc} & \textit{5!} \\
  \end{boxtable}

  In this case, the \gls{pc} rolls a 15, so he hits for 6 Damage, and the knight loses 6 \glspl{hp}.

}{

  \begin{tikzpicture}
      \node[anchor=south west,inner sep=0] (image) at (0,0) {\pic{Roch_Hercka/vitals_shot}};
      \begin{scope}[
          x={(image.south east)},
          y={(image.north west)}
      ]
          \foreach \mNum/\mX/\mY in {%
            {\huge$\leq 9$!}/22/25,
            10/54/20,
            11/30/63,
            12/37/45,
            13/52/63,
            14/31/70,
            {\huge 15!}/42/38,
          }{
            \mapLegend{\outline{\mNum}}{\mX}{\mY}{\large}
          }
      \end{scope}
  \end{tikzpicture}
}

\subsection{Armour}
\index{Armour}

\input{config/rules/armour.tex}

When armour works, its \gls{dr} reduces incoming Damage.
This occurs before Damage applies to \glspl{fp}, so \gls{dr} and \glspl{fp} make a powerful and steady combination.

\subsubsection{\Glsfmtplural{vitalShot}}
\label{vitals}
\index{Combat!\Glstext{vitalShot}}
Just as real armour covers parts of the body, armour in BIND covers numbers on the dice.
An armour with \gls{covering} 2 can protect characters from a little Damage whenever they miss an Attack roll by 1 or 2.
But if they roll lower than this, the armour does nothing, as the attack strikes flesh.
We call this a `\gls{vitalShot}'.

When trying to hit, \glspl{pc} will need to roll 3 over the creature's \gls{tn} (or more, for opponents with more \gls{covering}).

\armourchart

\noindent
\subsubsection{Armour Types}

\paragraph{Padded}
if you tie enough wool, dishcloths, old shoes and sticks together, you can make some passable armour.
Of course it stinks, and weighs a tonne, so best leave it until the most desperate of times.

\ifnum\value{r4}=3
  \paragraph{Elvish Ceramic}
  has very little strength on a piece-by-piece basic, but once worn in the correct arrangement, it gains surprising strength, and makes a pleasant `clinking' sound.
  It also sells well to human nobles who want to armour their children for long road trips. 
\fi

\paragraph{Leather}
armour is made by boiling leather for a long time, until it becomes extremely tough.
While anyone can pick up leather cheaply, very few have the skill to make this armour.

\paragraph{Chain mail}
is a covering of chain links, placed over some other protection.
It might be placed over hardened leather, or just some thick padding.
The chain itself only protects against a weapon's blade, not the weight, while the under-layer protects against heavy weapons, such as hammers or mauls.

\paragraph{Plate armour}
involves adding all of the above into one armour, with sheets of metal on top.
Allowing someone to move within this pile of metal requires rare artisans.


\paragraph{Natural Armour}
means tough skin (or scales, or chitin\ldots) thick enough to push back blades.
Natural armour always has \pgls{covering} of 4 unless otherwise specified, because it covers almost all of the body, but still leaves weak spots open such as the eyes or the kneecaps.

\begin{figure*}[b!]
  \projectilesChart
\end{figure*}

\subsubsection{Stacking Armour}
\label{stackingarmour}

Some creatures have a natural \gls{dr}, which would then stack with their armour.
The primary armour counts for its full value, and the lower \gls{dr} score counts for half.
Any tertiary armour counts for a quarter, and so on.
Once you have a total, round up anything over half.
Stacked armour can consist of different levels of \gls{covering}, meaning a roll could bypass one set of armour by rolling 3 over the creature's \gls{tn}, while another type of armour (with \pgls{covering} of 4) still applies.

Consider this convoluted example: a basilisk with natural \gls{dr} 4 dies, and then get raised from the dead by a necromancer.
The undead naturally have a \gls{dr} of 2, so this secondary source of damage would count for half, giving it a total \gls{dr} of 5.
If the necromancer fashioned plate armour to the basilisk, the total \gls{dr} would be\ldots

\begin{center}
{
  \LARGE $5 + $\Large$4\frac{1}{2} + $\normalsize$2\frac{1}{4} =  7.5$
}
\end{center}

\ldots or `8' (after rounding up).

If the plate armour had a covering of only 2 then rolling 3 over the creature's \gls{tn} would leave it with a \gls{dr} of only 5.

Standard armour cannot be stacked in this way.
We assume plate, chain, and some leather-based armours already have padded armour underneath.
Similarly, different types of natural \gls{dr} do not stack, and nobody becomes undead in different ways.

The only different \emph{kinds} of armour can stack up in this manner, so players will almost never see it in a game.

\end{multicols}

\section{Ranged Combat}\index{Ranged Combat}

\begin{multicols}{2}

\noindent
Projectiles have their own M\^el\'ee Skill, which covers everything from javelins to bows.
Archers roll to hit with Dexterity + Projectiles, resisted by the opponent's Speed + Vigilance, then (if successful) roll for Damage, just like toe-to-toe combat.

\input{config/rules/projectiles.tex}

However, unlike toe-to-toe combat, those on the receiving end cannot reflexively dodge; they must have at least 1 \gls{ap} to spend in order to add their Speed + Vigilance to the archer's \gls{tn}.

Just as with weapon combat, a high enough roll can be a \gls{vitalShot},%
\footnote{See \autopageref{vitals}.}
ignoring all \gls{dr}.

All projectiles suffer from the need to reload.
As with picking up any other item, characters must spend least 1 \gls{ap} to take out and use arrows.

\paragraph{Cover}
provides a variable Bonus, depending on the size.

\begin{boxtable}
  +1 & Large bushes \\
  +2 & Tower shield \\
  +3 & Murder hole  \\
\end{boxtable}

\subsubsection{The Long Bow}
\index{Projectiles!Bow}
\index{Bows}
\label{longbow}

Long bows (or `hunting bows') are difficult things to work but well worth it once the archer practices enough.
To use a bow, the archer must have a Strength Bonus at least as high as the bow's Damage.
So if a hunting bow deals $1D6+2$ Damage, the archer will need at least a Strength of +2 to draw the arrow properly (or at all).
Having a Strength of 3 will not increase the Damage, but it can decrease the firing time.

To pull back the heavy load on a long bow requires 2 \glspl{ap}, plus the bow's Damage bonus, so a bow which deals +3 Damage requires 5 \glspl{ap} to fire.%
\footnote{
  This may seem harsh, but bows really are harsh.  Archers pointing their bow downwards and drawing an arrow back must pull a weight similar to lifting a human by their foot.

  Oh, and nobody can aim a bow and just keep the arrow held like a gun.
  Nobody's that strong.
}

Long bows can be fired for hundreds of yards -- the maximum range is generally more determined by the archer's ability to aim rather than the bow.

\subsubsection{The Short Bow}
\index{Projectiles!Short Bow}
\index{Short Bow}

A short bow, or `trick bow', is a smaller, lighter thing which can be used by anyone.
What it lacks in punch it makes up for in quick draw time.
As usual, for every five steps beyond the first two the archer suffers a -1 penalty to hit.
The short bow takes 1 \gls{ap} points to fire, and 1 \gls{ap} points to reload, so many shots can be fired in a \gls{round}.

Shortbows have a maximum range of 20 steps and deal $1D6-1$ Damage.
They often bring down prey with multiple arrows rather than the one.

\subsubsection{The Crossbow}
\index{Projectiles!Crossbow}
\label{crossbow}
Crossbows can be powerful, but are not easy to reload.
They have a standard Damage of $2D6$, though different crossbows vary in quality.
Crossbows requires only 1 \gls{ap} to fire, but require 5 rounds, minus the user's Strength Bonus, to reload.
Reloading always takes a minimum of 1 round.

\subsubsection{Thrown Weapons}\index{Projectiles!Thrown Weapons}

Thrown weapons such as knives, spears or others are typically not great at killing enemies, but they can certainly wound them.
They work just as short bows, but their Damage is the normal weapon Damage minus 2.
\javelin s deal
\addtocounter{damage}{4}\dmg{damage}
when used in combat, but only
\addtocounter{damage}{-2}%
\dmg{damage} when thrown.

\subsubsection{Impromptu Weapons}
\index{Projectiles!Impromptu}

Weapons which were never made to be thrown, such as swords, axes, or most knives, receive a -2 penalty to hit for every 5 steps distance from the target, and a -2 penalty to Damage.
\longsword s don't make great projectiles, but they still deal
\addtocounter{damage}{2} \dmg{damage}
basic Damage.

\end{multicols}

\startcontents[Manoeuvres]

\section{Complications \& Manoeuvres}

\begin{multicols}{2}

\subsection{Complications}

These rules cover things that happen to characters.
You can refer back to them when necessary with the list in \autoref{combatAppendix}.

\subsubsection[Blindness: Roll at -6 penalty, + (Wits + Vigilance)/ 2]{Blindness}
\index{Combat!Blindness}

Fighting while blind is no fun -- your opponent can see you coming, and you can't see them.
Blinded suffer a -6 penalty, but can offset this with half their Wits + Vigilance total.
For example, a character with with a Wits + Vigilance total of -1 would receive a -7 penalty to attack, while their companion with a total Bonus of +3 would suffer only a -4 penalty.

This penalty only counts when one side of a fight is blind. When both sides are blind, we use the Darkness Fighting rules below.

While fighting blind, if the dice make a \gls{natural} roll equal to the number of people on the character's side side (including themself) then they hit a companion while also being hit.
If the character is fighting with just one companion then there are two of them and they hit a companion on the roll of a 2.
If they are part of a group of 5 people, any roll of 5 or under means they have accidentally hit a companion.
Companions who are are accidentally hit can evade by simply spending 1 \gls{ap}.
It is quite possible to kill a companion while fighting blind.

\subsubsection[Darkness: Roll Bonus cannot exceed Wits + Vigilance]{Darkness}
\label{darkness}
\index{Darkness}
\index{Combat!Darkness}
\index{Spelunking!Fighting in darkness}

Fighting in the darkness, or just twilight, can give a distinct advantage to those with sharper senses.
Those who retain some basic vision while their opponents have none are in a similar situation to fighting a blinded opponent.
However, when both sides suffer from the darkness, the battle changes very little.
Neither side can hit very accurately, but then neither side can dodge or parry very well either.

\paragraph*{When fighting in total darkness}
characters Attribute bonuses cannot go beyond the character's \roll{Wits}{Vigilance}.

\begin{exampletext}
  For example, a human guard has caught a room full of elves with stolen goods.
  Thinking quickly, one of the elves douses the room's only lantern.
  The human has a Wits Bonus of -1 and no Vigilance Skill, so his maximum roll has a -1 penalty.
  The elves have a total \roll{Wits}{Vigilance} of +3, so their \roll{Dexterity}{Combat} has only a +3 cap.
\end{exampletext}

\paragraph*{Fighting in minimal light}
(such as a moonless night)
only gives a -1 penalty to characters with a \roll{Wits}{Vigilance} lower than their roll.

\subsubsection[Dual Wielding: Both weapons count has having +1 \glsentrytext{weight}]{Dual Wielding}
\index{Combat!Dual Wielding}

A character using two weapons -- perhaps a shield in one hand and a sword in the other -- can use either weapon at any point.
However, both count as having double their standard \glsentryname{weight}.

Shields can be strapped to the arm, without requiring any kind of dual-wielding.

\makeAutoRule{enclosedcombat}{Enclosed Combat}{penalty equals weapon's \glsentrytext{ap} cost minus 1}
\index{Spelunking!Enclosed Combat}

Enclosed spaces cause serious problems for people wielding longswords, battle axes, and other large weapons.
Daggers and shortswords often have an easier time in these locations.

When a character has no space to swing a weapon -- either vertically or horizontally -- their Attack gains a penalty equal to the weapon's \gls{ap} cost minus 1.

\subsubsection[Holding Breath: 1 \glsentrytext{ep} per negative \glsentrytext{ap} at the end of the round, plus 1~per round]{Holding Breath}
\label{holdingBreath}
\index{Combat!Staying Silent}

Anyone holding their breath takes a number of \glspl{ep} equal to their negative \glspl{ap} at the end of the round, plus one.
So someone ending a round on -3 \glspl{ap} would take 4~\glspl{ep}.

Those simply holding their tongue, and trying to kill silently, without battle-cries or similar, can do so as long as they do not fall below 0 \glspl{ap}.

These \glspl{ep} points disappear at a rate of 1 per round as soon as the character can breathe again.

\makeAutoRule{entangled}{Trapped, Entangled, or Prone}{the character loses their Dexterity Bonus, and takes -2 Attack penalty}
\label{trapped}
\index{Combat!Prone}
\index{Combat!Entangled}
\label{prone}

Characters caught in mud, who slip over, or get shackled to a spot cannot move or dodge nearly as well as they could.
They cannot use their Dexterity Bonus in Combat (penalties still apply), and take a -2 penalty to rolls using their Physical Attributes.

\subsubsection[+1 to attack]{Higher Ground}

Gaining the higher ground grants +1 to the Attack Bonus, as you target the enemy's head easier and gravity aids your downward swings.

\subsection{Manoeuvres}

These additional actions cover different ways to engage with enemies.
Anyone can use them at any point, if they use the right weapons.

\subsubsection[Brawling: Make a normal attack roll, but any attack with a Margin less than 5 only inflicts \glspl{ep} rather than Damage]{Brawling}
\index{Combat!Brawling}
\index{Brawling}

Anyone can go for a brawling manoeuvre, even while using a weapon.
Swinging an axe can place one in a vulnerable position -- on negative \glspl{ap}!
But since these attacks cost only 1 \gls{ap}, they won't deplete \glspl{ap} too fast.

Punches and kicks all use the Combat bonus.
Such attacks inflict \glspl{ep}.
Everyone gains a \gls{dr} against Brawling Damage equal to their Strength Bonus, which stacks with armour (\gls{dr} cannot be negative).
This counts as Complete armour, so hitting someone in Partial chainmail with a \gls{tn} of 8 and a Strength of +1 would mean they have a total \gls{dr} of 6.
However, an attack score of 11 would mean that the Partial armour's \gls{dr} could be ignored, leaving only a \gls{dr} of 1.
An attack score of 13 would ignore both types of \gls{dr}, leaving nothing at all.
Attacks which bypass a body's natural armour count as normal Damage as such attacks might hit vulnerable locations such as the eyes or crotch or twist an opponent's arm till breaking point.

\subsubsection[Close Magic: Casters roll vs the enemy's standard Attack score]{Close Magic}
\index{Combat!Magic}
\index{Magic!Close}

Casting spells in close-quarters combat works just like any other type of combat.

\begin{exampletext}
  A mage with Air 2 finds himself surprised by a cut-throat.
  The cut-throat attacks with a knife, and with Dexterity +1 and Combat +2, making the attack's \tn[10].

  The caster decides to spend only 1 \gls{mp} for a fast, easy, \textit{Wind Blast} spell, and with a \roll{Charisma}{Air} total of +4, he will reach a tie on a \gls{natural} of 6.
\end{exampletext}

Casters must meet their usual \gls{tn} to cast a spell -- if this fails, then they lose to their opponent, and receive Damage as usual.

If a mage fails to cast a spell in combat, they do not lose \glspl{mp} -- the enemy's weapon interrupts the spell before they can begin to cast.

\subsubsection[Drawing Weapon -- Cost: 1 \glsentrytext{ap}]{Drawing Weapons}

Drawing a weapon costs 1 \gls{ap} if it is placed in an easy place to draw, like a scabbard on the side of a belt.
If a character holds weapons on the back or in a bag, they have to rummage for an entire round or more.

\subsubsection[Dropping Weapon -- Cost: 0 \gls{ap}]{Dropping Weapons}

Dropping a weapon costs no \glspl{ap}, though they will be defenceless unless they do this while picking up another weapon.

\pic{Roch_Hercka/stances}

\subsubsection[Flanking: Gain +2 to attack]{Flanking}\label{flank}

Attacks from someone's anterior side gain a +2 Bonus.
Up to 6 opponents can attack a lone character, but any available walls can reduce this number.

\subsubsection{Grabbing \& Grappling}
\index{Combat!Grappling}
\label{grappling}
Some animals, with \textit{Teeth}, can grab and deal Damage.
But most humanoids need to grab first, and \emph{then} deal Damage.

\paragraph[Grabs: Make an attack without any weapon bonus. Both combatants are \textit{Entangled}. Cost: 1 \gls{ap}]{Grabbing:}
requires a standard attack roll, without a weapon.
Both combatants then count as \textit{Entangled}, as neither can move properly to defend themselves.
\label{grab}

No weapons can be used while grappling if they have a \gls{weight} above 1.

\paragraph[Grapple: Make an opposted roll of Strength + Combat.  Success means the combatant can either break free or inflict Damage.  Cost: 3 \gls{ap}]{Grappling:}
allows someone to deal Damage, or break free of a Grab.
Both combatants engage in a resisted \roll{Strength}{Combat} roll.
If the winner decides to deal Damage, they inflict 1D6 + Strength.
Otherwise, they break free, but are still lying prone until they get up.
\label{grapple}

\subsubsection[Guard: Someone must successfully hit you before they are allowed to hit whomever you are guarding. Cost: 1 \gls{ap}]{Guarding}
\index{Guarding}

If you guard someone by standing in front of them then all attacks have to go through you first.%
\footnote{This includes missile attacks only if you could otherwise evade them.}
Any enemy making a successful attack on you can choose to damage you, or to make another roll (as a free action, costing no \glspl{ap}) at their real target.

Guarding costs 1 \gls{ap}, and after that engaging in attacks costs the usual amount.
If either character moves away from the other, the guarding stops.

\subsubsection[Ram: Push the enemy back 2 steps plus the difference between your Strength Bonuses. Resisting costs 2 \glspl{ap}, and requires a resisted Strength + Combat roll. Cost: 3 \glspl{ap}]{Ram}
\index{Combat!Ram}
\label{ram}

In combat, it is possible to scare, push and stab at someone to force them to move backwards.
The attacker spends 3 \glspl{ap} points to rush forward.
The defender can either spend 3 \glspl{ap} and attempt to resist, or can simply acquiesce with a normal movement action, spending 1 \gls{ap}.

Resisting means engaging in a Strength + Combat roll.
When moving back, targets are pushed back 2 steps; the attacker's Strength adds to this and the opponent's Strength decreases it.
Strong characters might also sacrifice the use of 1 point of Strength to push back an additional person.

Characters who have been rammed must be able to move far back enough as part of their normal movement action, otherwise they fall \textit{Prone}.

\subsubsection[Sneak Attack: Dexterity + Stealth vs Wits + Vigilance; automatic \glsentrytext{vitalShot} with +2 Damage.]{Sneak Attacks}
\label{sneakattack}
\index{Combat!Sneak Attack}

Sneaking up on someone requires a Dexterity + Stealth check, vs the target's Wits + Vigilance, but circumstances will change this roll dramatically.
A sleeping target will suffer extreme penalties, while anyone attempting this manoeuvre on a battlefield should receive a hearty laugh from the \gls{gm}, along with \gls{tn} 20.

Heavy weapons do not help much with surprise attacks, as one needs to swing them up into position.
Any weapon with a \gls{weight} higher than 1 adds its \gls{weight} to the \gls{tn}.

If successful, the sneak attack deals +2 Damage, and provides an automatic \gls{vitalShot}.

\stopcontents[Manoeuvres]

\end{multicols}

\section{Chases}
\index{Chases}
\label{chases}

\chasechart

\begin{multicols}{2}

\subsection{Fleeing}

Chases form some of the most dramatic scenes in any mission.
When running on an open field without any barriers, everyone simply runs at full speed -- whoever has the highest Speed + Athletics total succeeds in running away or catching up with an opponent.
But when running through marshes, down alleys, climbing up cliffs, or otherwise finding a reason to change direction, \glspl{pc} must roll.

The system is simple -- one player rolls $2D6$ for the group.
Each person then modifies this group score.
Since the troupe will probably run at different paces, they have the option of abandoning slower members or slowing down to the pace of the slowest member.

The \gls{tn} is 6 plus the enemy's \roll{Speed}{Athletics} Bonuses.
Failure means the characters are instantly caught, before they are able to run anywhere.
If the players hit the \gls{tn} they manage to run through 1 area while being chased.
For every Marginal point, they run through an additional area.
If the Margin is ever 3 or more then they completely evade the enemy.
If the troupe obtain less than total success, they and their pursuers both move and must roll again.

The table is a guide to an unaltered roll. In most situations enemy Traits will affect the actual results of such a total by increasing or decreasing the \gls{tn}.

The \gls{gm} is encouraged to give a fast-paced description of fast-moving scenery, hurriedly telling the players about a new area before moving instantly on.
Each area covered holds new opportunities for getting away, or trapping the quarry -- whether that is the players or their prey.

Characters running through forests might encounter a marshy area, a stream, dense thickets, an open plain and then a sudden, steep hill.
Those crossing plains might find a random encounter in their path, then a copse of trees.
Those running up a mountain could find an area of loose rocks where the ground slides away from under their feet, a narrowing path upwards as rocky walls envelop them and then a misty lake covered in low-lying cloud.

Each area covered also inflicts 1 \gls{ep} in addition to any for wearing armour or for Encumbrance Points.
These \glspl{ep} are applied after every roll rather than waiting until the end of the interval.

Players are encouraged to suggest Skills which might help.
While running away from a band of guards, a character could use the Stealth Skill, quickly dipping into an alleyway to hide.
When jumping around a busy area of town, the character might leap over a moving cart to gain some headway.
Characters can, with \gls{gm} permission, use their Skills to aid an entire group.
The Stealth Skill, in particular, might be used to aid the entire troupe to hide by finding the right spot.
The Empathy Skill might be used to quickly convince farmers to hide the characters.

\subsection{Hunting}

Running after prey follows exactly the same rules, but in reverse.
The players roll to have the \glspl{pc} catch their prey.
As per the previous chart, a near-failure can be worse than a total failure.
With a complete failure, the enemy simply gets away.
With a partial failure, the \glspl{pc} run a long way, get very tired, then fail.
Such is life.

\end{multicols}

\huntchart

\section{Further Dangers}

\begin{multicols}{2}

\makeAutoRule{falling}{Falling Damage}{characters suffers 1 Damage per \glsfmttext{step} in height, plus their own Strength}
\index{Falling}
Falling Damage is +1 per \gls{step}, plus the characters Strength.
Every +4 Damage converts to $1D6$ as usual.

Characters who fall off a building suffer $1D6-2$ + Strength.

Smaller creatures suffer less from falling, so a gnome with Strength -2 can easily fall 2 \glspl{step} without pain, while a human who lands wrong can break an arm by falling badly from a horse.

Characters falling straight downward can attempt to mitigate up to 4 Damage by rolling \roll{Dexterity}{Athletics} (\tn[5] +1 per Damage).
Those falling forward and down in an arc can mitigate up to 6.

The maximum Damage someone can suffer from a fall is 18, equating to $4D6+2$ + Strength.

\subsection{Animals}

\subsubsection{Claws}
\label{claws}
Long claws inflict +1 Damage during Brawl-based attacks.

\subsubsection{Fangs}
\label{teeth}
\label{fangs}

Pronounced fangs and allow animals to grapple and damage with the same attack.
So when an attack is successful, the target both receives Damage and counts as \textit{grappled}.%
\footnote{See above, \autopageref{grappling}.}

\subsubsection{Flight}

Flying creatures generally have light, hollow bones, and delicate wings instead of muscular forearms -- they cannot take off into the air properly if their bodies are too heavy.

For full flight, a creatures needs a higher Speed Bonus than their own Strength Bonus.
These creatures can take off in a moment, and begin to soar at eight times their usual movement speed.
Those with equal Bonuses can still fly, but must spend a full round sprinting at full speed in order to take off.
Finally, heavier flying creatures can only begin a glide by first climbing somewhere high, and then leaping into the air.
Once soaring, they can often remain in the air for some time, and even gain a little altitude.

Half of a character's items' total \gls{weight} forms a penalty -- flying with sacks full of gold won't last long.

Anyone with the Air Sphere can use this as a Bonus to aid their flight, but will have to spend 1 \gls{mp} to begin the spell each \gls{interval}.

\pic{Roch_Hercka/conjuration_right}

\makeAutoRule{swarms}{Swarms}{each \glsfmttext{hp} occupies one \glsfmttext{step}, which inflicts 1 \glsfmttext{hp} Damage per round. Swarms take up to 1~Damage per attack}
\index{Swarms}

This abstract spread of \glspl{hp} can represent a swarm of feral rats, a buzzing hive of blood-draining stirges, or any other tiny thing which can act as a unit.
We can't track individual bites and stings -- instead the swarm moves as one, but dies separately.

Swarms ignore Partial armour covering (as they can wriggle across the uncovered parts), but cannot attack anyone in Complete armour at all, unless they achieve a \gls{vitalShot}.

\swarm{Rats}{3}{2}{3}{0}

\input{config/rules/swarms.tex}

\end{multicols}
