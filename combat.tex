\chapter{Combat}
\index{Combat}
\label{combat}

\newcommand{\initiativechart}{

  \begin{nametable}[XX]{Initiative Costs}

  \textbf{Action} & \textbf{Init. Cost} \\\hline

  \hspace{3em}\textbf{Striking} & \\\hline

  Drawing weapon & 2 \\

  Guard Someone & 2 \\

  Heavy weapon & 8 \\

  Light weapon & 4 \\

  Medium weapon & 6 \\

  Ram & 3 \\\hline

  \hspace{3em}\textbf{Projectiles} & \\\hline

  Crossbow & 3 \\

  Improvised projectile & 7 \\

  Reloading & 2 \\

  Shortbow & 4 \\

  Thrown weapon & 4 \\\hline

  \hspace{3em}\textbf{\Glspl{quickaction}} & \\\hline

  Defence & 2 \\

  Keeping Edgy & 2 \\

  Moving & 2 \\

  Speaking & 2 \\\hline

  \hspace{3em}\textbf{Magic} & \\\hline

  Cast a spell & 3+level \\

  Use magic item & 8 \\

\end{nametable}

}

\startcontents[Combat]

\iftoggle{verbose}{
\begin{multicols}{2}

\noindent
These life and death rolls are handled somewhat differently from other tasks.
Let's start with an overview of the basic features, then cover the details later.

\begin{exampletext}
You enter a cavern.
You hear goblins sprinting towards you around the bend.

One of the players rolls Initiative for the group and gets `5'.
Your Initiative Factor is +2, so your Initiative will be 7.

The goblins are going at 12.

\paragraph{12:}
The goblins spend 2 Initiative to run round the corner.

\paragraph{10:}
The goblins stab at everyone at the front of the party.
Your companion rolls $2D6$ to defend at \gls{tn} 10, without any bonus, and receives a nasty spear-wound.

You decide to defend actively -- you have to spend 2 Initiative, but you can roll $2D6+3$ to defend, and the roll succeeds.

\paragraph{8:}
Your companion strikes back at a goblin, putting the goblin on Initiative 4.

\paragraph{6:}
The goblin in front of you strikes at you again. 
If you defend yourself, you'll lose 2 Initiative, and go down to 3, then attack once, and the round will end.

However, if you take this hit without dodging, you can spend 4 Initiative to hit a goblin with your shield, putting you on Initiative 1.
At Initiative 1, you could then spend 6 Initiative to attack with your sword.

You decide to trust to fate, the goblin hits you, for 5 Damage, and you mark off 5 \glspl{fp} -- the `plot armour' points which keep you alive.

\end{exampletext}

A successful fight depends as much on proper pacing and timing as anything else.
Each initiative click brings new decisions to the group with a kind of 1-dimensional chess.

\end{multicols}

}{}
\section{Basic Combat}

\begin{multicols}{2}

\subsection{Initiative}
\label{initiative}

At the start of each \gls{round} the leader of each group rolls $2D6$ and the result is the group's Initiative.\footnote{The ``party leader'', here means `whoever rolls the Initiative dice first'.}
Each character then adds their \textit{Initiative Factor} to get their Initiative Score.
The Initiative Factor is given by characters' Speed Attribute plus weapon modifiers.
If you roll 5 and have a Speed Bonus of 1, your Initiative Score is 6.

\iftoggle{verbose}{}{
  \begin{figure*}[t!]
  \footnotesize
  \initiativechart
  \end{figure*}
}

The \gls{gm} then counts downwards from the highest Initiative score.
When your number comes up, you can act.
Each time the character takes an action they pay a cost in Initiative -- once it reaches below 1 that character can no longer act.
Moving costs only 2 Initiative, while swinging an axe costs 6.
You can spend as much as you like, and even go down to an Initiative score of -5, but once the Initiative count reaches 0, the round ends.

Medium weapons are generally more effective than Light weapons, but they cost 6 Initiative points to take a swing, while Light weapons cost only 4.

Medium weapons are those with a \gls{weightrating} of -1 or greater.
Smaller weapons, those with a \gls{weightrating} of -2 or less, and brawling attacks with fists, all count as light weapons.

\iftoggle{verbose}{
  The total bonus to Initiative, including bonuses from weapons, is called the \textit{Initiative Factor}.
}{}

\subsubsection{\Glspl{quickaction}}
\label{quickaction}

\Glspl{quickaction} can interrupt the usual Initiative priorities.
Any time someone attempts a \gls{quickaction}, they take their action immediately, even if they have a negative Initiative score.
If two characters interrupt the Initiative flow with \gls{quickaction} then whoever currently has the highest Initiative Score goes first.

\gls{quickaction} allow characters to guard someone as soon as they see an attack impending upon a friend, to defend against missile attacks, or to shout a few words.

Characters on less than 1 Initiative can continue taking \gls{quickaction}, but suffer a -1 cumulative penalty for each \gls{quickaction} below.

For example, you can move, then Keep Edgy, even after you're too disoriented to attack anyone, but that movement will suffer a -1 penalty.
Meanwhile, Keeping Edgy requires no roll and has no associated numbers, so it does not incur any penalty.
However, defending oneself after this point would have a -2 penalty, and further \gls{quickaction} would suffer a -3 penalty.

\iftoggle{verbose}{
  \initiativechart
}{}

\subsection{Attack}
\label{attack}

To attack an opponent, you roll $2D6$ as usual, but only add your Combat \gls{skill}.
The \gls{tn} is 7 plus your opponent's Dexterity.

\iftoggle{verbose}{
  Your total bonus to attack (usually just your Combat Bonus), is known as the \textit{Strike Factor}.
}{}

\subsubsection{Aggression}
\label{aggression}
\index{Aggression} 

Animals use a \gls{skill} called Aggression.
It works exactly like the Combat \gls{skill} but only adds to the Strike Factor, and never to Initiative or Evasion.

\subsection{Damage}
\index{Damage}

\iftoggle{verbose}{
  \begin{figure*}[t]
    \begin{boxtext}[title=Dicey Damage]
  
      If you prefer your Dice in a more old-school format, you can easily give each weapon a different Damage die.
      Weapons which would normally inflict +1 Damage can instead roll their Damage as 1D8, while weapons with +2 Damage would instead leave players rolling 1D10, leaving weapons of +3 Damage to be replaced with a D12.
      
      Whether the players are rolling $1D6+1$ for a dagger or $1D8$, both have the same average of 4.5, so this system will not change things significantly.
      However, Stacking Damage occurs less often, and the die rolls will tend to swing more wildly to the highs and lows.
      
      If you don't own a D14, then simply add +1 Damage to all Damage totals above +3.
      
      +0 Damage should remain as $1D6$ and anyone with a Strength score of +4 should replace the bonus with a $D6$ as normal.
      Spells are unaffected.
  
    \end{boxtext}
  \end{figure*}

}{}

If you hit, roll $1D6$ plus your Strength Bonus to determine Damage.
The Damage is then taken off the enemy's \gls{hp}.
Everyone has a number of \gls{hp} to withstand Damage. When your opponent is reduced to 0 \gls{hp}, they are defeated.

\subsubsection{Stacking Damage}
\index{Combat!Stacking Damage}

Damage Bonuses cannot extend forever. If the Damage bonus ever exceeds +3 then 4 points of the bonus are replaced with a die. Therefore, what might usually be $1D6+4$ Damage becomes $2D6$ Damage.

This applies to all Damage, including magical Damage. It continues through all Damage Bonuses, so $1D6+9$ Damage would be simply $3D6+1$ Damage after conversion.

This also applies to lower Damage, so `2 Damage', would be $1D6-2$ damage.

\subsection{Defence}
\label{defence}
\index{Active Defence}
\index{Passive Defence}

When the enemy attempts to hit you, roll $2D6$ against \gls{tn} 8 plus your enemy's Strike Factor (this is generally their Combat score).
If you want to have an \gls{activeDefence}, you can spend 2 Initiative as a \gls{quickaction} to add your Dexterity Bonus, and any bonus from weapons.
Otherwise, a \gls{passiveDefence} means you just roll the dice, and hope for a good result.

Characters on 0 initiative or below suffer a -1 cumulative penalty to defence for each additional defence action.

Characters with a negative Evasion Factor must add it, even if they do not have an active defence.

The total bonus to defence, including any bonuses from weapons, is called the \textit{Evasion Factor}.

\iftoggle{verbose}{

  Passive defence is a perfectly valid tactic -- you can rely on armour and luck (i.e. \glspl{fp}) for a while if you don't want to spend your Initiative.

}{}

\subsection{Movement}
\index{Movement}
\label{movement}

By spending two Initiative, characters can run as a \gls{quickaction}, acting before all other actions.
Characters can run 3 squares plus their Speed Bonus during this time.
This movement can be chopped up into any number of pieces -- once the Initiative is spent, a character with Speed +1 might run only one square, then 2 more, then 1 more square later.

Characters who spend the entire turn running can move 10 squares plus their Speed Bonus plus their Athletics \gls{skill} Bonus; so someone with Speed +1 and Athletics +1 would move 12 squares per turn of flat-out running.

\subsection{Hit Points}

\index{Hit Points}
Each character has a number of \glsentryfullpl{hp} equal to 6 plus their Strength Bonus.
Small gnomes typically have 4 \glspl{hp} while big, strong humans typically have 7.
Losing even a single \gls{hp} means the character has suffered serious Damage.
A long fall might have broken the character's bone.
A dagger could have slashed open several veins.
Characters do not have many \glspl{hp} so losing even one is a serious matter.

\subsubsection{Healing}
\index{Healing}
Characters heal a quarter their \gls{hp} each week, rounded up.
Once someone receives a serious wound, it's a good time to call for \gls{downtime}.

\subsubsection{Vitality \& Death}
\index{Death}
Once a \gls{pc} reaches 0 \gls{hp} they must make a \index{Vitality Check}
Vitality Check in order to stay alive.
This is rolled at \gls{tn} 4 plus one for every negative \gls{hp} level.
\iftoggle{verbose}%
  {\footnote{Traits such as Strength do not affect the Vitality check because in a way, they already have.
  Stronger characters already have more \gls{hp}, which has already kept them farther from death.}
  For example, if someone with 3 \glspl{hp} left were to take a further 6 Damage, this would put them at -3 \glspl{hp}.
  That makes the \gls{tn} 7 for the Vitality Check.
}{}%

\glspl{npc} roll Vitality checks at a basic \gls{tn} of 7 instead of 4.

A failed Vitality check means that the character is dead.%
\iftoggle{verbose}{%
  \footnote{See page \pageref{pcdeath} on what to do once a \gls{pc} dies.}%
}{%
  The player then selects one of the \glspl{npc} introduced through spending \glspl{storypoint} to play.
  That second character begins with half the \glspl{xp} of whichever \gls{pc} in the group has accumulated the most total \glspl{xp}.
  The player taking control of the \gls{npc} should spend any additional experience this grants immediately.

  If no such \gls{npc} exists, one should be introduced through \glspl{storypoint} at the next available opportunity.
}%
A successful one means that the character is unconscious for the remainder of the scene but alive.
At the end of the scene they can make further Vitality Checks to see if they wake up.
When waking up, all actions relying on movement take a penalty equal to the number of \gls{hp} beyond 0 the character has lost.

\iftoggle{verbose}{
  At this point, the rest of the party will have to carry their fallen comrade back to safety -- if they can.
  Everyone's \gls{weightrating} equals their maximum \glspl{hp}, so a character with Strength +2 can carry someone with up to 7 \glspl{hp}, or drag someone with up to 12 \glspl{hp}.%
  \footnotesize{See page \pageref{weightrating} for \nameref{weightrating}.}
}{}

\end{multicols}

\section{Weapons}

\begin{multicols}{2}

\iftoggle{verbose}{%
\noindent
Weapons are a great way of inflicting additional Damage, but they are an equally excellent way of defending oneself. Having a longsword to keep scary opponents at bay is always better than trying to nimbly dodge about. Longer weapons also grant a bonus to Initiative, representing the fighter's ability to hit opponents before they hit them due to the weapon's length.
}{}

Each weapons is rated for `Dam' (the Damage bonus), `Init' (the bonus to Initiative, generally through reach) and `Ev' (the weapon's Evasion bonus).

Each weapon has a \gls{weightrating}, just like any item.
For every point a weapon's \gls{weightrating} exceeds its wielder's Strength Bonus, the wielder gains 1 Encumbrance, which subtracts from the character's Effective Speed as they move slower and swings the weapon slower.
Weapons held in only one hand add +2 to their \gls{weightrating}.

Finally, some weapons also have an in-built `knack' -- a special ability they allow the wielder to use.
These weapon knacks are not counted towards the character's total knacks, except for the purposes of the weapon's knack.
See Chapter \ref{knacks} for a full list of knacks.

\end{multicols}

\newcommand{\weaponschart}{
  \begin{boxtable}[p{.20\textwidth}p{0.07\textwidth}rrrrX]

  \textbf{Light Weapons} & \textbf{Dam.} & \textbf{Init.} & \textbf{Ev.} & \textbf{Wt.R} & Cost & \textbf{Knacks} \\\hline

    Cudgel & +2 & \ 0 & \ 0 & -3 & & Stunning Strike (page~\pageref{stunningstrike}) \\

  Dagger & +1 & \ 0 & +1 & -4 & 60cp &  \\

  Firepoker & +1 & +1 & \ 0 & -2 & & Finishing Blow (page~\pageref{finishingblow}) \\

  Javelin & +1 & +2 & \ 0 & -2 & 50cp & \\

  Knife & +1 & 0 & \ 0 & -4 & 40cp & Precise Strike (page~\pageref{precisestrike}) \\

  Log & +1 & -1 & \ 0 & -2 & & \\

  Rapier & +1 & +2 & +1 & -2 & 15sp & \\

  Rock & +1 & \ 0 & \ 0 & -5 & & \\

  Stick & +1 & +1 & +1 & -2 & & \\

  \end{boxtable}

  \begin{boxtable}[p{.20\textwidth}p{0.07\textwidth}rrrrX]

  \textbf{Medium Weapons} & \textbf{Dam.} & \textbf{Init.} & \textbf{Ev.} & \textbf{Wt.R} & Cost & \textbf{Knacks} \\\hline

  Boulder & +4 & -1 & \ 0 & 6/8 & & Finishing Blow (page~\pageref{finishingblow}) \\

  Cast Iron Skillet & +2 & \ -1 & \ +1 & -1/1 & & Adrenaline Surge (page~\pageref{adrenalinesurge}) \\

  Chair & +1 & +1 & +1 & 1/ 3 & \\

  Club & +2 & +1 & +1 & 2/4 &  \\

  Great Axe & +3 & +1 & +1 & 3/5 & 8 sp & \\

  Great Sword & +2 & +1 & +2 & 3/5 & 8 sp & \\

  Maul & +3 & \ 0 & \ 0 & 4/6 & 1 sp & \\

  Large Rock & +2 & \ 0 & \ 0 & 4/6 & & \\

  Longsword & +1 & +1 & +3 & 1/3 & 9 sp & \\

  Shortsword & +1 & +1 & +2 & -1/1 & 6 sp & Furious Blows (page~\pageref{furiousblows}) \\

  Spear & +1 & +1 & +2 & 0/2 & 3 sp & First Strike (page~\pageref{firststrike}) \\

  Quarterstaff & \ 0 & +1 & +2 & 0/2 & 2 sp & First Strike (page~\pageref{firststrike}) \\

  Warhammer & +3 & 0 & +1 & 3/5 & 7 sp & Finishing Blow (page~\pageref{finishingblow}) \\

  Whip & \ 0 & +2 & \ 0 & -1/ 1 & 1 cp & First Strike (page~\pageref{firststrike}) \\

  Wood Axe & +2 & \ 0 & +1 & -1/1 & 1 sp & \\

  \end{boxtable}

  \begin{boxtable}[p{.20\textwidth}p{0.07\textwidth}rrrrX]

  \textbf{Heavy Weapons} & \textbf{Dam.} & \textbf{Init.} & \textbf{Ev.} & \textbf{Wt.R} & Cost & \textbf{Knacks} \\\hline

  Great Club & +4 & +1 & +1 & 5 & & \\

  Giant Boulder & +5 & 0 & \ -2 & 8 & & Finishing Blow (page~\pageref{finishingblow}) \\

  Giant Sword & +3 & +1 & +2 & 5 & 15 sp &  \\

  Poleax & +3 & +1 & +1 & 5 & 6 sp & First Strike (page~\pageref{firststrike}) \\

  \end{boxtable}

  \begin{boxtable}[p{.20\textwidth}p{0.07\textwidth}rrrrX]

  \textbf{Shields} & \textbf{Dam.} & \textbf{Init.} & \textbf{Ev.} & \textbf{Wt.R} & Cost & \textbf{Knacks} \\\hline

  Bucklar Shield & 0 & \ 0 & +2 & -2 & 4 sp & \\

  Kite Shield & 0 & \ 0 & +3 & 2/4 & 8 sp & Solid Defence (page~\pageref{soliddefence}), Dodger (page~\pageref{dodger}) \\

  Round Shield & +1 & \ 0 & +2 & 0/2 & 5 sp & Dodger (page~\pageref{dodger}) \\

\end{boxtable}}

\iftoggle{verbose}{%
  \weaponschart
}{
  \begin{footnotesize}
  \weaponschart
  \end{footnotesize}
}

\label{weaponschart}
\index{Weapons}

\begin{multicols}{2}

\subsubsection{Light Weapons}
\index{Light Weapons}

Light Weapons are those with a \gls{weightrating} of -2 or less. People wield them in one hand only, without problem, and can slash or stab with them in flurries of blows, quickly. They require only 4 Initiative points to attack with, so while an axe is far more damaging than a dagger, a dagger can unleash a flurry of blows before a single axe swing has taken place.

\subsubsection{Medium Weapons}
\index{Combat!Medium Weapons}

Swords, axes and all the regular weapons of warfare require a full 6 Initiative points to be swung.
They grant excellent Combat Bonuses, often increasing the effects of all three Attributes.
These weapons are the standard weapons which most people will be using throughout the campaign -- they cover the \gls{weightrating} from -1 to 4.

Medium weapons are usually wielded in both hands.
However, characters can try to hold one with only one hand, but the weapon's \gls{weightrating} increases by 2.
For example, a great sword can certainly be held up by one hand alone, but it will move from a \gls{weightrating} of 4 to 6, meaning that a normal human, with Strength +1, would suffer a -5 penalty to their Speed Bonus, and therefore Initiative.
While this is a steep penalty to Initiative, the price can be worth the wielding of a shield with a weapon.

Anyone wielding a medium (or indeed heavy) weapon with a \gls{weightrating} equal or greater than their racial maximum has an unwieldy weapon indeed, and suffers a -3 penalty to their Initiative.

\subsubsection{Heavy Weapons}
\index{Combat!Heavy Weapons}

Giants, monsters and a few extremely strong humans have the ability to heft weapons so large that they can only be used with both hands together -- all have a \gls{weightrating} of 4 or more.
They grant excellent Bonuses, but require 8 Initiative points to attack.

Anyone insane enough to attempt to use a large weapon one handed must suffer through a +4 increase in the weapon's \gls{weightrating}, which would make such weapons prohibitively heavy for most people.

\subsection{Shields}
\index{Shields}

Shields work like any other weapon, so they are useful both for attack and defence.
However, they work best in defence.

\subsection{Dual Wielding}
\index{Dual Wielding}
\label{dualWielding}

When you have two weapons, you must select one as the primary weapon.
On the first round, the primary weapon must be the first weapon you use, but after that, you can attack with either.

The secondary weapon adds half its Evasion Bonus, rounded up.

\end{multicols}

\newcommand{\armourchart}{

  \begin{boxtable}[ccccX]

  \textbf{Armour} & \textbf{\glsentrytext{dr}} & \textbf{Weight} & \textbf{Noise} & \textbf{Price} \\\hline

  \textbf{Partial} \\\hline

  Elvish & 2 & -2 & 0 & 3gp \\

  Padded & 2 & 0 & 0 & 1sp \\

  Leather & 3 & 0 & 0 & 5sp \\

  Chain & 4 &  1 & 1 & 10sp \\

  Plate & 5 &  2 & 4 & 12gp \\
  \hline

  \textbf{Complete} \\\hline

  Elvish & 2 & -1  & 0 & 9gp \\

  Padded & 2 & 1  & 0 & 3sp \\

  Leather & 3 & 1  & 1 & 15sp \\

  Chain & 4 &  2  & 2 & 30sp \\

  Plate & 5 &  3  & 5 & 36gp \\

  \end{boxtable}
}

\section{Armour}
\index{Armour}

\begin{multicols}{2}

\iftoggle{verbose}{%
  \armourchart
}{
  \begin{figure*}[t!]
  \footnotesize
  \armourchart
  \end{figure*}
}

\noindent
Armour defends characters by lowering incoming Damage. In game terms, armours have a \gls{dr} rating which subtracts from Damage.

Armour can cover more or less of a character, and therefore comes with three ratings -- Partial, Complete and very rare Perfect armour.

\paragraph{Partial armour}
covers the basics -- the character's chest and probably head, perhaps a basic arm-guard on top of that.

\paragraph{Complete armour}
covers the full character -- almost.
Complete armour, whether leather or plate, will come with a helmet, a neck-guard, gauntlets, shin guards, foot coverings and will overlap to protect the joints.

Complete armour adds +1 to the \gls{weightrating} and multiplies the price by 3.

\paragraph{Perfect armour}
is a rating used for certain creatures which have natural armour without weak spots (such as stone giants).

\subsection{Vitals Shots}
\label{vitals}
\index{Combat!Vitals Shots}

When attacking an opponent in armour, it is possible to make a shot so precise as to get a gap in a helmet, strike an opponent in the eye or slide a blade between overlapping plates. To get a Vitals Shot, one simply needs to roll high enough over the creature's regular \gls{tn} and all armour (meaning \gls{dr}) can be ignored.

For partial armour, anyone rolling a Margin of 3 (i.e. 3 points above the \gls{tn}) ignores the \gls{dr} from the armour. If the regular \gls{tn} is 8 then any roll of 11 or greater counts as a Vitals Shot. Complete armour requires a Margin of 5 to ignore the armour, so if the \gls{tn} were 10 then a Strike would require a total of 15 to bypass the armour. Perfect armour cannot be bypassed by a sufficiently high roll.

\iftoggle{verbose}{%
\pic{Roch_Hercka/vitals_shot}{\label{roch:vitals}}
}{}

Many creatures have a \gls{dr} from natural armour, representing especially thick skin or some other immunity to Damage. Natural armour always counts as Complete armour unless otherwise specified, because it covers almost all of the body, but often leaves weak spots open such as the eyes or the kneecaps.

\iftoggle{verbose}{

Vitals Shots provide an incentive for people to push their Strike Factor as high as possible, even at the expense of their own defence.
Various knacks and manoeuvres, such as charge (see page \pageref{charge}), can change the various combat factors.

}{}

\subsection{Stacking Armour}
\label{stackingarmour}

Some creatures have a natural \gls{dr}, which would then stack with their armour.
The primary armour counts for its full value, and the lower \gls{dr} score counts for half.
Any tertiary armour counts for a quarter, and so on.
Once you have a total, round up anything over half.
Stacked armour can consist of both partial and complete layers, meaning a roll could bypass one set of armour by rolling 3 over the creature's \gls{tn}, but bypass all armour with a roll of 5 over the \gls{tn}.
\iftoggle{verbose}{

  For example, a basilisk with \gls{dr} 4 might die, and then get raised from the dead by a necromancer.
  The undead naturally have a \gls{dr} of 2, so this secondary source of damage would count for half, giving it a total \gls{dr} of 5.
  If the mage were crazy enough to add plate armour to the basilisk, the total \gls{dr} would be $5 + \frac{4}{2} + \frac{2}{4} = 7.5$, or `8'.

  Of course if this were \textit{partial} plate armour, any roll which gets 3 over the basilisk's \gls{tn} would only get the \gls{dr} of 5.
}{}

Standard armour cannot be stacked in this way.
We assume plate, chain, and some leather-based armours already have padded armour underneath.
Similarly, different types of natural \gls{dr} do not stack, and nobody becomes undead in different ways.

\subsection{Weight}
\index{Weight}

All armour has a \gls{weightrating}, just like any other item.
The \gls{weightrating} above are for Partial Armour.
If anyone wears Complete armour the Weight is increased by 1, so Complete chain armour which comes past the knees, has a helmet and uses arm-guards, would have a \gls{weightrating} of 2.

Armour also inflicts \glspl{fatigue} quickly, as mentioned above.
Wearing armour in battle is a great idea, but characters attempting to sprint in full plate will find themselves unable to run before long.

\subsubsection{Stacking}

Adding extra weight works with same as adding armour -- just count the heaviest item, and half of the second, a quarter of the next, and so on.
Lifting a gnome with a \gls{weightrating} of 4, and a club with a \gls{weightrating} of 3 would have a total \gls{weightrating} of 6.

\subsection{Noise}

\iftoggle{verbose}{%
  Some armours make noise when walking around, which alerts everyone around, and typically stops sneaking attempts.
  The `Noise' rating shows the added difficulty to any Stealth rolls which involve moving silently.
  While sufficient padding can make Partial Armour silent, no amount of padding can remove the penalty from Complete.
}{
  The `Noise' rating shows the penalty to moving silently while wearing Complete armour of this type.
}

\subsection{Perfect Strikes}

\index{Combat!Perfect Strikes}
Rolling a \gls{natural} `12' in combat, i.e. rolling two 6's, means the roll was a Perfect Strike.
A Perfect Strike is guaranteed to hit even if it doesn't reach the opponent's \gls{tn}, it ignores both Partial and Complete armour (covered below) and it grants +2 Damage.

\end{multicols}

\section{\glsentrylongpl{fp}}\label{fate_points}\index{Fate Points}

\begin{multicols}{2}

\noindent
\iftoggle{verbose}{%
At this point you might be wondering how anyone is going to survive past their first battle.
6 or 7 \glspl{hp} is not a lot when the Damage is often $2D6$ or higher.
The mechanism which saves the plot-important character is \glsentryfullpl{fp}.
Every time someone would lose \glspl{hp}, the character marks off \gls{fp} instead and it is stipulated that the attack in fact misses, because the gods have fated this person to live another day.
}{
  If a character would lose \glspl{hp}, they can mark off \glspl{fp} instead.
}

Everyone in the world begins with 5 base \gls{fp}.
This is then modified by their Charisma Bonus, so someone with Charisma -2 starts with 3 \gls{fp}.
The difference between the \glspl{pc} and the \glspl{npc} is that \glspl{pc} start play with a full allotment of \gls{fp} at the beginning of each adventure.
\Glspl{npc} start with none, but regain \gls{fp} at the end of each scene as usual.
As a result, most \glspl{npc} effectively have 0 \gls{fp}.
The \gls{gm} can mostly ignore \gls{npc} \gls{fp} and Damage will be applied directly to \gls{npc} \glspl{hp}.

\subsection{Regaining \glsentrylongpl{fp}}

\newcommand{\FPRegen}{

  \begin{xpbox}{C}
    Base \glsentrytext{fp} & Regeneration \\\hline

    5 & 2 per scene \\

    10 & 4 per scene \\

    15 & 6 per scene \\

    20 & 8 per scene \\
  \end{xpbox}
}

\sidebox[25]{
  \FPRegen
}

At the end of each Scene, players regenerate 2/5ths of their \glspl{fp}.
Those with 5 \glspl{fp} total regenerate 2 temporary \glspl{fp}, and those with 10 \glspl{fp} regenerate 4 temporary \glspl{fp}, and so on.

While \glspl{npc} begin with 0 \gls{fp}, they too regenerate the normal amount each scene.
In this way, an \gls{npc} might accumulate quite a number of \gls{fp}, and when some climactic end scene arises where the \glspl{pc} finally confront them, they will have a harder time of it, because the \gls{npc} has now become plot-important enough to merit some plot immunity, just like them.

One exception here is creatures without a Charisma Attribute.
Animals, undead and other creatures without any Charisma Bonus can never store \gls{fp} except through the use of Magic.

\iftoggle{verbose}{
  \begin{exampletext}

The next morning the trio gave a wide berth to the area between the fallen village and the mountain, in the hopes of avoiding attack from the rear.
Unfortunately there was little they could do to hide themselves, and a band of hobgoblins from the village were following them.
The trio had reached halfway up the mountain by this point but the hobgoblins were faster than them, and stronger.
Sean thought for a moment about abandoning Hugi if there was a problem - his little dwarvish legs were no good for sprinting.
Of course Hugi's death would not buy them much time, and Arneson would never stand for it.
No, they would have to stick together to survive.

The trio climbed for a while longer, looking back every few moments to note how close the hobgoblins were behind them. An ogre was among their ranks, so they must have come from a very deep cavern.

Looking back, the enemy was nearing and everyone was out of breath.
Arneson suggested a rest to make sure they would be ready for the fight -- they could fight downhill against an enemy fatigued from walking upwards.
He calmly got out the rations -- some cheese, smoked pork, oatcakes and a flagon of wine.

\pic{Boris_Pecikozic/nura_brawl}{\label{boris:brawl}}

``May as well have the best of the rations now, eh, friends?'', Arneson said while smiling, and they slowly masticated their age-hardened meal and tried to smile back as the nine foot monstrosity which was so recently a man made its way up to them, pounding its great feet up the mountainous slopes, surrounded by half a dozen hobgoblins, each the size of a broad-shouldered man.

As the hobgoblins neared the plateau where the trio sat they began to make their war cries, but Arneson just sits and ate his last oatcake slowly. They began to sprint upwards across the rocky ground.

\end{exampletext}

\sideBySide{
  The \gls{gm} decides that since the players have the higher ground, they will receive +1 to all rolls until the hobgoblins can move up to where the \glspl{pc} stand.

  The \glspl{pc} have no need to fight, so they don't need to spend \glspl{ap} on movement, so they wait, while the hobgoblins have to spend their own \glspl{ap} on movement.
}{
  The hobgoblins scramble up the rocks desperately.
  They look too hungry to cook you before eating you, or even make sure you're all dead.
}

\sideBySide{
  Hugi's player rolls to fire at a hobgoblins.
  $$\gls{tn} 6 - Dexterity (1) - Projectiles (1) = 4$$
  The Damage is

  $$ 1D6 + 3 = 8$$
}{
  Hugi has been winding up his crossbow with malice while the others ate.
  He aims the bolt, but waits patiently until they come within a good range -- every enemy matters when you're outnumbered.

  He lets loose, hitting one straight in the head.
  The others don't seem scared -- just relieved they didn't lose their chance to feast.
}

\sideBySide{
  Arneson's player declares an attack first, and goes for the ogre.
  The ogre increases the \gls{tn} of 8 with his stats:
  $$Dexterity (0) + Combat (1) = 9$$
  Arneson reduces the \gls{tn} with
  $$Dexterity (0) + Combat (1) + Longsword (2) = 3$$
  At \gls{tn} 6, he hits, and spends 2 \gls{ap} for the attack.
}{
  As the enemy ascend to striking range, the ogres claws at Arneson, and misses as Arneson steps to the side.
  The step allows an oppening on the ogre's anterior side, and rushing in the sword enters the massive shoulder.
}

\sideBySide{
  Arneson's Damage total is:
  $$1D6 + Strength (2) + Longsword (2) = 1D6+4 = 2D6 = 4$$
}{
  The ogre shrieks in pain as Arneson's sword sticks in, and pulls away, eying up an easier breakfast.
}

\sideBySide{
  Hugi's player defends against the ogre's attack.
  $$\gls{tn} 8 + ogre (1) - Hugi (2) = \gls{tn} 7$$
  He rolls a `5', which misses.
  The ogre spends 1 \gls{ap} to move back, and Sean spends 1 to follow.
}{
  The ogre's eyes land on Hugi, so he grabs the dwarf by his beard and yanks him back while stepping back, behind the hobgoblins, who move in for the attack, leaving Arneson surrounded.

  Sean shouts after his companion, circles around the dangerous side of a hobgoblin, and jumps towards the ogre to save his companion.
}

\sideBySide{
  Sean spends his last 2 \glspl{ap} to attack with his sword.
  The ogre has grabbed Hugi, so he counts as \textit{Prone}, giving a +2 Bonus to attacks against him.
  $$\gls{tn} 8 - Prone (2) - Combat (1) = \gls{tn} 5$$
}{
  Sean lands on the ogre sword-first, the ogre writhes and its deformed rib-bones crack as the blade twists to the side.
  Hugi begins to crawl out from underneath.
}

\sideBySide{
  Arneson's player rolls for 3 Attacks, and spends \glspl{fp} to mitigate most of the damage, but marks off 2 \glspl{hp}.
  The third attack reduces Arneson's \glspl{ap} to -2, giving him a -2 penalty.

  \vspace{1em}
  \begin{tabular}{l|cc}
  Event & Areson's \glspl{ap} & and \glspl{fp} \\\hline
  Hobgoblin attacks & 2 & 10 \\
  Hobgoblin attacks & 0 & 6 \\
  Hobgoblin attacks & -2 & 0 \\
  \end{tabular}

}{
  Three hobgoblins had rushed to Arnesons front, and now pull their maces up to bring down upon him.
  He thinks he can take one, but doesn't know about the other.

  The first, he stabs, and it falls back bleeding.
  The second swings for his head, but intersects with the rocky ground as he slips to the side.

  He wobbles, confused, as the last stabs him the side with a dagger.
}

}{}

\end{multicols}

\section{Fatigue}

\newcommand{\fatiguechart}{

  \begin{nametable}{\gls{fatigue} Chart}
    \textbf{Action} & \textbf{\Glspl{fatigue}} \\\hline

    Armour & Wearing armour inflicts 1 \gls{fatigue} per \gls{weightrating} of the armour. \\

    Bleeding & 1 \gls{fatigue} per slashing damage which was not mitigated by armour. \\

    Climbing & 1 \gls{fatigue} per square. \\

    Fighting & Each round inflicts 1 \gls{fatigue}. \\

    Holding Breath & 1 \gls{fatigue} per round. \\

    Marching & 1 \gls{fatigue} per mile. \index{Marching}\\

    Starving & Each meal skipped inflicts 1 \gls{fatigue} plus half the character's Strength Bonus (rounded up). \\

    Swimming & Each square swum inflicts 1 \gls{fatigue}. \\
  \end{nametable}

}

\begin{multicols}{2}

\label{fatigue}
\index{Fatigue}
\iftoggle{verbose}{%
\noindent
  Fighting, running and swimming can really take it out of you, especially when wearing heavy armour.
  Characters gain \glspl{fatigue} for exerting themselves, and if they accrue too many then they will quickly start to become ineffective.

  \begin{boxtable}[lllllllllX]

  \multicolumn{10}{l}{\Glsentrytext{hp}} \\
  \CIRCLE & \CIRCLE & \CIRCLE & \CIRCLE & \CIRCLE & \CIRCLE & \Circle & \Circle & \Circle & \Circle \\
  \Square & \Square & \Square & \Square & \Square & \Square & \Square & \Square & \Square & \Square \\
  \multicolumn{10}{l}{\glspl{fatigue}} \\
  \XBox & \XBox & \Square & \Square & \Square & \Square & \Square & \Square & \Square & \Square \\
  \multicolumn{10}{l}{Penalty: 0} \\
  
\end{boxtable}

Below the character's \gls{hp} bar are spaces for \glspl{fatigue} to be gained.
Once the character has more \glspl{fatigue} than their current \glspl{hp}, they take a -1 penalty for every \gls{fatigue} in excess of their \glspl{hp}.


  \begin{boxtable}[lllllllllX]

  \multicolumn{10}{l}{\Glsentrytext{hp}} \\
  \CIRCLE & \CIRCLE & \CIRCLE & \CIRCLE & \CIRCLE & \CIRCLE & \Circle & \Circle & \Circle & \Circle \\
  \Square & \Square & \Square & \XBox & \XBox & \XBox & \Square & \Square & \Square & \Square \\
  \multicolumn{10}{l}{\glspl{fatigue}} \\
  \XBox & \XBox & \XBox & \XBox & \Square & \Square & \Square & \Square & \Square & \Square \\
  \multicolumn{10}{l}{Penalty: 1} \\
  
\end{boxtable}

This might happen because the character has, say, 6 \glspl{hp} but gains a total of 8 \glspl{fatigue}, and then gains a -2 penalty to all actions.
But it might also occur because the character has 4 \glspl{fatigue} and then Damage reduces them to only 2 \glspl{hp}, leaving them with a -2 penalty to all actions yet again.


  \begin{boxtable}[lllllllllX]

  \multicolumn{10}{l}{\Glsentrytext{hp}} \\
  \CIRCLE & \CIRCLE & \CIRCLE & \CIRCLE & \CIRCLE & \CIRCLE & \Circle & \Circle & \Circle & \Circle \\
  \Square & \Square & \XBox & \XBox & \XBox & \XBox & \Square & \Square & \Square & \Square \\
  \multicolumn{10}{l}{\glspl{fatigue}} \\
  \XBox & \XBox & \XBox & \XBox & \XBox & \Square & \Square & \Square & \Square & \Square \\
  \multicolumn{10}{l}{Penalty: 3} \\
  
\end{boxtable}

Characters may reach a maximum penalty of -5 due to \glspl{fatigue}, after which they fall unconscious.
If the character is accruing \glspl{fatigue} from running or wrestling, they would normally simply pass out at this point, but if they are gaining \glspl{fatigue} swimming or bleeding, the character will almost certainly just die.

\Glspl{fatigue} cannot be mitigated with \gls{fp}. Characters who can luck their way out of being shot by arrows and roasted by dragons can quite easily be punched and dragged away, or collapse after a long run.

\subsection{Gaining Fatigue}
}{}

\noindent
Each round running, climbing, in combat, or otherwise exerting oneself inflicts a \gls{fatigue}.
Armour also inflicts a number of \glspl{fatigue} equal to its \gls{weightrating} at the end of each scene.

\iftoggle{verbose}{%

  \Glspl{fatigue} gained extremely quickly, for all manner of reasons.
  However, it is only applied at the end of the scene.
  Running, fighting, and jumping generate a lot of adrenaline, which keeps any tiredness at bay while the action is on.
  The real danger in \glspl{fatigue} is persistent action, when characters have no chance to recover from a previous battle.

}{
  \Glspl{fatigue} can only be gained at the end of a scene.
}

\subsubsection{The Skill Discount}

Characters can use skills as a sort of `\gls{dr}' against \glspl{fatigue}.
3 \glspl{round} of combat inflicts 3 \glspl{fatigue}, but someone with Combat 1 can ignore 1 \gls{fatigue} which comes from fighting in the first round of combat.%
\footnote{Skills never help \glspl{fatigue} gained due to carrying heavy items.}
Athletics curbs \glspl{fatigue} accumulated through running, Wyldcrafting or Caving curbs \glspl{fatigue} gained through marching (depending upon the environment), and so on.

\subsubsection{Special Categories}

\Glspl{fatigue} can represent all manner of problems a character has -- not just tiredness.

\paragraph{Bleeding} occurs when a character has lost \glspl{hp} to piercing or slashing weapons.
They then gain \glspl{fatigue} equal to the number of \glspl{hp} lost.
These \glspl{fatigue} are marked with a `$B$' instead of the usual dash across a box and are healed at a rate of one per day rather than the usual, faster rate.
If the bleeding is not stopped, the character should bleed for the same number of points minus one on the next scene until they are dead or the bleeding has stopped on its own.
The \gls{tn} to stop the bleeding is always 6 plus the number of \glspl{fatigue} being lost on the current scene.

\paragraph{Poison} can become a nasty drag on a character, and a serious poisoning can prompt even the strongest fighter to return home.

\paragraph{Starvation} is another special case.
\glspl{fatigue} inflicted from starvation are marked with an `$S$', and each of these points only heal once the character has had a full meal.

\iftoggle{verbose}{%
  \fatiguechart
}{
  \begin{figure*}[t!]
  \footnotesize
  \fatiguechart
  \end{figure*}
}

\subsection{Healing Fatigue}
\index{Resting}

When the party take any part of the day to rest, they can heal a number of \glspl{fatigue} equal to half their \textit{current} \glspl{hp}; so someone with 4 out of 8 \glspl{hp} would be able to recoup 2 \glspl{fatigue} by resting, either for a full night, or by taking some chunk of the afternoon to sit quietly.%
\footnote{The day is divided into four parts. See page \pageref{time}.}

\iftoggle{verbose}{

  In most cases, \glspl{fatigue} will heal faster than they accumulate, so tiredness can be safely ignored while are in ideal circumstances.
  However, persistent battles, sprints, and poisons can quickly incapacitate the most seasoned warriors.


}{}

\end{multicols}

\startcontents[Manoeuvres]

\section{Complications \& Manoeuvres}

\begin{multicols}{2}

\subsection{Complications}

\subsubsection[Blindness: -8 penalty + Wits + Vigilance (maximum -6). Rolling equal to allies means hitting an ally]{Blindness}
\index{Combat!Blindness}

Fighting while blind is no fun -- your opponent can see you coming, and you can't see them.
Blinded opponents suffer a penalty equal to -8 plus their Wits and Vigilance Bonuses with a maximum penalty of -6.
\iftoggle{verbose}{%
For example, a character with with a Wits + Vigilance total of -1 would receive a -9 penalty to attack, except that the maximum penalty is -6.
Someone with Wits +1 and Vigilance +3 \ would suffer a -4 penalty to attack because both reduce the basic penalty of -8.
}{}

This penalty only counts when one side of a fight is blind. When both sides are blind, we use the Darkness Fighting rules below.

While fighting blind, if the dice make a \gls{natural} roll equal to the number of people on the character's side side (including themself) then they hit a companion.
If the character is fighting with just one companion then there are two of them and they hit a companion on the roll of a 2.
If they are part of a group of 5 people, any roll of 5 or under means they have accidentally hit a companion.
Companions who are are accidentally hit can attempt an Evasion roll by rolling with their current Evasion Factor against \gls{tn} 10; failure implies normal Damage from that attack.
It is quite possible to kill a companion while fighting blind.

\subsubsection[Darkness: Penalty equals difference between combatants' Wits + Vigilance]{Darkness}
\label{darkness}
\index{Darkness}
\index{Combat!Darkness}

Fighting in the darkness, or just twilight, can give a distinct advantage to those with sharper senses.
Those who retain some basic vision while their opponents have none are in a similar situation to fighting a blinded opponent.
However, when both sides suffer from the darkness, the battle changes very little.
Neither side can hit very accurately, but then neither side can dodge or parry very well either.

When fighting in the dark, each side receives a penalty to attacking the other equal to the difference between their respective Wits + Vigilance totals, up to a maximum of -6.

For example, a human guard has caught a room full of elves with stolen goods. Thinking quickly, one of the elves douses the room's only lantern. The human has a Wits Bonus of -1 and no Vigilance Skill. The elves have a minimum Wits of +1 and many also have the Vigilance Skill; that means the elves will receive a +2 bonus to striking the guard and those with the Vigilance Skill will receive a higher bonus.

Deep darkness can provide a maximum penalty of -6, while twilight is limited to a penalty of -3.

\subsubsection[Enclosed Spaces: Penalty equals difference between Initiative cost to attack and Enclosure Rating]{Enclosed Spaces}
\label{enclosedcombat}
\index{Enclosed Spaces}
\index{Combat!Enclosure Rating}

Enclosed spaces cause serious problems for people wielding longswords, battle axes, and other large weapons.
Daggers and rapiers often have an easier time in these locations.
Each location has an \gls{enclosurerating}; the smaller the number, the more narrow the space.

The amount of space required for a weapon is determined by the Initiative \textit{the character} spends to wield it.
Small hallways may have a maximum initiative of 5, meaning someone could wield a shortsword without penalty, but a longsword, spear, or kite shield would have problems, because they all require 6 Initiative to attack with.
For every initiative point that a weapon is over the maximum room space, the character gains a -1 penalty to attack.
Characters with Knacks like \textit{Flashing Blades} are better at getting in short, sharp, thrusts, so they suffer less of a penalty to attack.

The penalty to attack counts for all other actions, such as spell casting.
Higher level spells, which require a lot of Initiative points, also require space to move and create the grand gestures which bring forth the magic.
If the enclosure rating goes down to 3, then someone casting a 1st level spell, at 4 Initiative, would gain a -1 penalty to casting, while someone casting a 3rd level spell at 6 Initiative would gain a -3 penalty to casting.

\subsubsection[Passing Attacks: When passing someone, they can make a normal attack as a Quick Action]{Passing Attacks}\index{Combat!Passing Attacks}

If you try to run past an opponent during combat, they may make an attack against you as a \gls{quickaction}.

This might happen when someone is surrounded, but wants to run away.

\subsubsection[Spell Casting: take a -2 penalty to the spell.]{Switching Actions}
\index{Combat!Spell Casting}
\label{combatcasting}

Someone casting spells and swinging their swords cannot use an Initiative bonus of one for the other.
\iftoggle{verbose}{%
  For example, if Toria has Speed +2 and a sword, her total Initiative Factor would be +3 when attacking with the sword.
  The group rolls a `4' for Initiative, so at Initiative step 7 she decides to cast a \textit{Fast} spell.
  However, her +1 Wits means she cannot cast a spell until Initiative step 5.
}{}

Simply put, every character calculates their Initiative based on the action they're doing, and must have at least that amount of Initiative to proceed.

\subsubsection[Trapped/ Entangled: All attacks against the character count as a Sneak Attack, but they can still defend with full Dexterity Bonus as usual]{Trapped or Entangled}

Characters caught in mud, who slip over, or get shackled to a spot cannot move or dodge nearly as well as they could.
All attacks against them count as Sneak Attacks and they can no longer use the Knack: Fox Hop.
Despite the Sneak Attack Bonus, such characters can defend as normal, with their full Dexterity Bonus, and any weapon bonuses.

\subsubsection[Falling Prone: Same as `Trapped', but characters can spend a movement action to get up]{Falling Prone}
\index{Prone}
\label{prone}

Characters who fall over lose their ability to defend themselves, as above.
However, they can get up at the cost of 2 Initiative by using up their movement action.
If they've already moved this \gls{round}, they have to wait until the next \gls{round}.

\subsection{Manoeuvres}

\subsubsection[Brawling: Make a normal attack roll, but any attack with a Margin less than 5 only inflicts \glspl{fatigue} rather than Damage]{Brawling}
\index{Combat!Brawling}
\index{Brawling}

Punches and kicks all use the Combat bonus. Such attacks inflict \glspl{fatigue}. Everyone gains a \gls{dr} against Brawling Damage equal to their Strength Bonus, which stacks with armour (\gls{dr} cannot be negative). This counts as Complete armour, so hitting someone in Partial chainmail with a \gls{tn} of 8 and a Strength of +1 would mean they have a total \gls{dr} of 6. However, an attack score of 11 would mean that the Partial armour's \gls{dr} could be ignored, leaving only a \gls{dr} of 1. An attack score of 13 would ignore both types of \gls{dr}, leaving nothing at all. Attacks which bypass a body's natural armour count as normal Damage as such attacks might hit vulnerable locations such as the eyes or crotch or twist an opponent's arm till breaking point.

\subsubsection[Blind Rage: You can mitigate an enemy's weapon bonus to Evasion, but they can make an attack against you as a Quick Action]{Blind Rage}\label{blindrage}

\index{Blind Rage}Weapons can grant a bonus to the wielder's Evasion Factor because the wielder is keeping people at bay with it -- a spear might be waved in an opponent's face in a threatening manner or a sword might be on the ready to attack if someone gets within its range.
However, this marvellous defence only works against people who care about being hit.
Anyone can choose to attack someone while ignoring their opponent's weapon's bonus to Evasion; the penalty is simply that the opponent can choose to make a single Sneak Attack immediately.

\subsubsection[Charge: Swap your Strike and Evasion. Cost: 0 Initiative]{Charge}
\label{charge}
\index{Charge}

The character smashes into opponents recklessly, foregoing most of their usual defence.
The character's Strike and Evasion factors swap place until their next standard action (Quick actions do not count).

\iftoggle{verbose}{
  A character with Combat +2, Dexterity +1 and a longsword would normally have a Strike of +2, and an Evasion of +4 (because the longsword adds +3 to the Evasion).
  However, while using \textit{Fast Charge} manoeuvre, the character's Strike would be +4, and their Evasion +2.

This manoeuvre can be extremely effective at penetrating an enemy's defences, but also dangerous, as one's defences are lowered.

The charge manoeuvre does not require movement -- it can be used to attack enemies standing right next to the character.

}{}

\subsubsection[Drawing Weapon: Cost: 2 Initiative]{Drawing Weapons}

\index{Combat!Drawing Weapons}Drawing a weapon costs 2 Initiative if it is placed in an easy place to draw, like a scabbard on the side of a belt. If a character holds weapons on the back or in a bag, it costs 8 Initiative to remove them. If a knife's stuffed inside a pack, the \gls{gm} may stipulate a number of \glspl{round} required to draw the weapon.

\subsubsection[Dropping Weapon: Cost: 0 Initiative]{Dropping Weapons}

Dropping a weapon is free, but if the character has not made an attack then the weapon's Initiative Bonus is lost.

\subsubsection[Flanking: Gain +2 to attack]{Flanking}\label{flank}

Attacks from someone's anterior side gain a +2 Bonus.
Up to 6 opponents can attack a lone character, and any available walls can reduce this number.

\subsubsection{Grabbing \& Grappling}\index{Combat!Grappling}
\label{grappling}

\paragraph[Grabs: Make an attack without any weapon bonus. Both combatants are \textit{Entangled}. Cost: 4 Initiative]{Grabs:}

A grapple always starts with a grab.  A grab is a normal roll, made without any benefits from weapons.  If successful, the character has grabbed an opponent.

Once two people are grappling, neither can move and so both can be struck as per a Sneak Attack by anyone nearby.

No weapons can be used while grappling if they have a \gls{weightrating} above -2.

\paragraph[Grapple: Make an opposted roll of Strength x 2 + Combat.  Success means the combatant can either break free or inflict Damage.  Cost: 4 Initiative]{Grapples:}

Once two people are caught in the grapple, either can make a grappling roll at the cost of 4 Initiative.  They can then roll with double their Strength, plus their Strike factor, against 7 plus the enemy's Evasion score.

A successful roll implies the character can break the grapple and move freely, or can inflict $1D6$ plus their Strength Bonus in Damage.

\paragraph[Weapon Grab: Make a normal grab attack, ignoring the opponent's weapon bonus to Evasion.]{Weapon Grab}

This works exactly like a normal grab, except for two key differences.
The first is that the defending player cannot use the weapon's Dexterity Bonus to defend -- a sword which grants a +3 bonus to defend does not help when the sword itself is being grabbed.
The second difference is that a grappled target can simply drop the weapon at any point in order to ignore the grapple.
If a fighter's shield has been grabbed, they can just let it go, and the same with any sword.

\subsubsection[Guard: Someone must successfully hit you before they are allowed to hit whomever you are guarding. Cost: 2 Initiative]{Guarding}\index{Guarding}

If you guard someone by standing in front of them then all attacks have to go through you first.\footnote{This includes missile attacks only if you could otherwise evade them.}
Any enemy making a successful attack on you can choose to damage you, or to make another roll (as a free action, costing no Initiative) at their real target.

Guarding costs 2 Initiative, and after than any defence incurs the usual Initiative cost.
If either character moves away from the other, the guarding stops.

\subsubsection[Half Swording: Add your sword's Speed Bonus to its Damage. Cost: 2 Initiative]{Half Swording}
\index{Combat!Half Swording}

It is possible to hold a sword by the blade and use the guard to bludgeon one's opponent. This manoeuvre allows the weapon's Speed Bonus to be added to its Damage instead. It takes 2 Initiative points to change how one holds the sword.

\subsubsection[Hold Off: Just don't take your turn. Gain 1 Initiative when you decide to step in]{Holding Off}
\index{Combat!Holding Off}

Anyone can wait to see what the battle brings -- the character simply lowers their Initiative and can jump in at any point, acting at one Initiative higher than a declared action.

For example, someone might hold off their action at Initiative 5. They wait for the enemy to attack at Initiative 3 and notices that one of them is attempting to use a magical item. Immediately they retroactively performs an action at Initiative 4.

\subsubsection[Keep Edgy: Look out for missiles (allows you to defend against them with Speed Bonus). Cost: 2 Initiative]{Keeping Edgy}
\label{edgy}
\index{Combat!Keeping Edgy}

The character can take a moment to note their \index{Dodge!Long-range}long-range surroundings, including archers and potential spell casters.
This takes only 2 Initiative points and for the rest of the \gls{round}, any time the character is being fired upon in combat they can use their basic Speed Bonus in a resisted action to leap out of the way of an incoming missile or targeted spell, such as a fireball.
Spells which simply target people by gaze or magical effects such as Polymorphing are unaffected.

\subsubsection[Ram: Push the enemy back 2 squares plus the difference between your Strength Bonuses. Resisting costs 3 Initiative, and requires a resisted Strength + Combat roll. Cost: 3 Initiative]{Ram}
\index{Combat!Ram}
\label{ram}

In combat, it is possible to scare, push and stab at someone to force them to move backwards.
The attacker spends 3 Initiative points.
The defender can either attempt to resist, or can simply acquiesce and move back.
When moving back, targets are pushed back 2 squares; the attacker's Strength adds to this and the opponent's Strength decreases it.
Characters can sacrifice the use of 1 point of Strength to push back an additional person.

Those who resist must also sacrifice 3 Initiative. A resisted Strength + Combat Skill check is made. Successful resistance means that the defender is not pushed back.

A \textit{Ram} action must employ normal movement, and cannot move any character farther than their normal movement.  Characters who have been rammed but are unable to move far enough back fall \textit{prone}.\footnote{See page \pageref{prone} for details on falling prone.}

\iftoggle{verbose}{
  \begin{exampletext}

	Arneson decides he is going to \textit{ram} the crowd of hobgoblins to save his friend.
	Pushing the half dozen hobgoblins back is going to be tricky.
	He launches himself from the stony step they are on, pushes his chest into one then grabs two more hobgoblins.
	Since he is pushing back 2 extra figures, he takes a -2 penalty to the action.
	Arneson's rolling with +2 from his Strength Bonus and +1 bonus for his Combat Skill for a grand total of +1.
	The \gls{tn} is 7 plus the hobgoblins' Strength of +2 and Combat Skill of +2 for a total of 11.
	The \gls{gm} allows him a total of a +2 bonus for jumping off the step.
	The dice come up with an 8 and his total of +3 just passes the test.
	Normally, he would only push the hobgoblins back by 1 step, but they are on the side of a cliff and being pushed onto their back feet.

	The \gls{gm} decides some sort of check is in order to see how well the hobgoblins perform.
	Ordinarily, she would roll for each of them but there are six of them and that will take too long.
	Thinking quickly - because who wants to slow down combat? - she decides that all of them could potentially fall down the cliff since the first three are in front of the next three so Arneson is pushing against all of them one way or another.
	She gives them a \gls{tn} of 9 to stay up and a bonus of +6 because there are 6 of them.
	Each Margin they roll in the final score is one hobgoblin that has not fallen over.
	Dice clatter, she has rolled a `4' and that leaves a final score of 10.
	Everyone falls down the mountain's steep incline except for a single hobgoblin.

	Sean, on Initiative 1, is the last to act.
	He jumps off the cliff-side to attack the last hobgoblin.
	He strikes with a score of 11, bypassing the ugly creature's Partial chain armour, then rolls $1D6+2$ for the Damage for a total of 4.
	The creature is reduced to half its \glspl{hp} with a crimson gash across its throat.

	As Sean's sword swooped down it opened up his target's arm. The last one standing cries out and withdraws his arm then backs off.

	``End of the \gls{round}!'', cries the \gls{gm}. ``Round two! Roll for Initiative''.
\end{exampletext}


}{}

\subsubsection[Sneak Attack: +4 to attack and +2 Damage. Surprised enemies cannot use their Evasion Bonus. Weapon's \glsentrytext{weightrating} creates a penalty to attack]{Sneak Attacks}
\label{sneakattack}
\index{Combat!Sneak Attack}

When taking someone by surprise, the attacker gains a +4 bonus to the attack and a +2 bonus to Damage. Opponents cannot use any Evasion bonuses from Dexterity, weapon Bonuses or the Combat Skill.

Sneak Attacks also gain a penalty equal to the weapon's \gls{weightrating} (if positive).
Warhammers are not the best choice for assassination weapons, while daggers and hand axes do much better.

\subsubsection[Two Weapons: both weapons count as having +2 \glsentrytext{weightrating} when used in one hand, but either weapon can be used at any time, and the weapons Evasion Bonuses stack (the second counts for only half)]{Two Weapon Combat}\index{Combat!Two Weapons}

A character using two weapons -- perhaps a shield in one hand and a sword in the other -- can use either weapon to attack, or gain Initiative.
The Evasion bonuses from the weapons stack, so the second weapon adds \textit{half} its Evasion bonus.
Each weapon will have to be held in one hand, increasing its \gls{weightrating} by 2.

\end{multicols}

\stopcontents[Manoeuvres]

\section{Ranged Combat}\index{Ranged Combat}

\begin{multicols}{2}

\newcommand{\projectilesChart}{
\begin{boxtable}[p{.3\textwidth}XXXr]

  \textbf{Projectile} & \textbf{Initiative} & \textbf{Damage} & \textbf{Weight}  & \textbf{Cost} \\\hline

  Crossbow &  3 & $2D6$ & 1/3 &  {20 sp} \\

  Longbow &  0 & varies & -4/-2 &  {10 sp} \\

  Shortbow &  +2 & $1D6-1$ & -5/-3 &  {5 sp} \\

  Throwing knives & +1 & $1D6-1$ & -5/-3 &  {100 cp} \\

\end{boxtable}
}
\iftoggle{verbose}{
  \begin{figure*}[t]
  \projectilesChart
  \end{figure*}
}{
  \begin{figure*}[t]
  \footnotesize
  \projectilesChart
  \end{figure*}
}


\noindent Projectiles have their own \gls{skill} which is bought just like the Combat Skill.
Archers roll to hit with Dexterity + Projectiles, then roll for Damage, just as with Combat.
The \gls{tn} is always 6 plus one for every five full squares away the target is.
Targets 14 squares away would have a \gls{tn} of 8 to hit.
Most targets cannot use any weapons to add to their Evasion Factor (except shields) but can use the Speed Bonus to evade missile attacks if they are on the run or Keeping Edgy.\footnote{See page \pageref{edgy}.}

Just as with weapon combat, a high enough roll can be a Vitals Shot, ignoring all \gls{dr}.

When someone with a bow is attacked, they can use their Combat Skill and Dexterity to Evade as per usual.

\subsection{The Long Bow}\index{Projectiles!Bow}\index{Bows}

Long bows (or `hunting bows') are difficult things to work but well worth it once the archer practices enough.
To pull back the heavy load on a long bow takes 1 \gls{round}, and the arrow flies at the very end of the round.
Each bow has its own Strength rating and anyone without at least that much Strength cannot use the bow; the bows deal $1D6$ +Strength Rating.
So if a bow has a Strength rating of 2 then it deals $1D6+2$ Damage but requires a Strength of 2, at least, to operate.
Having a Strength of 3 will not increase the Damage.

Long bows can be fired for hundreds of yards -- the maximum range is generally more determined by the archer's ability to aim than by the range of the bow.

\subsection{The Short Bow}\index{Projectiles!Short Bow}\index{Short Bow}

A short bow, or `trick bow', is a smaller, lighter thing which can be used by anyone. What it lacks in punch it makes up for in quick draw time. As usual, for every five squares beyond the first two the archer suffers a -1 penalty to hit. The bow takes 4 Initiative points to fire so many shots can be fired in a \gls{round}.

Short bows have a maximum range of 20 squares and deal $1D6-1$ Damage. They often bring down prey by multiple arrows rather than the one.

Firing a short bow requires 4 Initiative points but reloading takes another 2.

\subsection{The Crossbow}\index{Projectiles!Crossbow}
\label{crossbow}
Crossbows can be powerful, but are not easy to reload. They have a basic Damage of $2D6$ though different crossbows vary in quality. Crossbows take a number of \glspl{round} to reload equal to 6 minus the character's Strength score (minimum of 1).
Firing a crossbow takes only 3 Initiative points.

\subsection{Thrown Weapons}\index{Projectiles!Thrown Weapons}

Thrown weapons such as knives, spears or others are typically not great at killing enemies, but they can certainly wound them. They work just as shortbows, but their Damage is the normal weapon Damage -2. Someone with Strength +1 throwing a dagger would deal $1D6$ Damage, while someone with Strength -1 would deal $1D6-2$ Damage.

\subsection{Impromptu Weapons}\index{Projectiles!Impromptu}

Weapons which were never made to be thrown, such as swords, axes, or most knives, receive a -4 penalty to hit for every five squares distance from the target.
Weapons also receive a -2 penalty to Damage.

\iftoggle{verbose}{

  \input{story/12-advanced.tex}
}{}

\end{multicols}

\section{Morale}
\label{morale}
\index{Morale}
\newcommand{\moralechart}{
  \begin{nametable}{Morale Chart}
    Bonus & Situation \\\hline

    +4 & Monsters outnumber characters 3:1. \\

    +2 & Monsters outnumber characters 2:1. \\

    +2 & Character's top Strength Bonus is lower than the monster's.  \\

    -2 & Character's top Strength Bonus is higher than the monster's.  \\

    -2 & Characters outnumber the monsters. \\

    -2 & Monster is wounded. \\

    -1 & Players have displayed awesome magical abilities. \\

      \end{nametable}
}

\begin{multicols}{2}

\noindent
\iftoggle{verbose}{%
  Unsure if your \glspl{npc} want to fight?
  Roll their Combat or Aggression Skill at \gls{tn} 7, plus the modifiers in the Morale Chart.
  This group roll counts for everyone, so if the group roll a total of \gls{tn} 7, but one member is wounded, that member will fail the roll and flee.
  Of course on the next round, this may prompt others to flee, as it changes the proportions of creatures to \glspl{pc}.

  Most combats will end with one side or the other running away -- few troops want to fight to the last man when they could potentially be safe at home by the end of the day.

  The players do not take morale checks -- they decide when it's time to run away by the look of the situation. Usually a good time is when all the \gls{fp} have run out.\footnote{The \glsentrytext{gm} may also wish to cut all Morale checks for any \glspl{npc} with remaining \glsentrytext{fp}.}

  You don't need to create a separate roll for this check -- just use the standard Initiative roll instead.
  For example, you might roll $2D6$ at the start of a round, and hit `8'.
  The monsters gain a +2 bonus for outnumbering the \glspl{pc}, but a -2 penalty because the \glspl{pc} strongest person is stronger than any of them.
  Overall, the result is `9`, so the enemies pass the morale check, stay to fight, and take this as their Initiative roll.
  With their Initiative bonus of +3, their use an Initiative score of `11'.

}{%
  Each \glspl{npc} Initiative roll also counts as a \emph{group} Morale check.
  The roll checks at \gls{tn} 7, plus the \glspl{npc}'s Combat or Aggression Skill (whichever is higher), plus the modifiers in the Morale chart.
  If the \glspl{npc} fail the roll, they attempt to flee, otherwise, they fight.
}

When an enemy flees the scene after a fight has begun, characters still gain full \gls{xp} for the fight, since they still `defeated' the enemy.

\iftoggle{verbose}{

  \input{story/13-morale.tex}
}{}

\end{multicols}

\moralechart

\newcommand{\chasechart}{

  \begin{boxtable}

  Total & Result \\\hline

  11+ & The characters immediately escape their pursuers. \\

  10 & The characters escape their pursuers after travelling through 2 areas. \\

  9 & The characters escape their pursuers after travelling through 3 areas. \\

  8 & The characters are chased through 3 areas and reroll. \\

  7 & The characters are chased through 2 areas and reroll. \\

  6 & The characters are chased through 1 area and reroll. \\

  {\textless}5 & The characters are immediately caught. \\

\end{boxtable}

}

\newcommand{\huntchart}{

  \begin{boxtable}

  Total & Result \\\hline

  10+ & The characters immediately capture their quarry. \\

  9 & The characters chase their quarry through 3 areas, then capture them. \\

  8 & The characters chase their quarry through 2 areas before catching up with them. \\

  7 & The characters chase their quarry through 1 area and then catch up with them. \\

  6 & The characters chase their quarry through 3 areas, then lose them. \\

  5 & The characters chase their quarry through 2 areas before losing them. \\

  4 & The characters chase their quarry through 1 area, then lose them. \\

  {\textless}3 & The characters immediately lose their quarry. \\

\end{boxtable}

}
\section{Chases}\index{Chases}
\label{chases}

\begin{multicols}{2}

\iftoggle{verbose}{
  \begin{figure*}[t]
  \chasechart
  \end{figure*}
}{
  \begin{figure*}[t]
  \footnotesize
  \chasechart
  \end{figure*}
}

\subsection{Fleeing}

Chases form some of the most dramatic scenes in an adventure. When running on an open field without any barriers, everyone simply runs at full speed -- whoever has the highest Speed + Athletics total succeeds in running away or catching up with an opponent.  But when running through marshes, down alleys, climbing up cliffs, or otherwise finding a reason to change direction, \glspl{pc} must roll.

The system is simple -- one player rolls $2D6$ for the group. Each person then modifies this group score. Since the party will probably run at different paces, they have the option of abandoning slower members or slowing down to the pace of the slowest member.

The \gls{tn} is 6 plus the enemy's Speed + Athletics Bonuses.
Failure means the characters are instantly caught, before they are able to run anywhere.
If the players hit the \gls{tn} they manage to run through 1 area while being chased.
For every Marginal point, they run through an additional area.
If the Margin is ever 3 or more then they completely evade the enemy.
If the party obtain less than total success, they and their pursuers both move and must roll again.

The table is a guide to an unaltered roll. In most situations enemy Traits will affect the actual results of such a total by increasing or decreasing the \gls{tn}.

The \gls{gm} is encouraged to give a fast-paced description of fast-moving scenery, hurriedly telling the players about a new area before moving instantly on.
Each area covered holds new opportunities for getting away, or trapping the quarry -- whether that is the players or their prey.

Characters running through forests might encounter a marshy area, a stream, dense thickets, an open plain and then a sudden, steep hill.
Those crossing plains might find a random encounter in their path, then a copse of trees.
Those running up a mountain could find an area of loose rocks where the ground slides away from under their feet, a narrowing path upwards as rocky walls envelop them and then a misty lake covered in low-lying cloud.

Each area covered also inflicts 1 \gls{fatigue} in addition to any for wearing armour or for Encumbrance Points.
These \glspl{fatigue} are applied after every roll rather than waiting until the end of the scene.

Players are encouraged to suggest Skills which might help. While running away from a band of guards, a character could use the Stealth Skill, quickly dipping into an alleyway to hide. When jumping around a busy area of town, the character might leap over a moving cart to gain some headway. Characters can, with \gls{gm} permission, use their Skills to aid an entire group. The Stealth Skill, in particular, might be used to aid the entire party to hide by finding the right spot. The Empathy Skill might be used to quickly convince farmers to hide the characters.

\iftoggle{verbose}{
  \sideBySide{
  Everyone makes a \textit{Group Roll} at \gls{tn} 9 to escape.
  The dice land on a `10', so everyone travels through 2 areas, then escapes.
}{
  Arneson sees his companions sprinting away, and decides it's time for him to leave too, before more hobgoblins surround him.

  The trio sprint up the mountain, entering a slippery area, where the waters have softened the earth and left mosses growing over every rock-face.
  The hobgoblins struggle with keeping, and a steep slope greets them soon after.
  Grabbing onto the sickly, little trees which live this high up, they enter a low cloud, and start to job, then stumble, and finally talk about hiding.
}

}{}

\subsection{Hunting}

Running after prey follows exactly the same rules, but in reverse.
The party roll for catching up with their prey.
As per the previous chart, a near-failure can be worse than a total failure.
With a complete failure, the enemy simply gets away.
With a partial failure, the party run a long way, get very tired, then fail.
Such is life.

\end{multicols}

\huntchart

\section{Further Dangers}

\begin{multicols}{2}

\subsection[Falling Damage]{Falling Damage}

\index{Falling}
Characters who fall from a height suffer 2 Damage per square the character fell.
2 Damage alone converts to $1D6-2$ Damage, while 4 Damage would simply be $1D6$ and so on.
Characters falling straight downward can attempt to mitigate 4 Damage by rolling Dexterity + Athletics at \gls{tn} 9.
Those falling forward and down in an arc can try to roll along the ground to mitigate the Damage; they roll Dexterity + Athletics at \gls{tn} 7 and a successful roll indicates that they reduce incoming Damage by 4.

The maximum Damage someone can suffer from a fall is 18, equating to $4D6+2$.

\iftoggle{verbose}{
  \begin{exampletext}
``That's the end of the scene'', the \gls{gm} says. ``You can each regain 2 \glsentrylongpl{fp}.''

``I've got 10 \glsentrylongpl{fp} in total'', mentions Arneson's player, ``So I'm getting 4. But doesn't this rest period count as a new scene too?''.

``Sure, says the \gls{gm}. ``You can regain four more \glsentrylongpl{fp} for hiding in the tops of the mountains.''

With their \glspl{fp} now replenishing quickly, the group can rest and worry less about being hit again.

``Oh! I've been forgetting about the Fatigue'', says the \gls{gm}.
Your \gls{gm} will probably say the same at some point.

``Everyone got 1 \glsentryfull{fp} from being in one \gls{round} of combat, then three more for running through three areas, so that's four in total.
Once you rest for the scene, you should be fine.

\vspace{1em}

\textbf{\Glsentrylongpl{hp} and Fatigue}

\begin{tabularx}{\linewidth}{cccccccccc}

\Repeat{7}{\statDot & }
\Repeat{2}{\statCircle & }
\statCircle \\

\Repeat{4}{\sqr & }
\Repeat{5}{\sqn & }
\sqn \\

&&&&&&& -1 & -2 & -3 \\

\end{tabularx}

\vspace{0.4em}
Unfortunately, Arneson lost 5 \glspl{hp} during the fight, so with only 3 \glspl{hp} left, he only heals 2 \gls{fatigue}.

\vspace{1em}
\begin{tabularx}{\linewidth}{cccccccccc}

\Repeat{7}{\statDot & }
\Repeat{2}{\statCircle & }
\statCircle \\

\Repeat{2}{\sqr & }
\Repeat{7}{\sqn & }
\sqn \\

&&&&&&& -1 & -2 & -3 \\

\end{tabularx}

\end{exampletext}

}{}

\end{multicols}

\section{Advanced Combat}
\label{divide_combat}

\begin{multicols}{2}

\noindent
\iftoggle{verbose}{%
  Characters can focus on different parts of combat -- perhaps attempting to strike quickly, to hit the enemy, or to keep themself safe.

}{}

The Combat Skill can be added piece by piece to any of the Combat Factors.
Those with Combat +1 can put it on Strike, Evasion or Initiative.
Those with Combat +2 allows you to place +1 on Strike and +1 on Evasion, or +2 on Initiative, or any other combination.

The character sheet has a space for coins on top of the Combat Factors so you can place your Dexterity Bonus and the Combat Skill on top to remember what you have.

At the end of the round, the Combat Factors reset, and everyone chooses what they want to do again.

\iftoggle{verbose}{

In all cases there is an optimal configuration which will itself depend upon the enemy's placement of resources.\footnote{Players and \glsentrylongpl{gm} are free to cover their coins with their hand until everyone has placed their resources for the round.}

}{}

\iftoggle{verbose}{
  \toppic{Roch_Hercka/stances}{\label{roch:stances}}
}{}

\end{multicols}

\iftoggle{verbose}{

\pagebreak

\section{Combat Summary}\index{Combat Summary}

\begin{multicols}{2}

\begin{enumerate}
  \item{The \gls{gm} rolls for enemy Morale if appropriate.}
  \item{If using advanced rules, each character divides the Combat score (if any) between Initiative, Strike and Evasion. Page \pageref{divide_combat}.}
  \item{One Initiative score is rolled for \glspl{pc} and \glspl{npc}. Page \pageref{initiative}.}
  \begin{itemize}

    \item{Each player adds their \gls{pc}'s own Initiative Bonus to make their own Initiative score.}
  \end{itemize}
  \item{Actions are resolved in order of Initiative, each reducing the Initiative score.}
  \begin{itemize}
    \item{Attacking with a medium weapon costs 6 Initiative.}
    \item{Attacking with a small weapon costs 4 Initiative.}
    \item{Defence costs 2 Initiative.}
    \item{Guarding another character costs 2 Initiative.}
    \item{Movement costs 2 Initiative.}
    \item{Speaking costs 2 Initiative.}
    \item{Ramming into someone costs 3 Initiative.}
  \end{itemize}
\end{enumerate}

\end{multicols}

}{}

\stopcontents[Combat]
