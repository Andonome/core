\chapter{The Rules}
\label{coreRules}

\settoggle{bestiarychapter}{true}

\section{Basic Actions}
\label{basicaction}

\begin{multicols}{2}

\newcommand{\TNChart}{

  \begin{rollchart}

    \textbf{\glsentrytext{tn}} & \textbf{Task} \\\hline

    2 & Automatic \\

    4 & Trivial \\

    6 & Easy \\

    8 & Serious \\

    10 & Professional \\

    12 & Specialist \\

    14 & Extreme \\

    16 & Epic \\

    18 & Legendary \\

    20 & Implausible \\

  \end{rollchart}

}

\sidebox{
  \TNChart
}

\noindent
When \glspl{pc} attempt something dangerous and difficult, the \gls{gm} states the \gls{tn}, and the players try to beat it by rolling $2D6$ plus any bonuses.

\begin{itemize}

  \item
  If the player rolls above the \gls{tn}, they succeed!
  \item
  If they roll below, they fail, and the danger occurs.
  \item
  If they roll equal to the \gls{tn}, the \gls{gm} can give them the choice to succeed, as long as they accept the danger.

\end{itemize}

\begin{exampletext}
  Hugi listens carefully at the keyhole, trying to figure out what the elves on the other side are plotting.
  The \gls{tn} is 10, and he matches it exactly.

  ``One of the elves sounds like he's wandering closer to the door, but you think you almost heard a familiar word'', the \gls{gm} says.

  ``I'll stay and listen'', Hugi's player says.

  ``They use your name, though you can't understand the elvish -- just your name -- then one opens the door, saying `well here he sits, or stands, I can never tell with dwarves'.
  The other elves stand up quickly.''

\end{exampletext}

A basic action is performed by rolling $2D6$ equal or higher than the \gls{tn} for the action.
The more difficult the action, the higher the \gls{tn}.
Players add their character's Attribute and \gls{skill} to the roll.
Attributes and \glspl{skill} usually go as high as +3, so a +6 bonus is possible, and higher bonuses are possible with knacks other bonuses.

Poor Attributes give a penalty, rather than a bonus.

All actions are assumed to have a \gls{tn} of 7 unless your \gls{gm} states otherwise.
Don't ask -- just roll!

\subsection{One Roll Only}

Players only make one roll per action.

If the player wants to attempt to re-try an action, the result remains the same unless circumstances change.

When many characters are trying to do the same task, a single roll is made, and they all consult the same results.

\begin{exampletext}
\begin{itemize}
  \item
  Everyone wants to kick in the door, the \gls{tn} is 10, and the roll uses Strength + Crafts.
    \begin{itemize}
    \item
    Alicia's player rolls the dice. Her Strength + Crafts Bonus is 0, so she fails.
    \item
    Snowstorm's Strength + Crafts total is +2, so his total is 9, and he also fails.
    \item
    Chatrik's Strength + Crafts total is +4, so her total is 11 -- she succeeds.
    \end{itemize}
  \item
  These look like the famous Catacombs from the national anthem, but how do the words go?
  Did the hero take two lefts and a right, or two rights and a left?
    \begin{itemize}
    \item
    Water is filling the area, so time will be lost if they get muddled.
    \item
    Everyone rolls, and the highest result is Kraal's -- but he only equals the \gls{tn}.
    \item
    Kraal's player decides to guess left, rather let his companions think about the song for another moment.
    They fail, but avoid wasting time.
    \end{itemize}
\end{itemize}

\end{exampletext}

If the party are all attempting the same action, then they only make one roll, while adding different attributes to obtain their individual result.

\subsection{\glsentrytext{restingaction}}\label{restingactions}

Difficult, but safe actions allow players to repeat the same task until they get it right or give up.
In these cases, the player sets one die to a `6', and rolls only one die.

\begin{exampletext}
\begin{itemize}
  \item
  The group want to sneak into a noble's house, and have plenty of time to plan the heist.
  The \gls{gm} says they can wander past without suspicion, but it will take many nights to plan and gather all the information they need, the Bonuses are Intelligence plus Stealth, and the \gls{tn} is 12.
    \begin{itemize}
    \item
    The players accept, and roll a single die, achieving a `4'.
    \item
    With the other die automatically on `6', their roll is `10', and the total is `13' -- a narrow success.
    \end{itemize}
  \item
  When the group find a powerful, but mysterious artefact.
  \begin{itemize}
    \item
    Without any danger reading, the \gls{gm} lets them study it for a month as a \gls{restingaction}.
  \end{itemize}
  \item
  A local noble has little time for the troupe, but they really need to make a proposal he can't refuse.
  \begin{itemize}
    \item
    Unfortunately, the \gls{gm} disallows this as a resting action.
    \item
    After the first failure, the noble tells his servants not to pay them any more attention -- they can no longer succeed via the official channels.
  \end{itemize}

\end{itemize}
\end{exampletext}

\subsection{Teamwork}
\label{teamwork}
\index{Teamwork}
\index{Group actions}

Some tasks lend themselves to working with others. Others can be difficult or impossible to do with companions. Some tasks, such as fleeing or sneaking, do not benefit at all from having a load of friends right behind you.

When acting as a group provides no benefit, one player rolls the dice and the same result counts for everyone.  If that player rolls a 9, then everyone's score is 9 and they add their own bonuses and penalties.

If, on the other hand, working together can benefit a situation, one character takes the lead, and up to three other characters can add up to half their bonus (rounded up).
Two companions with a +3 bonus would add a total of a +2 bonus.


\begin{exampletext}
  Example Team Actions include:

  \begin{itemize}

  \item Getting a broken cart down a hill without damaging it.
  \item Tracking down a local thief in a large city.
  \item Spotting danger in the wild.

  \end{itemize}
\end{exampletext}

\subsubsection{Stacking}
\index{Stacking}
\label{stacking}

In general, whenever you want to see how something stacks, add the second lot as half its usual value.
If two people are pushing with Strength +2, they count as having a total Strength of +3.
If others want to join, add any third items as worth a quarter, then an eighth, and so on.

\begin{exampletext}

Convincing the townsfolk that they need to rebel against the baron, and could easily succeed, the troupe work together on a Charisma + Tactics roll.

\end{exampletext}

\noindent%
\begin{footnotesize}%
\begin{tabularx}{\linewidth}{Y |cccc}
                    & Alicia & Snowstorm & Chatrik & Drake \\
\hline
Charisma + Tactics: & +3     & $+\frac{3}{2}$      & $+\frac{2}{4}$      & $+\frac{1}{8}$    \\
Roll Bonus:         &  +3     &     +2               &       +1             &  0 \\
\hline
Total:      &          +6 & & &\\
\end{tabularx}
\end{footnotesize}

\subsection{Resisted Actions}
\index{Resisted Actions}
\label{resistedactions}

When \glspl{npc} resist the players actions, one side rolls as normal, while the other adds their ability to the \gls{tn}.


\subsection{Margins}
\index{Margins}
\index{Failure Margin}
\label{margin}

If you ever need detail on how well an action went, look at how many points above the \gls{tn} the dice show.
With a \gls{tn} of 12, rolling 14 means a margin of 2.

The \gls{gm} might use a Margin for some variable, for example a bard attempting to charm a crowd into giving him money might gain $2D6$ copper pieces plus the Margin, so if the Margin is 3 then he would get $2D6+3$ copper pieces.
Margins might also be used to gain bonuses on later rolls.
Someone attempting to impress a noble court might roll Charisma with the Tactics Skill; the bigger the Margin the more troops they will be trusted with.

\subsection{What the Dice Mean}

You might think of the dice as representing random chance in the environment. Just how irritated is that person you're trying to question, and how creative is that craftsman feeling today? Dice are never re-rolled for different results on the same action because once the dice have told you what the situation is, the situation stays put.

Such a do-over still suggests initial failure; it just means that the character is trying over and over again until a better result is obtained.
Actions cannot be attempted multiple times with rerolls unless the situation has changed notably.

\end{multicols}

\section{Gold \& Goods}
\label{goods}
\index{Equipment}

\begin{multicols}{2}

\subsection{Money}
\index{Money}

An open ended list of equipment is provided to give a basic idea of costs.
The basic coinage covered here is human coinage, but each culture will use their own currency and exchange rates.
A hundred \glsentryfullpl{cp} is worth 1 \gls{sp}.
10 \glspl{sp} is worth 1 \gls{gp}.

An average villager will make little spare money -- perhaps 10 \glspl{sp} in a year if they bother to save.
Sellswords can expect to make upwards to 10 gold per year if they are hired by a villagemaster.
The average free trader -- a blacksmith or cloth dyer -- can expect to make 5 \gls{sp} in a month.

Prices for weapons are placed next to the weapon in chapter \ref{combat}, page \pageref{weaponschart}.

\begin{boxtable}[Xcc]

  \textbf{Animal} & & \textbf{Cost} \\\hline

  Dog & & 2 \gls{sp} \\

  Donkey &   &  2 \gls{sp} \\

  Horse &   &  5 \gls{gp} \\

  War Horse &   &  8 \gls{gp} \\

  Leather Barding &   &  1 \gls{gp} \\

  Chain Barding &   &  2 \gls{gp} \\

  Plate Barding &   &  15 \gls{gp} \\

\end{boxtable}

\begin{boxtable}[Xcc]

  \textbf{Buildings} & & \textbf{Cost} \\\hline

  Cottage & &  20 \gls{gp} \\

  Keep & &  1,000 \gls{gp} \\

  Small Castle & &  4,000 \gls{gp} \\

  Medium Castle & &  10,000 \gls{gp} \\

  Large Castle & &  30,000 \gls{gp} \\

\end{boxtable}

\begin{boxtable}[Xcc]
  \index{Clothes}

  \textbf{Clothing} & \textbf{Weight} & \textbf{Cost} \\\hline

  Peasant clothes &  -3 &  50 \gls{cp} \\

  Noble clothes &  -4 &  1 \gls{gp} \\

  Lavish clothes &  -5 &  3 \gls{gp} \\

  \label{warmClothes}
  Warm clothes &  -2 &  5 \gls{sp} \\

\end{boxtable}

\begin{boxtable}[Xcc]

  \textbf{Professional Tools} & \textbf{Weight} & \textbf{Cost} \\\hline

  Bandages & -5 & 2 \gls{cp} \\

  Grappling hook &  -2 &  1 \gls{sp} \\

  Ink bottle &  &  2 \gls{sp} \\

  Lantern &  -2 &  3 \gls{sp} \\

  Lock pick set &   &  10 \gls{sp} \\

  Metallurgy set &  6 &  40 \gls{sp} \\

  Parchment sheet &   &  1 \gls{cp} \\

  Quill &   &  4 \gls{cp} \\

  Rope, 50' &  -1 &  2 \gls{sp} \\

  Rope, silk, 50' &  -4 &  3 \gls{sp} \\

\end{boxtable}

\index{Camping Equipment}
\index{Food}
\index{Rations}

\begin{boxtable}[Xcc]

  \textbf{Travel} & \textbf{Weight} & \textbf{Cost} \\\hline

  Alcohol lantern & -5 & {4 \gls{sp}} \\

  Cart & 10 &  1 \gls{gp} \\

  Camping equipment & 1 & {3 sp} \\

  Iron rations &  -2 &  10 \gls{cp} \\

  Torch & -4 & {8 \gls{cp}} \\

\end{boxtable}

\subsection{Working Beasts}

Animal stats vary, but you can use the below as a go-to standard for working animals.
Quadrupeds can run at double the standard speed when going full pace, so horses can allow a party to travel at far higher speeds than normal.

\settoggle{examplecharacter}{true}
\horse

Characters can significantly improve their average travel-times with a horse, covering 25 miles per day without tiring.

\settoggle{examplecharacter}{true}
\warhorse

War horses aren't much faster than regular horses, but they won't become so easily frightened when danger approaches.

\settoggle{examplecharacter}{true}
\huntingdog[\npc{\A}{Hunting Dog}]

Hunting dogs are mostly useless in warfare, but they make excellent watchmen.

\pic{Roch_Hercka/dwarf_encumbrance}

\begin{figure*}[t!]
  \settoggle{examplecharacter}{false}
  \weaponsChart
\end{figure*}

\subsection{Weight \& Encumbrance}
\index{Weight}
\index{Encumbrance}
\label{weightrating}

We measure weight in broad terms.
Characters have a \glsentryname{weightrating} equal to their \glspl{hp}, so elves tend to have 5, while humans tend to have a \gls{weightrating} of 7.
Items work similarly, with \gls{weightrating} between -4 (for very light items), through +11 (for wardrobes, carts, and boulders), and so on.

If an item's \glsentryname{weightrating} is equal or below your character's Strength, you can lift it easily.
However, if the items has a greater \gls{weightrating} than your Strength Bonus, you gain a point of Encumbrance for every increment that item is above your Strength Bonus.
Encumbrance slows you down and makes you tired, detracting from your Speed Bonus, and adding to your \glspl{fatigue} each Scene.

Characters can carry items with a maximum \glsentryname{weightrating} of their Strength Bonus plus 6, so a man with 7 \gls{hp} could only be carried with a Strength Bonus of +1 or greater.
Depending upon the circumstances, the \gls{gm} may allow heavier objects to be dragged or rolled.

Items carried in only one hand count as having +2 to the \gls{weightrating}, so hefting a battle axe in only one hand would mean it has an effective \gls{weightrating} of 5.

Characters cannot carry any item which gives them a -5 Encumbrance rating or higher.
They can, however, drag items with up to a \gls{weightrating} of up to 10 points above their Strength Attribute (rough surfaces can increase the requirement substantially).

\begin{figure*}[t!]
\servicesChart
\end{figure*}

\subsection{Services}
\label{services}

Money can buy you more than things.  In fact, for the right money in a large city, characters can buy a full entourage.  Villages, however, will not admit of the same opportunities.

The costs above show the starting price for a few services, plus additional fees for the details.
For example, hiring a guide for an uncharted and dangerous area for 5 days would cost 800 \glspl{cp}.

\iftoggle{aif}{
  The \gls{guard} regularly form an exception to these service charges, as people expect them to work wherever \gls{king} sends them, so they regularly undercharge or simply refuse to work.
}{}

Hiring someone generally requires a Wits + Empathy roll, \gls{tn} 7, to determine their capability.
Failure means that this person is useless.
Perhaps they want to work with you because they have no idea how bad they are at their job, or perhaps they simply want to rip you off.

The Failure Margin should indicate just how bad the henchman is, so the \gls{gm} is encouraged to make the roll in secret.

\subsection{Cultures \& Exchange Rates}
\index{Exchange Rates}

Different cultures have different exchange rates -- the elven versions of standard equipment are always artistically engraved and in high demand; the elves also value the coinage and materials of outsiders very little, so they will not part with their items for human or dwarvish gold easily.
As a result, their --- and other --- culture's items are more expensive than human items.
\sidebox{
  \begin{rollchart}
    Race & Multiplier \\\hline

    Elves & $\times 3$ \\

    Dwarves & $\times 2$ \\

    Gnomes & $\times 2$ \\

    Gnolls & $\frac{1}{2}$ \\
  \end{rollchart}
}


The costs of the items here are based on the most common race -- humans.
Other races have a multiplier effect based on how expensive their equipment is.

Different races will also have different items available.
In general, anything of a basic (non adjusted) value of over 2 \gls{sp} will not be available in a village, while towns will not have anything of over 1 \gls{gp} in value.
Characters can only buy expensive, artisan, items in cities.

Services suffer the same adjustments.
The party will not find gnomes willing to guide `a bunch of giants' as easily as the young humans in a village.


\end{multicols}

\section{Time \& Space}

\begin{multicols}{2}

\noindent
This game uses the entirely abstract measurements of the `scene' and `step' for time and space.
They are more compliant to narrative than physics, and form the basis of all movement and actions whenever people start tracking how long something takes and where everyone is.

\subsection{Time as Scenes}
\label{time}

\subsubsection{Rounds}

When everyone wants to talk and act at the same time, time is tracked in \glspl{round}.
This period of time is used almost exclusively while tracking combat.
The \gls{round} itself can then be further divided into \glspl{ap} if you want real detail, but that's covered later.
All that matters is that a \gls{round} is a period of time in which people attempt to hit each other, then another \gls{round} occurs.

\subsubsection{Scenes}

Most of the time, actions will not occur through \glspl{round} but rather scenes. A scene is just any unit of time in which the \glspl{pc} take on a task or two, usually within a single area. We track scenes only because a few game effects occur at the end of each scene -- mostly these are narrative effects such as regaining \glspl{fp}\footnote{See page \pageref{fate_points}.} in order to regain plot-immunity from Damage. The scene lasts until the \gls{gm} says that it's over.

\subsubsection{Day}
\label{daytimes}

We divide days into four parts -- morning, afternoon, evening and night.
These areas are broadly there for rests -- anyone resting for one of these periods can heal \glspl{fatigue}.%
\footnote{\Glspl{fatigue} are covered on page \pageref{fatigue}.}

\paragraph{Travel}
\index{Marching}
\index{Travel}
happens just as fast as players like.
If they want to march 10 miles in a morning, they can -- they have no hard limit.
Of course after 10 miles, they will have 10 \glspl{fatigue} (except humans, who will have 5), and have to stop for a rest.

\subsubsection{The \Glsentrytext{adventure}}

The \gls{adventure} lasts until the current plot-thread is resolved, or some period of `sandboxing' through a world until a proper use of one's time can be found.
At the end of every \gls{adventure}, \gls{downtime} should be called, and all characters should heal all \glspl{hp}, \glspl{mp}, \glspl{fp}, and \glspl{fatigue}.

\subsubsection{\glsentrytext{downtime}}

\Gls{downtime} is when the current stories come to a close and the \glspl{pc} take a rest.
This non-\gls{adventure} period allows the \glspl{pc} to heal, and advance Traits.
It can be weeks, years, or even decades.
The party can declare \gls{downtime} at any point once the \glspl{pc} have reached a safe area, although the \gls{gm} is free to interrupt that \gls{downtime} with events.
Likewise, the \gls{gm} can declare a \gls{downtime} at any point, but the players can interrupt this with personal missions.

\subsubsection{Healing}
\label{healing}
\index{Healing}
Characters heal a quarter their \gls{hp} each week, rounded up.
Once someone receives a serious wound, it's a good time to call for \gls{downtime}.

\subsection{Space as Squares}\index{Space}\index{Squares}\index{Areas}
\label{space}

\subsubsection{Squares}

Space is tracked through \glspl{step}.
A \gls{step} is just any unit of space within the battlefield.
If you are using a battlemap which has squares marked out on it, then those squares are the size of a step, even if those squares happen to look very large or hexagonal.
A step might be ten metres wide as each one covers an entire house when the battlefield is a large town, or it might be just two yards wide when moving through a detailed map of a dungeon.
The precise distances represented do not matter, just so long as they consistently balance one character's ability to run away with another's ability to hit someone with a projectile.

\subsubsection{Areas}

An \gls{area} is just any place which looks different from another.
While traipsing through a small dungeon, each room and cavern entered might be thought of as an \gls{area}.
When gallivanting through open plains one \gls{area} might be a copse of trees, another a lake, and then the next area a village.

\subsubsection{Region}

Regions encompasses a full forest, a town, or a collection of villages.
Each region has its own set of likely encounters, such as tradesmen in the villages, cut-throats in town, and elves in the forest.%
\footnote{If all this looks like a repugnant abstraction, just set a step to a yard, an area to one mile, a \gls{round} to six seconds and a scene to one hour.}

\end{multicols}

\settoggle{bestiarychapter}{false}
