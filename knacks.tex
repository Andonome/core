\chapter[Hall of Knacks]{Knacks}
\label{knacks}
\index{Knacks}

\noindent
Characters can individuate themselves by learning Knacks -- special talents for combat manoeuvres, magic, skills or other abilities.
Most people can pick up a couple of Knacks easily but further Knacks become progressively less intuitive.

\section{Combat Knacks}

\begin{multicols}{2}

\subsubsection{Adrenaline Surge}
\label{adrenalinesurge}

When a player activates Adrenaline Surge, their character gains +1~Strength for a single action.

Characters can only activate Adrenaline Surge a number of times equal to the number of Knacks they have, and never more than once per round.
This limit resets at the end of each \gls{interval}.

\subsubsection{Berserker}

The character can enter a bloodthirsty rage when in battle.
On the second round, they gain a +1 Bonus to \glspl{ap}.
On the third round, they gain a +1 Bonus to Damage.

They lose the Bonuses if they spend a round without attacking.

\subsubsection{Brawler}

The character's brawling actions receive a bonus to Attack and Damage equal to half the number of Knacks they have, rounded up.

\subsubsection{Cutting Swing}

The character can cut through more than one opponent at a time, or slice open multiple skulls with a single arc of metal.
Any time the character reduces an opponent to 0~\glspl{hp}, the remaining damage transfers to any other opponent in range as the weapon slices across multiple throats, stomachs, and limbs.
This process recurs until either no damage or no enemies remain.

The same attack roll can be recycled for the secondary target, so if a character rolls `11' to attack, then the next opponent receives this strike only if the roll of `11' would hit them.

This Knack can only be used with missile weapons if enemies are standing in a direct line.

\subsubsection{Dodger}
\label{dodger}

The character is an expert at dodging long-ranged attacks and gains a +1~Bonus for each Knack they have at all times.

This Knack grants immunity to all Sneak Attacks from Ranged weapons, such as bows or throwing knives, just as long as the user knows an attack might be coming.

\subsubsection{Fast Charge}

When the character makes an uninterrupted movement of at least 6~\glspl{step}, they gain a Bonus to Strength, and Dexterity equal to half the number of Knacks they have (rounded up), for a single attack.

The character can spend multiple \glspl{ap} to move, as long as they don't do anything else with their \glspl{ap} before making an attack.

\subsubsection{Guardian}

The character can defend others at a cost of 0~\glspl{ap}, and gains a +1~Bonus when defending someone from attack (but only when someone attempts to hit the target -- not when they try to hit the defending character).

Those being guarded must be close beside or behind them, as usual.

\subsubsection{Last Stand}

Any time the character loses \glspl{hp} they immediately gain 2~\glspl{ap} plus one per Knack the character has.

The character also gains a number of \glspl{mp} equal to the number of Knacks they have.

Characters can only use this Knack when there is a legitimate grievance.
The character does not gain the Bonus when they have harmed themself.

\subsubsection{Lucky}

The character has an additional 4~\glspl{fp}.

\subsubsection{Mighty Draw}

The character can pull back a longbow at lightning speed.
They reduce the \gls{ap} cost by a number equal to half the Knacks they have (rounded up).

Those with a crossbow can reload it one round faster than normal, but the minimum is 1~round.%
\footnote{This would normally be 6 rounds minus the character's Strength score. See page \pageref{crossbow} for more.}

\subsubsection{Perfect Sneak Attack}
\index{Sneak Attack}

Any Sneak Attacks the character completes inflicts an additional +1 Damage for each Knack they have.
Normally, Sneak Attacks inflict +2 Damage, so someone with 3 Knacks would inflict +5 Damage.

\subsubsection{Precise Strike}\label{precisestrike}

The character requires 1 less to achieve a \gls{vitalShot} (see \autopageref{vitals}).

\begin{exampletext}
  For example, when targeting an opponent with an Attack score of +2 and Partial armour, someone would normally require a score of 9 to hit and a score of 12 to make a \gls{vitalShot} which ignores all armour.
  With this Knack they still require a score of 9 to hit but only a score of 11 to make a \gls{vitalShot}.
\end{exampletext}

Characters with this Knack can also bypass Perfect armour by rolling 6 points above the target's \gls{tn}.

\subsubsection{Psycho}

This one doesn't fuck about, and it shows.
Sentient enemies receive a -2 penalty when taking Morale Checks.%
\exRef{judgement}{Judgement}{morale}

\subsubsection{Snap Draw}

The character pays 0~\glspl{ap} to nock an arrow or draw a weapon.

\subsubsection{Stunning Strike}\label{stunningstrike}

The character can declare that they are attempting to stun an opponent.
They then take a -1 Penalty to Attack, but if they successfully hit an opponent, the opponent loses a number of \glspl{ap} equal to half the Damage dealt (not \glspl{hp} lost).

\subsubsection{Unstoppable}

The character is particularly tough and gains +2~\glspl{hp} and immunity to the Knack: Stunning Strike.
They can also carry more, without tiring, just as if they had additional Strength.

All medicine rolls to save the character from death receive a bonus equal to half the number of Knacks the character has, rounded up.

\subsubsection{Weapon Master}

The character has trained long and hard with a particular weapon, such as a longsword, spear, shortbow, or rocks.
They gain +1 to their Attack Bonus when using that weapon.

\end{multicols}

\section{Spellcasting Knacks}

\begin{multicols}{2}

\subsubsection{Autophage}

The caster can draw magic from their own blood, even when they have \gls{mp} available to draw from.
Every \gls{mp} stolen in this way inflicts \pgls{ep}.

These spells always look obvious, as they make the caster's nose bleed, or eyes turn pale.

The \glspl{ep} heal at the normal rate, allowing the caster to regenerate their magical potential in two ways at once.

\subsubsection{Ritual Caster}
\label{ritualCaster}
\index{Rituals, Magical}
\index{Spells!Ritual}

Casters can use this Knack to cast a spell as \pgls{restingaction}.%
\footnote{See \autopageref{restingactions}.}
They use Intelligence as the action's Attribute (rather than the standard, Charisma), and must take one hour per \gls{mp} spent to cast the spell.

Ritual spells can employ two \glspl{boon}, rather than the usual limit of one.
This lets them bolster their spell-casting limitations beyond any other caster.

Any spell the caster knows can be used as a ritual.
Dwarves often embellish these spells by a grand rune-carving ceremony (which results in chiselled letters on a stone surface) or long, slow, chants; humans use rituals that resemble theatre; and elves use meditation or song.

\subsubsection{Snap Caster}
\label{snapCaster}

The character can cast spells with movements, rather than speech.
They can ignore any parts of a spell which demand the caster speak, and instead focus on movement.

Spellcasters using this Knack use Wits, rather than Charisma, as the Attribute to roll with spells.
All spells cast this way cost 1~\gls{ap} less.

A lot of spellcasters rely solely on this type of magic to cast spells, focussing on their insight into grabbing the world's hidden flows, rather than trying to sweet-talk the elements.

\pic{Roch_Hercka/dwarvish_runes}

\subsubsection{Vengeful}

The caster's magic is fuelled by hatred and tenacity.
If the character has 0~\glspl{fp} and loses a single \gls{hp} then they gain +2 to their effective chosen casting Attribute Bonus until the end of the \gls{interval}.
If they have lost half their \gls{hp} then they gain an additional bonus equal to the number of Knacks they have.

For example, a caster with just this Knack might lose 2~\glspl{hp} then gain an effective +2 bonus to casting Fire spells and a +2 bonus to the Damage inflicted by such spells.
When they are later struck again and go down to 1~\gls{hp} then they gain a +3 bonus to such spells and a +3 bonus to Damage.

This Knack can only be used when there is a legitimate grievance.
The mage does not gain the bonus when they have harmed themself.
It lasts only until the end of the \gls{interval} and can reactivate only once the mage has lost further \glspl{hp}.

The Knack might also be used when a \gls{pc} or member of \pgls{characterPool} has been killed.

\end{multicols}

\section{Other Knacks}

\begin{multicols}{2}

\subsubsection{Chosen Enemy}

The character has a burning hatred and fascination for a particular type of creature.
The character gains a -2 penalty when interacting socially with such creatures and a +1 when performing actions such as tracking them, attacking them or intimidating them.

For each Knack the player has, they may select a new chosen enemy, so those with a total of 3 Knacks may select 3 chosen enemies. Those enemies may be chosen at any time, including long after a new Knack has been bought.

Possible enemies include: forest beasts, bandits, underground creatures, undead, goblinoids, nobles, and witches.

The character must have witnessed this creature in combat multiple times, although this can be justified by spending \pgls{storypoint}.%
\exRef{stories}{Stories}{stories}

Chosen enemies never `stack', so an undead forest creature only counts as one chosen enemy.

\subsubsection{Fast Healer}

The character regenerates unusually fast.
Any \gls{interval} which they spend resting allows them to heal 2 additional \glspl{ep} and 1~\gls{mp}.

Armour blocks the \gls{mp} regeneration, as usual.

\subsubsection{Specialist}
\label{specialist}
\index{Specialisation}

Specialists are those who work a particular job all day, and gain exceptional proficiency with that single narrow task, and plenty of competence in their general profession.

When using their specialization, characters can add a +2 Bonus.
They also gain +1 to related rolls.

For example, ambush specialists could focus on nocturnal guerilla raids, awaiting at a nest, or surrounding a campfire.

This Knack would grant a +2 Bonus when using one of these exact specialization, or a +1 Bonus when attempting any kind of ambush, relating the history of ambushes, drawing an ambush, or writing poetry about ambushes.

While a focus on `ambushes' naturally lends itself to the Tactics Skill more than others, specializations can cut across Skill-boundaries, adding to any roll.

Suggested specializations include:

\begin{description}
  \item[Auroch]
  experts might focus on their stampedes, their migration patterns, or their mating cycles.
  \item[History]
  specializations might consist of knowing about particular battles, or knowing everything about a particular set of magical items.
  \item[Metallurgy]
  specializations include brass locks, horse-shoes, donkey-shoes, longswords, and gemstone insets.
\end{description}

These specializations cannot grant Bonuses to martial Skills, such as Brawl, or the Air Sphere.

Even if a particular skill inspired a specialization, it still applies to any skill.
Someone with a specialization in aurochs can use that bonus for tracking them (with Wyldcrafting), using their hide to make armour (assuming they can make armour), and healing them (if they have the Medicine Skill).

A good deal of professionals have this specialization -- in fact you can almost assume that any blacksmith has Crafts, but knows metallurgy better than anything else, and that any serious academic will have some `special interest' in a narrow field.

Bonuses from multiple specializations never stack with each other.

\end{multicols}
