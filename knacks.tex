\chapter[Hall of Knacks]{Knacks}
\label{knacks}
\index{Knacks}

\begin{multicols}{2}

\noindent
Characters can individuate themselves by learning various Knacks -- special talents for combat manoeuvres, magic, skills or other abilities.
Most people can pick up a couple of Knacks easily but further Knacks become progressively less intuitive.

\end{multicols}

\section{Combat Knacks}

\begin{multicols}{2}

\subsubsection{Adrenaline Surge}
\label{adrenalinesurge}

The player can declare that super-human effort is being thrown into an action, and gain +1 Strength for a single action.

Adrenaline surge can be used once each interval for each knack the character has, and no more than once a \gls{round}.

\subsubsection{Berserker}

The character can enter a bloodthirsty rage when in battle.
On the second round, they gain a +1 bonus to \glspl{ap}.
On the third round, they gain a +1 bonus to Damage.

They lose the bonuses if they spend a round without attacking.

\subsubsection{Brawler}

The character's brawling actions receive a bonus to Attack and Damage equal to half the number of Knacks they have, rounded up.

\subsubsection{Cutting Swing}

The character can cut through more than one opponent at a time, or slice open multiple skulls with a single arc of metal.
Any time the character reduces an opponent to 0 \glspl{hp}, the remaining damage transfers to any other opponent in range as the weapon slices across multiple throats, stomachs, and limbs.
This process recurs until either no damage or no enemies remain.

The same attack roll can be recycled for the secondary target, so if a character rolls `11' to attack, then the next opponent receives this strike only if the roll of `11' would hit them.

This knack can only be used with missile weapons if enemies are standing in a direct line.

\subsubsection{Dodger}
\label{dodger}

The character is an expert at dodging long-ranged attacks and gains a +1 Bonus for each knack they have at all times.

This Knack grants immunity to all Sneak Attacks from Ranged weapons, such as bows or throwing knives, just as long as the user knows an attack might be coming.

\subsubsection{Fast Charge}

If the character starts at least 6 steps from an enemy, they can charge them.
They may do nothing but move towards the enemy at their full pace.

If they arrive uninterrupted, they gain a bonus to Strength, Dexterity, and Speed, equal to half the number of Knacks they have (rounded up), for a single, immediate, attack.

\subsubsection{Guardian}

The character can defend others at a cost of 0 \glspl{ap}, and gain a +1 bonus when defending someone from attack (but only when someone attempts to hit the target -- not when they try to hit the defending character).

Those being guarded must be close beside or behind them, as usual.

\subsubsection{Last Stand}

Any time the character loses \glspl{hp} they immediately gain +2 \glspl{ap} plus one per Knack the character has.

The character also gains a number of \gls{mp} equal to the number of Knacks they have.

This Knack can only be used when there is a legitimate grievance.
The character does not gain the bonus when they have harmed themself.

\subsubsection{Lucky}

The character has an additional 4 \glspl{fp}.

\subsubsection{Mighty Draw}

The character can pull back a longbow at lightning speed.
They reduce the \gls{ap} cost by a number equal to half the Knacks they have (rounded up).

Those with a crossbow can reload it one round faster than normal, but the minimum is 1 round.%
\footnote{This would normally be 6 rounds minus the character's Strength score. See page \pageref{crossbow} for more.}

\subsubsection{Perfect Sneak Attack}

Any Sneak Attacks the character completes inflicts an additional +1 Damage for each Knack they have.
Normally, Sneak Attacks inflict +2 Damage, so someone with 3 Knacks would inflict +5 Damage.

\subsubsection{Precise Strike}\label{precisestrike}

The character requires 1 less to achieve a Vitals Shot (see page \pageref{vitals}).
For example, when targeting an opponent with an Attack score of +2 and Partial armour, someone would normally require a score of 9 to hit and a score of 12 to make a Vitals Shot which ignores all armour.
With this Knack they still require a score of 9 to hit but only a score of 11 to make a Vitals Shot.
People with this Knack can also bypass Perfect armour by rolling 6 points above the target's \gls{tn}.

\subsubsection{Snap Shot}

The character pays 0 \glspl{ap} to reload an arrow onto a bow or draw a weapon.

\subsubsection{Stunning Strike}\label{stunningstrike}

The character can declare that they are attempting to stun an opponent.
They then take a -1 penalty to Attack, but if they successfully hit an opponent, the opponent loses a number of \glspl{ap} equal to half the Damage dealt (not \glspl{hp} lost).

\subsubsection{Unstoppable}

The character is particularly tough and gains +2 \glspl{hp} and immunity to the Knack: Stunning Strike.

All medicine rolls to save the character from death receive a bonus equal to half the number of knacks the character has, rounded up.

\subsubsection{Voice of Wrath}

The character's battle cries and demeanour are particularly fearsome.
Enemies receive a -2 penalty when taking Morale Checks.

\subsubsection{Weapon Master}

The character has trained long and hard with a particular weapon, such as a longsword, spear, shortbow, or rocks.
They gain +1 to their Attack Bonus when using that weapon.

\end{multicols}

\section{Spellcasting Knacks}

\begin{multicols}{2}

\subsubsection{Autophage}

The caster can draw magic from their own blood, even when they have \gls{mp} available to draw from.
Every \gls{mp} stolen in this way inflicts \pgls{fatigue}.

These spells always look obvious, as they make the caster's nose bleed, or eyes turn pale.

The \glspl{fatigue} heal at the normal rate, allowing the caster to regenerate their magical potential in two ways at once.

\subsubsection{Ritual Caster}

A staple of dwarves and humans -- this knacks allow any caster to use Intelligence as their casting Attribute.
It also allows the caster to spend one more \gls{mp} than usual when casting spells as a resting action.%
\footnote{See \autopageref{restingactions}.}

However, the rituals required to cast spells in this way require long periods of focus -- specifically half an hour per \gls{mp} spent.
Casting as a resting action requires a full hour per \gls{mp} spent.

\subsubsection{Snap Caster}

The character is particularly adept at casting spells quickly, and therefore in Combat.
Spellcasters using this Knack use Wits, rather than Charisma, as the Attribute to roll with spells.
All spells cast this way cost 1 \gls{ap} less.

A lot of spellcasters rely solely on this type of magic to cast spells, focussing on their insight into the world's hidden powers, rather than trying to sweet-talk the elements.

\subsubsection{Vengeful}

The caster's magic is fuelled by hatred and tenacity.
If the character has 0 \gls{fp} and loses a single \gls{hp} then they gain +2 to their effective chosen casting Attribute Bonus until the end of the interval.
If they have lost half their \gls{hp} then they gain an additional bonus equal to the number of Knacks they have.

For example, a caster with just this knack might lose 2 \glspl{hp} then gain an effective +2 bonus to casting Fireball spells and a +2 bonus to the Damage inflicted by such spells.
When they are later struck again and go down to 1 \gls{hp} then they gain a +3 bonus to such spells and a +3 bonus to Damage.

This Knack can only be used when there is a legitimate grievance.
The mage does not gain the bonus when they have harmed themself.
It lasts only until the end of the interval and can reactivate only once the mage has lost further \glspl{hp}.

The Knack might also be used when a member of the party has died, or when someone the character has spent \glspl{storypoint} on has been killed.%
\exRef{stories}{Stories}{stories}

\pic{Roch_Hercka/dwarvish_runes}

\end{multicols}

\section{Other Knacks}

\begin{multicols}{2}

\subsubsection{Chosen Enemy}

The character has a burning hatred for a particular type of creature.
The character gains a -2 penalty when interacting socially with such creatures and a +1 when performing actions such as tracking them, attacking them or intimidating them.

For each Knack the player has, they may select a new chosen enemy, so those with a total of 3 Knacks may select 3 chosen enemies. Those enemies may be chosen at any time, including long after a new Knack as been bought.

Possible enemies include: forest creatures, bandits, magic users, any humanoid race (e.g. dwarves, humans, et c.), underground creatures, undead, nura humanoids, and nura beasts.

Chosen enemies never stack, so an undead forest creature only counts as one chosen enemy.

Characters who wish to swap out a chosen enemy can remove one any time, but can only regain a new one during \gls{downtime}.

\subsubsection{Fast Healer}

The character regenerates unusually fast.
Any interval which they spend resting allows them to heal 2 additional \glspl{fatigue} and 1 \glspl{mp}.

Armour blocks this \gls{mp}, as usual.

\subsubsection{Specialist}
\label{specialist}
\index{Specialisation}

A specialist has exceptional abilities within a fairly narrow environment or domain, and gains a +2 bonus on relevant rolls.
This general Knack allows Academics to specialize in history, for craftsmen to devote themselves to metallurgy, and for performers to achieve exceptional performances with their favoured instrument.

These specializations cannot include combat skills, but can add to rolls for casting spells.
Even if a particular skill inspired a specialization, it still applies to any skill.
Someone with a specialization in aurochs can use that bonus for tracking them (with Wyldcrafting), using their hide to make armour (assuming they can make armour), and healing them (if they have the Medicine Skill).

A good deal of professionals have this specialization -- in fact you can almost assume that any blacksmith has Crafts, but knows metallurgy better than anything else, and that any academic will have some `special interest'.

\end{multicols}
